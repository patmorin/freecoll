\documentclass[twoside,leqno,twocolumn]{article}
\usepackage{ltexpprt}

\newif\ifSODA
\SODAtrue


%%%%%%%%%%%%%%%%%%%%%%%%%%%%%%%%%%%%%%%%%%%%%%%%%%%%%%%%%%%%%%%%%%%%%%%%%%%%%%%%%%%%%%%%%%
%% Packages
%%%%%%%%%%%%%%%%%%%%%%%%%%%%%%%%%%%%%%%%%%%%%%%%%%%%%%%%%%%%%%%%%%%%%%%%%%%%%%%%%%%%%%%%%%
\usepackage{amsmath}
\usepackage{amsfonts}
%\usepackage{amsthm} % in conflict with SIAM style
\usepackage{graphicx}
%\usepackage{xspace}
%\usepackage{wrapfig}
%\newenvironment{wrapfigure}[9]{\begin{figure}}{\end{figure}}
\usepackage{enumerate}
\usepackage{cite}

% \usepackage{pat}

\usepackage[dvipsnames]{color}
\definecolor{linkblue}{named}{MidnightBlue}

\usepackage{amsfonts}
%\usepackage{amsthm}

%\def\@endproof{\outerparskip 0pt\endtrivlist}
%
% FROM AMSTHM.STY
%
\makeatletter
\let\@old@endproof=\@endproof
\def\@endproof{\qed\@old@endproof}
\makeatother
\let\qedhere\relax
\newcommand{\openbox}{\leavevmode
  \hbox to.77778em{%
  \hfil\vrule
  \vbox to.675em{\hrule width.6em\vfil\hrule}%
  \vrule\hfil}}
\providecommand{\qedsymbol}{\openbox}
\DeclareRobustCommand{\qed}{%
  \ifmmode \mathqed
  \else
    \leavevmode\unskip\penalty9999 \hbox{}\nobreak\hfill
    \quad\hbox{\qedsymbol}%
  \fi
}


\newcommand{\comment}[1]{}

\newcommand{\seclabel}[1]{\label{sec:#1}}
\newcommand{\Secref}[1]{Section~\ref{sec:#1}}
\newcommand{\secref}[1]{\mbox{Section~\ref{sec:#1}}}

\newcommand{\tablabel}[1]{\label{tab:#1}}
\newcommand{\Tabref}[1]{Table~\ref{tab:#1}}
\newcommand{\tabref}[1]{Table~\ref{tab:#1}}

\newcommand{\figlabel}[1]{\label{fig:#1}}
\newcommand{\Figref}[1]{Figure~\ref{fig:#1}}
\newcommand{\figref}[1]{\mbox{Figure~\ref{fig:#1}}}

\newcommand{\eqlabel}[1]{\label{eq:#1}}
%\renewcommand{\eqref}[1]{(\ref{eq:#1})}
\newcommand{\myeqref}[1]{(\ref{eq:#1})}
\newcommand{\Eqref}[1]{Equation~(\ref{eq:#1})}

\newtheorem{thm}{Theorem}{\bfseries}{\itshape}
\newcommand{\thmlabel}[1]{\label{thm:#1}}
\newcommand{\thmref}[1]{Theorem~\ref{thm:#1}}

\newtheorem{lem}{Lemma}{\bfseries}{\itshape}
\newcommand{\lemlabel}[1]{\label{lem:#1}}
\newcommand{\lemref}[1]{Lemma~\ref{lem:#1}}

\newtheorem{conj}{Conjecture}{\bfseries}{\itshape}
\newcommand{\conjlabel}[1]{\label{lem:#1}}
\newcommand{\conjref}[1]{Conjecture~\ref{lem:#1}}

\newtheorem{cor}{Corollary}{\bfseries}{\itshape}
\newcommand{\corlabel}[1]{\label{cor:#1}}
\newcommand{\corref}[1]{Corollary~\ref{cor:#1}}

\newtheorem{obs}{Observation}{\bfseries}{\itshape}
\newcommand{\obslabel}[1]{\label{obs:#1}}
\newcommand{\obsref}[1]{Observation~\ref{obs:#1}}

\newtheorem{prop}{Proposition}{\bfseries}{\itshape}
\newcommand{\proplabel}[1]{\label{prop:#1}}
\newcommand{\propref}[1]{Proposition~\ref{prop:#1}}

\newtheorem{clm}{Claim}{\bfseries}{\itshape}
\newcommand{\clmlabel}[1]{\label{clm:#1}}
\newcommand{\clmref}[1]{Claim~\ref{clm:#1}}

\newtheorem{assumption}{Assumption}{\bfseries}{\rm}
\newenvironment{ass}{\begin{assumption}\rm}{\end{assumption}}
\newcommand{\asslabel}[1]{\label{ass:#1}}
\newcommand{\assref}[1]{Assumption~\ref{ass:#1}}

%\theoremstyle{definition}

\newtheorem{defn}{Definition}
\newcommand{\deflabel}[1]{\label{rem:#1}}
\newcommand{\defref}[1]{Definition~\ref{rem:#1}}


\newtheorem{rem}{Remark}
\newcommand{\remlabel}[1]{\label{rem:#1}}
\newcommand{\remref}[1]{Remark~\ref{rem:#1}}

\newtheorem{lesson}{Lesson}
\newcommand{\leslabel}[1]{\label{les:#1}}
\newcommand{\lesref}[1]{Lesson~\ref{les:#1}}

\newtheorem{op}{Open Problem}
\newcommand{\oplabel}[1]{\label{op:#1}}
\newcommand{\opref}[1]{Open Problem~\ref{op:#1}}
\newtheorem{prb}{Problem}{\bfseries}{\rm}

%\theoremstyle{plain}

\newcommand{\etal}{et al.}

\newcommand{\keywords}[1]{\noindent\textbf{Keywords:} #1}
\newcommand{\voronoi}{Vorono\u\i}
\newcommand{\ceil}[1]{{\lceil #1 \rceil}}
\newcommand{\Ceil}[1]{{\left\lceil #1 \right\rceil}}
\newcommand{\floor}[1]{{\lfloor #1 \rfloor}}
\newcommand{\Floor}[1]{{\left\lfloor #1 \right\rfloor}}
\newcommand{\R}{\mathbb{R}}
\newcommand{\N}{\mathbb{N}}
\newcommand{\Z}{\mathbb{Z}}
\newcommand{\Sp}{\mathbb{S}}
\newcommand{\E}{\mathrm{E}}



\usepackage{paralist}
\usepackage[usenames]{xcolor}

\usepackage{hyperref}
\hypersetup{colorlinks=true, linkcolor=linkblue,  anchorcolor=linkblue,
citecolor=linkblue, filecolor=linkblue, menucolor=linkblue,
urlcolor=linkblue, pdfcreator=Me, pdfproducer=Me} 
%\setlength{\parskip}{1ex}
\usepackage{algorithm}
\usepackage{subfig}
%\usepackage{subfigure}
\usepackage{array}
\usepackage[noend]{algpseudocode}
\usepackage{stmaryrd} % for lightning-symbol
\usepackage{mathtools} % for mathclap in \conf-definition
\usepackage{todonotes}
%\usepackage{compress}
%\usepackage{times}
\usepackage{lineno}


\newtheorem{claimx}{Claim}{\bfseries}{\itshape}

%\newcommand{\false}{\texttt{false}}
%\newcommand{\true}{\texttt{true}}
%\newcommand{\remove}[1]{}
%\newcommand{\red}[1]{\textcolor{red}{#1}}

%% FROM PAT FILE 
\newcommand{\reals}{\mathbb{R}}
\newcommand{\integers}{\mathbb{Z}}
\newcommand{\naturals}{\mathbb{N}}
\newcommand{\dist}{{d}}
\newcommand{\Fary}{F\'ary}


\newcommand{\xxx}{\includegraphics[width=.9ex]{figs/sim-crop}}
\newcommand{\yyy}{\includegraphics[width=.9ex]{figs/bar}}
\newcommand{\straight}[1]{\stackrel{\yyy}{#1}}
\newcommand{\squiggle}[1]{\stackrel{\xxx}{#1}}

\DeclareMathOperator{\tv}{\squiggle{\mathit{v}}}
\DeclareMathOperator{\sv}{\straight{\mathit{v}}}

\begin{document}
%\setcounter{chapter}{2} % If you are doing your chapter as chapter one,
%\setcounter{section}{3} % comment these two lines out.



\title{%\MakeUppercase
  {Every Collinear Set in a Planar Graph Is Free}\thanks{%
    The work of VD and PM was partly funded by NSERC.
    The work of FF was 
partially supported by MIUR Project “MODE” under PRIN 20157EFM5C and by 
H2020-MSCA-RISE project 734922, “CONNECT”.
    The work of DG  was 
partly funded by the ANR project GATO, under contract
     ANR-16-CE40-0009.}}

\author{Vida Dujmovi\'c\thanks{Department of Computer Science and Electrical Engineering, University of Ottawa}\and
        Fabrizio Frati\thanks{Dipartimento di Ingegneria, Universit\'a Roma Tre}\and
        Daniel Gon\c{c}alves\thanks{LIRMM, Universit\'e de Montpellier, CNRS}\and
        Pat Morin\thanks{School of Computer Science, Carleton University}\and
 G\"unter Rote\thanks{Institut f\"ur Informatik, Freie Universit\"at Berlin}}

%	\institute{
%		University of Ottawa, Canada \hspace{5mm} \email{vida.dujmovic@uottawa.ca} \and 
%		Roma Tre University, Italy \hspace{5mm} \email{frati@dia.uniroma3.it} \and
%		LIRMM (CNRS \& Universit\'{e} de Montpellier), France \hspace{5mm} \email{daniel.goncalves@lirmm.fr} \and
%		Carleton University, Canada \hspace{5mm} \email{morin@scs.carleton.ca}\and
%		Freie Universit\"at Berlin, Germany \hspace{5mm} \email{rote@inf.fu-berlin.de}} 
%
%
%}

%\author{Corey Gray\thanks{Society for Industrial and Applied Mathematics.} \\
%\and
%Tricia Manning\thanks{Society for Industrial and Applied Mathematics.}}
\date{}

\maketitle

% Copyright Statement
% When submitting your final paper to a SIAM proceedings, it is requested that you include 
% the appropriate copyright in the footer of the paper.  The copyright added should be 
% consistent with the copyright selected on the copyright form submitted with the paper.
% Please note that "20XX" should be changed to the year of the meeting.

% Default Copyright Statement
\fancyfoot[R]{\scriptsize{Copyright \textcopyright\ 2019 by SIAM\\
Unauthorized reproduction of this article is prohibited}}

% Depending on which copyright you agree to when you sign the copyright form, the copyright 
% can be changed to one of the following after commenting out the default copyright statement
% above.

%\fancyfoot[R]{\scriptsize{Copyright \textcopyright\ 20XX\\
%Copyright for this paper is retained by authors}}

%\fancyfoot[R]{\scriptsize{Copyright \textcopyright\ 20XX\\
%Copyright retained by principal author's organization}}


%\pagenumbering{arabic}
%\setcounter{page}{1}%Leave this line commented out.

%\pagenumbering{roman}

%\begin{titlepage}
%\maketitle

\begin{abstract} \ifSODA \small\baselineskip=9pt\fi
  We show that if a planar graph $G$ has a plane straight-line drawing in
  which a subset $S$ of its vertices are collinear, then for any set of
  points, $X$, in the plane with $|X|=|S|$, there is a plane straight-line
  drawing of $G$ in which the vertices in $S$ are mapped to the points
  in $X$.  This solves an open problem posed by Ravsky and Verbitsky in
  2008.  In their terminology, we show that every collinear set is free.

  This result has applications in graph drawing, including untangling,
  column planarity, universal point subsets, and partial simultaneous
  drawings.
  \par
\end{abstract}
%\end{titlepage}

%\tableofcontents

%\newpage
%\pagenumbering{arabic}


%!TEX root = main.tex


\section{Introduction}

%A \emph{plane straight-line embedding} of a planar graph $G$ is a
%geometric representation of $G$ where vertices of $G$ are represented
%as a set of points in the plane and each pair of adjacent vertices
%$\{v,w\}$ is connected by a line segment $\overline{vw}$ that
%intersects only $v$ and $w$ and no other edge or vertex in $G$. 

A \emph{straight-line embedding} of a graph $G$ maps each vertex to a point in the plane and each edge to a line segment between its endpoints. A straight-line
embedding is \emph{plane} if no pair of edges cross other than at a
common endpoint. A set
of vertices  $S\subseteq V(G)$ in a planar graph $G$ is a \emph{free
  set} if for any set of points $X$ in the plane with $|X|=|S|$, $G$ has a plane
straight-line embedding in which the vertices of $S$ are mapped to the points in $X$.  Free sets are useful tools in graph drawing
and related areas and have been used to settle problems in untangling~\cite{bose.dujmovic.ea:polynomial,dalozzo.dujmovic.ea:drawing,dujmovic:utility,ravsky.verbitsky:on,ravsky.verbitsky:on-arxiv}, column planarity~\cite{dalozzo.dujmovic.ea:drawing,dujmovic:utility}, universal point subsets~\cite{dalozzo.dujmovic.ea:drawing,dujmovic:utility},
and partial simultaneous geometric embeddings~\cite{dujmovic:utility}.


 A set of vertices  $S\subseteq V(G)$ in a planar graph $G$ is a
 \emph{collinear set} if $G$ has a plane straight-line embedding in
 which all vertices in $S$ are mapped to a single line.  A collinear set $S$
is a \emph{free collinear set} if, for any collinear set of points in
the plane $X$ with $|X|=|S|$, $G$ has a plane straight-line embedding in
which the vertices of $S$ are mapped the points in $X$.  
Ravsky and Verbistky \cite{ravsky.verbitsky:on,ravsky.verbitsky:on-arxiv}
define $\sv(G)$ and $\tv(G)$ as the respective sizes of the
largest collinear set and largest free collinear set in $G$, and ask
the following question:
\begin{quote}
	How far or close are parameters $\tv(G)$ and $\sv(G)$? It
	seems that \emph{a priori} we even cannot exclude equality. To clarify
	this question, it would be helpful to (dis)prove that every collinear
	set in any straight-line drawing is free.
\end{quote}
%
Here, we answer this question by proving that, for every planar graph $G$,
$\tv(G)=\sv(G)$, that is:

\begin{thm}\thmlabel{our-bang}
Every collinear set is a free collinear set. 
\end{thm}

Let $v(G)$ denote the largest free set for a planar graph $G$. Clearly, we have $v(G)\leq \tv(G) \leq \sv(G)$. Further, as discussed in detail below, it is well-known that $v(G)=\tv(G)$. However, prior to our work, the best known bound between $v(G)$,
$\tv(G)$, and $\sv(G)$ in the other direction was $v(G),\tv(G) \geq \sqrt{\sv(G)}$, proved by Ravsky and Verbitsky~\cite{ravsky.verbitsky:on}. 
Thanks to \thmref{our-bang}, we now know a stronger bound, in fact the ultimate $v(G)=
\tv(G) = \sv(G)$ relationship. This relationship was
previously only known for planar $3$-trees
\cite{dalozzo.dujmovic.ea:drawing}. \thmref{our-bang}, in fact, implies a stronger result than $v(G)= \tv(G) = \sv(G)$:


%Ravsky and Verbitsky
%\cite{ravsky.verbitsky:on} showed earlier that $2$-trees have large, $\Omega(n)$,
%free-colinear sets, however their result did not give
%\thmref{our-bang} for $2$-trees. 




%Before
%\thmref{our-bang}, the following relationships were known between
%$\sv(G)$, $\tv(G)$ and $v(G)$, starting with the obvious inequality:
%$v(G)\leq \tv(G) \leq \sv(G)$. It was
%also known that $v(G)=\tv(G)$, as discussed in detail below. However, in
%the other direction, the best known bound between $v(G)$,
%$\tv(G)$ and $\sv(G)$ was $\tv(G) \geq \sqrt{\sv(G)}$ and thus $v(G)\geq
%\sqrt{\sv(G)}$ as proved by Ravsky and Verbitsky~\cite{ravsky.verbitsky:on}.
%%as implied by Theorem 2 in \cite{dujmovic:utility}. 
%This bound, $v(G)\in \Omega(\sqrt{\sv(G)})$, was not strong enough for
%any novel results in the graph drawing applications of free sets. 
%Thanks to \thmref{our-bang}, we now know a much better and more useful bound, in fact the ultimate $v(G)=
%\tv(G) = \sv(G)$ relationship. This relationship was
%previously only known for planar $3$-trees
%\cite{dalozzo.dujmovic.ea:drawing}. Ravsky and Verbitsky
%\cite{ravsky.verbitsky:on} showed earlier that $2$-trees have large, $\Omega(n)$,
%free-colinear sets, however their result did not give
%\thmref{our-bang} for $2$-trees. 


\begin{cor}\corlabel{our-all}
For every planar graph $G$ and every $S\subseteq V(G)$, $S$ is a free set if
and only if it is a \mbox{collinear set}.
\end{cor}

%\corref{our-all} is a corollary of  \thmref{our-bang} for the
%following reasons. 
That every free set is a collinear set is immediate. \thmref{our-bang} then implies \corref{our-all} since every free collinear set is also a free set. 
This fact, which implies that $v(G)=\tv(G)$, has been observed by several
authors~\cite{bose.dujmovic.ea:polynomial,dalozzo.dujmovic.ea:drawing,dujmovic:utility,gkossw-upg-09}. To
see this,
let $X=\{(x_1,y_1),\ldots,(x_{|S|},y_{|S|})\}$ be a point set in which
no two points have the same y-coordinate, and let
$X_0=\{(0,y_1),\ldots,(0,y_{|S|})\}$.  By the definition of free
collinear set, $G$ has a plane straight-line embedding $\Gamma_0$ in
which $S$ maps to $X_0$.  Since the set of plane straight-line embeddings of
$G$ is an open set, there exists some $\epsilon >0$ such that $G$ has a
plane straight-line embedding $\Gamma_{\epsilon}$ in which $S$ maps to
$X_\epsilon=\{(\epsilon x_1,y_1),\ldots,(\epsilon x_{|S|},y_{|S|})\}$.
Dividing all the $x$-coordinates of $\Gamma_\epsilon$ by $\epsilon$ then
yields a plane straight-line embedding $\Gamma$ in which $S$ maps to
$X$. 

Thus, \thmref{our-bang} is our main result and this paper is dedicated to
proving it. The following
characterization of collinear sets by Da Lozzo \etal\
\cite{dalozzo.dujmovic.ea:drawing}  is helpful in that goal.

\begin{thm}\cite{dalozzo.dujmovic.ea:drawing} \thmlabel{collinear-set}
	A set $S$ of the vertices of a graph $G$ is a collinear set if and
	only if there is a plane embedding of $G$ and a Jordan curve $C$
	that contains every vertex in $S$, that intersects the interior of
	at least one face of $G$, and whose intersection with
	each edge of $G$ is either empty, a single point, or the entire edge.
\end{thm}

 \thmref{collinear-set} is helpful because it reduces the problem of
finding large collinear sets in a graph $G$ to a topological game in
which one only needs to find a curve that contains many vertices
of $G$.  Indeed, Da Lozzo \etal\ used \thmref{collinear-set} to give
tight lower bounds on the sizes of collinear sets in planar graphs
of treewidth at most 3 and triconnected cubic planar graphs. Despite the conceptual simplification provided by \thmref{collinear-set},
the identification of collinear sets is highly non-trivial:  Mchedlidze
\etal\ \cite{mchedlidze.radermacher.ea:aligned} showed that it is NP-hard to
determine if a given set of vertices in a planar graph is a collinear
set.
%
Nevertheless, \thmref{collinear-set} is a useful tool for finding large 
collinear sets. This in combination with \corref{our-all} gives a useful
tool for finding free sets, which have a wide variety of applications,
as outlined in the next section.


\subsection{Applications and Related Work}

The applicability of \corref{our-all} comes from the fact that a number of graph drawing applications require (large) free sets, whereas finding large collinear sets
is an easier task. Indeed there are planar graphs for which large collinear sets were known to exist, however large free sets were not. Those include 3-connected cubic planar graphs
and planar graphs of treewidth at least $k$.
%
%Free collinear sets have a number of applications in graph drawing and related areas. 
We now review applications of our result. 

%A \emph{geometric graph} is a graph $G$ whose vertices are distinct
%points in the plane (not necessarily in general position) and whose
%edges are straight-line segments between pairs of points.  If the
%underlying combinatorial graph of $G$ belongs to a class of graphs
%$\mathcal K$, then we say that $G$ is a \emph{geometric $\mathcal K$ graph}. 

%\paragraph{Untangling} \cite{bose.dujmovic.ea:polynomial,cano.toth.ea:upper,c-upg-10,dalozzo.dujmovic.ea:drawing,dujmovic:utility,gkossw-upg-09,kpr-upg-11,pt-up-02,ravsky.verbitsky:on,ravsky.verbitsky:on-arxiv}

\paragraph{Untangling.}  Given a straight-line embedding of a planar
graph $G$, possibly with crossings, to \emph{untangle} it means to assign
new locations to some of the vertices of $G$ so that the resulting
straight-line embedding of $G$ is plane. The goal is to do so while
\emph{keeping fixed} (that is, while not changing the location of) as many vertices as possible. In 1998 Watanabe asked if every polygon can be untangled while keeping at least $\varepsilon n$ vertices
fixed, for some $\varepsilon >0$. Pach and Tardos\cite{pt-up-02} answered that question in
the negative by providing an $\mathcal{O}((n\log n)^{2/3})$ upper bound on the
number of fixed vertices. This has almost been  matched by
an 
$\Omega(n^{2/3})$ lower bound by Cibulka~\cite{c-upg-10}. Several papers have studied the untangling
problem~\cite{pt-up-02,cano.toth.ea:upper,c-upg-10,bose.dujmovic.ea:polynomial,gkossw-upg-09, kpr-upg-11,ravsky.verbitsky:on}. Asymptotically tight
bounds are known for paths \cite{c-upg-10}, trees \cite{gkossw-upg-09}, outerplanar graphs
\cite{gkossw-upg-09}, and planar graphs of treewidth two and three \cite{ravsky.verbitsky:on,
  dalozzo.dujmovic.ea:drawing}. For general
planar graphs there is still a large gap. Namely, it is known that every planar graph can be untangled while
keeping $\Omega(n^{0.25})$ vertices fixed
\cite{bose.dujmovic.ea:polynomial} (this answered a 
question by Pach and Tardos \cite{pt-up-02})  and that there are planar graphs
that cannot be untangled while keeping $\Omega(n^{0.4948})$ vertices
fixed \cite{cano.toth.ea:upper}. \thmref{our-bang} can help close this gap, whenever a good bound
on collinear sets is known.  %The connection between untangling and
                              %free sets comes from the following. 
Namely, Bose \etal\cite{bose.dujmovic.ea:polynomial} implicitly and  Ravsky and Verbitsky
\cite{ravsky.verbitsky:on} explicitly, proved that every straight-line
embedding of a planar graph $G$ can be untangled while keeping
$\Omega(\sqrt{|S|})$ vertices fixed, where $S$ is a free set of
$G$. Together with \corref{our-all} this implies that, for untangling, it is enough to
find large collinear sets.

\begin{thm}\thmlabel{our-untang}
Let $S$ be a collinear set of a planar graph $G$. Every straight-line embedding of $G$ can be untangled while keeping $\Omega(\sqrt{|S|})$ vertices fixed.
\end{thm}

Da Lozzo \etal~\cite{dalozzo.dujmovic.ea:drawing}  proved that every 3-connected cubic planar graph has
a $\Omega(n)$ collinear set. Then \thmref{our-untang} implies the
following new result, for which $\Omega(n^{0.25})$ was a previously
  best known \mbox{untangling bound.} 
%  Note that triconnected cubic planar
%graphs form a rich subclass of planar graphs as they are duals of
%(non-trivial) triangulations.

\begin{cor}\corlabel{our-cubic-unt}
Every straight-line embedding of any $n$-vertex triconnected cubic
planar graph can be untangled while 
keeping $\Omega(\sqrt{n})$ vertices fixed. 
\end{cor}

\corref{our-cubic-unt} is almost tight due to the $\mathcal{O}(\sqrt{n\log^3n })$ upper bound for 3-connected cubic planar graphs of diameter $\mathcal{O}(\log n)$ \cite{c-upg-10}. \corref{our-cubic-unt} cannot be extended to all bounded-degree planar graphs (see \cite{dujmovic:utility,DBLP:journals/dm/Owens81} for
reasons why).  Da Lozzo \etal\ also proved that planar graphs of treewidth at least
$k$ have $\Omega(k^2)$-size collinear sets. Together with
\thmref{our-untang}  that implies that 
%
%, the following:
%\begin{cor}\corlabel{our-tw}
every straight-line embedding of an $n$-vertex planar graph of treewidth
at least $k$ can be untangled while keeping $\Omega(k)$ vertices fixed. 
%\end{cor}
%
This gives, for example, a tight $\Theta(\sqrt{n})$
untangling bound for planar graphs of treewidth
$\Theta(\sqrt{n})$.  


%Many of these are outlined by Dujmovi\'c \cite{dujmovic:utility}, who will write the rest of this section\ldots

 \paragraph{Universal Point Subsets.}%~\cite{abehlmmo-ups-12,dalozzo.dujmovic.ea:drawing,dujmovic:utility},

% %A set of points $P$ is \emph{universal} for a set of planar graphs if
% %every graph from the set has a plane straight-line embedding where
% %each of its vertices maps to a distinct point in $P$.   It is known
% %that,  for all large enough $n$,  no universal pointset of size $n$
% %exists for all $n$-vertex planar graphs -- as first proved by de
% %Fraysseix~\etal~\cite{dFPP90}. Currently the best known lower bound
% %on the size of a smallest universal pointset for $n$-vertex planar
% %graphs is $1.235n-o(n)$ \cite{DBLP:journals/ipl/Kurowski04} and the
% %best known upper bound is $n^2/4 - O(n)$
% %\cite{DBLP:journals/jgaa/BannisterCDE14}. 

Closing the gap between $\Omega(n)$ and $\mathcal{O}(n^2)$ on the size of the
smallest \emph{universal point set} (definition omitted) for $n$-vertex planar graphs is a major, extensively studied, and difficult graph
drawing problem, open since $1988$~\cite{deFraysseix:1988:SSS:62212.62254, dFPP90, DBLP:journals/ipl/Kurowski04, DBLP:journals/jgaa/BannisterCDE14}. The interest for this problem motivated the following notion introduced by Angelini~\etal~\cite{abehlmmo-ups-12}. 
A \emph{universal point subset} for  a set $\mathcal{G}$ of $n$-vertex planar graphs is a
set $P$ of $k\leq n$ points in the plane such that, for every
$G\in\mathcal{G}$, there is a plane straight-line
embedding of $G$ in which $k$ vertices of $G$ are mapped to the $k$
points in $P$. Every set of $n$ points in general position is a
universal point subset for the $n$-vertex outerplanar graphs
\cite{GMPP,DBLP:journals/comgeo/Bose02,DBLP:conf/cccg/CastanedaU96};  every
set of $\lceil \frac{n-3}{8}\rceil$ points in the plane is a universal
point subset for the $n$-vertex planar graphs of treewidth at most
three \cite{dalozzo.dujmovic.ea:drawing}; and, every set of $\sqrt{\frac{n}{2}}$ points in the plane is a universal point
subset for the $n$-vertex planar graphs \cite{dujmovic:utility}. Dujmovi\'c~\cite{dujmovic:utility}
  proved that every set of $v(G)$ points in the plane is a universal point subset
  for a planar graph $G$. Together with \corref{our-all} this implies
  that, in order to find large universal point subsets, it is enough to look for large collinear sets.

\begin{thm}\thmlabel{our-subset}
Let $S$ be a collinear set for a graph $G$. Then every set of $|S|$ points in the
plane is a universal point subset for $G$.
\end{thm}

As was the case with untangling, \thmref{our-subset} implies new results
for universal point subsets of 3-connected cubic planar graphs and
treewidth-$k$ planar graphs. In particular, \thmref{our-subset} and the
fact that every triconnected cubic planar graph has a collinear set of
size $\ceil{\frac{n}{4}}$ \cite{dalozzo.dujmovic.ea:drawing} imply the
following asymptotically tight result. The previously best known bound
was $\Omega(\sqrt{n})$ \cite{dujmovic:utility}.

\begin{cor}\corlabel{our-cubic-sub}
Every set of $\ceil{\frac{n}{4}}$ points in the plane  is a universal
point subset for  every $n$-vertex 3-connected cubic
planar graph.
\end{cor}

Similarly, \thmref{our-subset} and the fact that planar graphs of
treewidth at least $k$ have collinear sets of size $ck^2$, for some
constant $c$ \cite{dalozzo.dujmovic.ea:drawing}, imply that every set
of  $c k^2$ points in the plane is a universal point subset for  such
graphs. This gives, for example, an asymptotically tight  $\Theta(n)$
results on the size of the largest universal point subset for planar
graphs of treewidth $\Theta(\sqrt{n})$. 

For similar applications of \thmref{our-bang}
and \corref{our-all}, such as \emph{column
planarity}~\cite{behks-cppsge-17,dalozzo.dujmovic.ea:drawing,dujmovic:utility}
and \emph{partial simultaneous geometric embeddings with and without
mappings}~\cite{behks-cppsge-17,ddlmw-pqp-15,dujmovic:utility} see a
survey by Dujmovi\'c~\cite{dujmovic:utility}.


% Cano \etal\ \cite[Theorem~2]{cano.toth.ea:upper} show that if a Jordan
% curve $C$ intersects each edge of a plane embedding of a graph $G$ in
% at most one point and does not contain any vertex of $G$, then $G$ has
% a straight-line plane embedding in which the edges of $G$ intersected
% by $C$ become line segments that cross the $y$-axis, and these crossings
% occur in the same order.  A restatement of \thmref{collinear-set} that
% we describe as \thmref{dujmovic-frati} in \secref{definitions} gives an
% extension of this result to curves that include vertices of $G$.

\subsection{Proof Outline for \thmref{our-bang}}

We assume w.l.o.g.\ that $G$ is a plane straight-line
embedded graph in which the collinear set $S\subseteq V(G)$ is embedded
on the $y$-axis $Y=\{(0,y):y\in\R\}$. Let
$L=\{(x,y)\in\R^2:x<0\}$ and $R=\{(x,y)\in\R^2: x >0\}$ denote the open
halfplanes to the left and right of $Y$. When talking about the order of points on the $y$-axis $Y$, we are referring to the total order $\prec_Y$ in which $(0,a) \prec_Y (0,b)$ if and only if $a<b$. We assume, furthermore, that we are given $|S|$ distinct $y$-coordinates and the goal is to find another plane straight-line embedding of $G$ in which the vertices in $S$ are embedded on $Y$ with the given $y$-coordinates.

Tutte's Convex Embedding Theorem \cite{tutte:how} allows one to (plane
straight-line) embed an internally 3-connected graph with the vertices
of the outer face embedded on any prescribed convex polygon having the
correct number of vertices.  If the vertices in $S$ induce a path with
both endvertices on the outer face of $G$, then no edge of $G$ crosses
$Y$. Then it is easy to prove that $S$ is a free
collinear set using two applications of \cite{tutte:how}, on (suitable augmentations of) the graphs induced
by $V(G)\cap(L\cup Y)$ and $V(G)\cap(Y\cup R)$.

Thus, the main difficulty comes from edges of $G$ that cross $Y$.
These edges must be embedded so that they cross $Y$ in prescribed
intervals between the prescribed locations of vertices in $S$, and
these intervals may be arbitrarily small.  An extreme version of this
(sub)problem is the one in which $G$ is an embedded graph where every
edge intersects $Y$ in exactly one point (possibly an endpoint) and
the location of each crossing point is prescribed.  The most difficult
instances occur when $G$ is edge-maximal.

In \secref{quadrangulations} we describe these edge-maximal graphs, which
we call A-graphs.  A-graphs are a generalization of quadrangulations in
which every face is either a quadrangle whose every edge intersects $Y$ or a triangle with one vertex in
each of $L$, $Y$, and $R$.  \thmref{a-graph} in this section shows that it
is possible to find a plane straight-line embedding of any A-graph where
the intersections of the embedding with $Y$ occur at prescribed locations.
This is done by showing that a certain system of linear equations has
a solution. This proof involves some linear algebra and some arguments
that use continuity.

In \secref{triangulations} we prove that every collinear set is free.
The technical statement of this result, \thmref{main}, shows a somewhat
stronger result for triangulations that makes it possible not only to
prescribe the locations of vertices on $Y$ but also to nearly prescribe the
points at which edges of the triangulation cross $Y$.  This proof uses combinatorial
reductions that are applied to a triangulation $T$ that either reduce its
size or increase the number of edges that cross $Y$.  When none of these
reductions is applicable to $T$, removing the edges of $T$ that do not
cross $Y$ creates an A-graph, $G$, on which we can apply \thmref{a-graph}.

\secref{definitions}, next, begins our discussion with definitions and results that we use throughout.

%!TEX root = main.tex
\section{%Standard Definitions}%
  Preliminaries}
\seclabel{definitions}

We recall some % begin with fairly
standard definitions. % and then move on to problem-specific definitions and results.

%\subsection{Standard Definitions}

% For a function $f\colon\R\to\R$, we use $\lim_{x\downarrow t} f(x)$ and
% $\lim_{x\uparrow t} f(x)$ to denote the one-sided limits of $f(x)$
% as $x$ approaches $t$ from above and below, respectively.  
%
A \emph{curve} $C$ is a continuous function from $[0,1]$
to $\mathbb{R}^2$.  The points $C(0)$ and $C(1)$ are the \emph{endpoints} of $C$.  A curve $C$ is \emph{simple} if $C(s)\neq C(t)$
for any $0\le s<t< 1$; it is \emph{closed} if $C(0)=C(1)$.  A \emph{Jordan
	curve} $C:[0,1]\to\R^2$ is a simple closed curve.  
%Starting in the current paragraph, 
We will often fail to distinguish between a curve $C$ and its
image $\{C(t):0\le t\le 1\}$.  When this happens, we may qualify the curve as
\emph{open} in which case we are referring to the set $\{C(t):0< t< 1\}$.
A point $x\in\R^2$ is \emph{on} $C$ if $x\in C$.  
%
For any Jordan curve $C$, $\R^2\setminus C$ has two connected
components: One of these, $C^-$, is finite (the {\em interior} of $C$) and the other, $C^+$, is
infinite (the {\em exterior} of $C$).  
%We say that a Jordan curve is \emph{well-oriented} if walking
%along $C$ from $C(0)$ to $C(1)$ results in a counterclockwise traversal
%of the boundary of $C^-$, so that $C^-$ is locally to the left of $C$
%and $C^+$ is locally to the right of $C$. NOT USED?
%
The points on a simple curve $C$ are associated with a partial order $\prec_C$ such that $C(a)\prec_C C(b)$ if and only if $a<b$.  

%For any $0\le a\le b\le 1$, the \emph{subcurve}
%of $C$ between $a$ and $b$ is the curve $C'(t)=C(a+t(b-a))$.  When talking about the subcurve of $C$ between points $x,y\in C$ where $x\prec_C y$, we mean the subcurve of $C$ between the unique $a< b$ with $x=C(a)$ and $y=C(b)$. NOT USED?

All graphs $G$ considered in this paper are finite, simple, and
undirected.   We use $V(G)$ and $E(G)$ to denote the vertex set and edge
set of $G$, respectively.
% For any two vertices $x,y\in V(G)$,
We use $xy$
to denote the edge between the vertices % of $G$ incident to
$x$ and $y$. % We denote by $d_v$ the degree of a vertex $v$ in a graph. 

A \emph{drawing} $\Gamma=(\varphi,\rho)$ of a graph $G$ consists
of a one-to-one mapping $\varphi\colon V(G)\to\R^2$ and a mapping $\rho$ from
$E(G)$ to curves in $\R^2$ such that, for each $xy\in E(G)$, $\rho(xy)$
has endpoints $\varphi(x)$ and $\varphi(y)$.
%Starting immediately,  
We usually say that $G$ is a drawing
without explicitly referring to the pair $\Gamma=(\varphi,\rho)$.
In these cases, we identify vertices of $G$ with their
points and edges of $G$ with their curves. By default, an edge curve includes
its endpoints, otherwise we refer to it as an \emph{open} edge.
%
In a \emph{straight-line drawing} each edge is a straight-line segment. In a \emph{plane drawing} each edge is a simple curve, and no two edges intersect, except possibly at common endpoints. A \emph{F\'ary drawing} is a plane straight-line drawing. For a given F\'ary drawing $G$, we often address the problem of constructing a different F\'ary drawing of $G$, satisfying certain properties; by this we mean that we will construct a F\'ary drawing of the plane graph of which $G$ is a F\'ary drawing.

The \emph{faces} of a plane drawing $G$ are the maximal connected
subsets of $\R^2\setminus\bigcup_{xy\in E(G)} xy$.  One of
these faces, the \emph{outer} face, is unbounded; the other faces are called \emph{inner} or \emph{bounded} faces. A {\em boundary vertex} is incident to the outer face, other vertices are called \emph{interior} vertices.
%A \emph{chord} is an edge of $G$ with two endpoints on the outer face but whose
%interior is not on the outer face.  
%When discussing plane drawings, we use the convention of listing
%the vertices of a face as they appear when traversing the face in
%counterclockwise order. NOT REALLY

A \emph{triangulation} (a \emph{quadrangulation}) is a plane drawing
in which each face is bounded by a 3-cycle (respectively, a 4-cycle). 
%Every quadrangulation has
%$n\ge 4$ vertices and Euler's formula implies that it has $2n-4$ edges. NOT USED?

%A \emph{separating cycle} of a graph $G$ is a sequence of vertices
%that form a cycle and whose removal disconnects $G$.
A \emph{separating triangle} of a graph $G$ is
a cycle of length 3
 whose removal disconnects $G$.
 If
 $C$ is a separating triangle in a plane drawing, then
 % for any separating cycle $C$,
 there is a 
%well-oriented 
Jordan curve whose image
is the union of edges in $C$.  In this case, the interior and exterior of
$C$ refer to the interior and exterior of the corresponding Jordan curve.

The \emph{contraction} of an edge $xy$ in a graph $G$ identifies $x$
and $y$
into a new vertex~$v$.
Formally, we obtain a new graph $G'$ with
$V(G')=V(G)\cup\{v\}\setminus\{x,y\}$ and $E(G')=E(G)\setminus\{ab\in
E(G): \{a,b\}\cap\{x,y\}\neq\emptyset\}\cup\{va: xa\in E(G)\}\cup
\{va:ya\in E(G)\}$.  
%In this case, we say that we \emph{contract} $xy$ in $G$ to obtain the graph $G'$.  EASY TO FIGURE
If $G$ is a triangulation and we
contract the edge $xy\in E(G)$, then the resulting graph $G'$ is also
a triangulation provided that $xy$ is not part of a separating
triangle. %More specifically,
Any plane drawing of $G$ leads naturally
to a plane drawing of $G'$.

\subsection{Characterization of Collinear Sets}


\begin{figure}[htb]
  \centering
  \includegraphics[page=2]{figs/nice-curve}
  \caption{An admissible curve}
  \label{fig:admissible}
\end{figure}

A %well-oriented 
Jordan curve $C$ is \emph{admissible} for a drawing $G$ if $C(0)=C(1)$ is in the outer face
of $G$ and the intersection between $C$ and each edge $e$ of $G$ is
either empty, a single point, or the entire edge $e$.  
%We say that $C$
%is \emph{proper} for $G$ if it is admissible and it does not contain any
%vertex of $G$; i.e., $C$ is proper if its intersection with any edge $e$
%of $G$ is either empty, or in the interior of $e$.

%We say that a Jordan curve $C:[0,1]\to\R^2$ is \emph{nice} for an embedded
%graph $G$ if $C$ intersects each edge of $G$ in at most one connected
%component and the endpoint $C(0)=C(1)$ of $C$ is in the interior of the
%outer face of $G$.  We say that $C$ is \emph{clean} (for $G$) if its
%intersection with each edge of $G$ is either empty or a single point.
%We say that $C$ is \emph{tidy} (for $G$) if it does not contain any
%vertex of $G$.

We will make use of the following restatement of \thmref{collinear-set}
which follows from the proof in \cite{dalozzo.dujmovic.ea:drawing}:
\begin{thm}\thmlabel{dujmovic-frati}
	For any planar graph $G$, the following two statements are equivalent:
	\begin{compactenum}
		\item There exists a plane drawing $\Gamma_1$ of $G$ and an
		admissible curve $C\colon [0,1]\to\R^2$ such that the sequence of edges
		and vertices intersected by $C$ is $r_1,\ldots,r_k$.
		\item $G$ has a \Fary\ drawing $\Gamma_2$ in which the sequence
		of edges and vertices intersected by $Y$ is $r_1,\ldots,r_k$.
	\end{compactenum}
\end{thm}


%%% Local Variables:
%%% mode: latex
%%% TeX-master: "freecoll"
%%% End:

%!TEX root = main.tex
\section{A-Graphs}
\seclabel{quad}
\seclabel{quadrangulations}

In this section, we study a special class of graphs that are closely
related to quadrangulations in which every edge crosses $Y$. (See \figref{a-graph} for an example.)

\begin{defn}\deflabel{a-graph}
	An \emph{A-graph}, $G$, is a \Fary\ embedding of a graph with $n\ge 3$ vertices that has the following properties:
	\begin{compactenum}
		\item Every edge of $G$ intersects $Y$ in exactly one point, possibly an endpoint.
		\item Every face of $G$, including the outer face, is a quadrilateral or a triangle.
		\item Every quadrilateral face of $G$ is non-convex.
		\item Every triangular face contains one vertex in each of $Y$, $L$,
		and $R$.  
		\item Every vertex $v$ on $Y$ is incident to precisely
		two triangular faces, one ``above $v$'' whose interior contains the open line segment with endpoints $v$ and $v+(0,\epsilon)$ for some $\epsilon>0$ and one ``below $v$'' whose interior contains the open line segment with endpoints $v$ and $v-(0,\epsilon)$ for some $\epsilon >0$.
	\end{compactenum}
\end{defn}

%\begin{wrapfigure}[11]{r}{.3\textwidth}
%		\Vspace{-1mm}
\begin{figure}
		\centering{\includegraphics[scale = 0.95]{figs/a-graph-new}}
		\caption{An A-graph with 2 vertices in $Y$.}
		\figlabel{a-graph}
\end{figure}
%	\end{wrapfigure}
	
In the special case where $G$ has no vertices in $Y$, the graph $G$ is a quadrangulation in which every edge crosses $Y$. Further, Property~5 applies even if $v$ is on the outer face of $G$ (in which case it implies that the outer face of $G$ must be a triangle).
Some additional properties of $G$ follow from \defref{a-graph}:
\begin{compactenum}\setcounter{enumi}{5}
	\item $G$ is connected.
	\item Every vertex of $G$ has degree at least 2.   
	\item If $n\ge 4$, then every vertex in $Y$ has degree at least 3. 
\end{compactenum}
Property~6 follows directly from Property~2.
Property~7 follows from the fact that every vertex is incident to at
least one face and every face is a simple cycle.
Property~8 follows from the fact that every vertex on $Y$ is incident
to at least two triangular faces, which involve at least 4 vertices, unless $n=3$.

%We will show that every A-graph $G$ has a \Fary\ embedding with prescribed
%intersections with $Y$ and a prescribed outer face.  Since the outer
%face of an A-graph can be a triangle or a quadrilateral, in the following
%theorem, $\Delta$ is a triangle or quadrilateral defined as follows:
%\begin{enumerate}
%   \item If $(0,y_1)$ is a vertex of $G$, then $\Delta$ is a triangle
%   with one vertex at $(0,y_1)$ and the opposite edge edge crossing $Y$ at $y_m$.
%
%   \item If $(0,y_m)$ is a vertex of $G$, then $\Delta$ is a triangle
%   with one vertex at $(0,y_m)$ and the opposite edge crossing $Y$ at $y_1$.
%
%   \item Otherwise,
%      $\Delta$ is a quadrilateral whose edges cross $Y$ at $y_1$, $y_a$,
%      $y_b$, and $y_m$, where $e_1$, $e_a$, $e_b$, and $e_m$ are the four edges on the outer face of $G$
%\end{enumerate}

We will show that every A-graph $G$ has a \Fary\ embedding with prescribed intersections with $Y$ and a prescribed outer face, as in the following theorem. 

\begin{thm}\thmlabel{a-graph}
	Let
	\begin{compactenum}
		\item $G$ be an A-graph;
		\item $e_1,\ldots,e_m$ be the sequence of edges in $G$,
		in the order they are intersected by $Y$;
		\item $y_1\le\cdots\le y_m$ be any sequence of numbers where, for
		each $i\in\{1,\ldots,m-1\}$, $y_i=y_{i+1}$ if and only if $e_i$
		and $e_{i+1}$ have a common endpoint in $Y$;
		\item $\Delta$ be a triangle or quadrilateral, where:
		\begin{compactenum}
			\item If $(0,y_1)$ is a vertex of $G$, then $\Delta$ is a triangle
			with a vertex at $(0,y_1)$ and the opposite edge crossing $Y$ at $y_m$.
			
			\item If $(0,y_m)$ is a vertex of $G$, then $\Delta$ is a triangle
			with a vertex at $(0,y_m)$ and the opposite edge crossing $Y$ at $y_1$.
			
			\item Otherwise, $\Delta$ is a quadrilateral whose edges cross $Y$ at $y_1$, $y_a$,
			$y_b$, and $y_m$, where $e_1$, $e_a$, $e_b$, and $e_m$ are the four edges on the outer face of $G$.
		\end{compactenum}
	\end{compactenum}
	Then $G$ has a
	\Fary\ embedding in which the outer face is $\Delta$
	and, for each $i\in\{1,\ldots,m\}$, the intersection between $e_i$ and $Y$
	is the single point $(0,y_i)$.
\end{thm}

%%%%% copia da qui %%%%%%
The rest of this section is devoted to prove \thmref{a-graph}. We make some simplifying assumptions. First, we assume w.l.o.g.\ up to a uniform scaling that $\Delta$ and all the vertices of $G$ are contained in $[-1,1]^2$. Second, we assume w.l.o.g.\ up to a reflection  with respect to $Y$ that, if the outer face of $G$ is delimited by a quadrilateral, then the vertex incident to both $e_1$ and $e_m$ is in $L$, as in \figref{a-graph}. In such a case we also assume that the vertex of $\Delta$ incident to both $e_1$ and $e_m$ is in $L$; this is also not a loss of generality, as if the vertex of $\Delta$ incident to both $e_1$ and $e_m$ is in $R$, then $\Delta$ can be reflected with respect to $Y$, obtaining a quadrilateral $\Delta'$ whose vertex incident to both $e_1$ and $e_m$ is in $L$, then a \Fary\ embedding of $G$ can be constructed in which the outer face is $\Delta'$, and finally the \Fary\ embedding can be reflected with respect to $Y$, thus obtaining a \Fary\ embedding of $G$ in which the outer face is $\Delta$. 

If $m=3$ or $m=4$, then $G$ is a 3- or a 4-cycle, respectively, hence it suffices to embed it as $\Delta$. Therefore we assume, from now on, that $m\ge 5$.  
%%%%% copia fino a qui %%%%%%

%	\begin{wrapfigure}[10]{r}{.32\textwidth}
%		\Vspace{-1mm}
\begin{figure}
		\centering{\includegraphics[scale = 0.95]{figs/ab}}
		\caption{The ordering of the edges incident to a vertex $v$ on $Y$.}
		\figlabel{ab}
\end{figure}
%	\end{wrapfigure}
Before continuing, we pause to fully specify the ordering
$e_1,\ldots,e_m$. This ordering is unambiguous except where
some vertex $v\in Y$ is incident to several edges
$e_{i},\ldots,e_{i+d}$, where $d\ge 2$ by Property~8 of A-graphs. Refer to \figref{ab}.  In this case we partition $v$'s neighbors into two
sets $\alpha_1,\ldots,\alpha_k\in L$ and $\beta_1,\ldots,\beta_\ell\in
R$, where $\alpha_1,\ldots,\alpha_k$ are ordered clockwise around $v$
and $\beta_1,\ldots,\beta_\ell$ are ordered counterclockwise.  We then use
the convention that $e_i,\ldots,e_{i+k-1}=v\alpha_1,\ldots,v\alpha_k$
and $e_{i+k},\ldots,e_{i+d}=v\beta_1,\ldots,v\beta_\ell$.

We will describe the desired \Fary\ embedding by assigning a slope
$s_i$ to each edge $e_i\in E(G)$.  
Since there can be no vertical edges, each slope $s_i$ is well-defined.
We have $m=|E(G)|$ slope variables, $s_1,\ldots,s_m$.  Since every edge
$e_i$ contains the point $(0,y_i)$, the slope $s_i$
fixes the line through $e_i$.  Since every vertex $v$ not on $Y$ is
incident to at least two edges that contain distinct points on $Y$,
the location of $v$ is fixed.  (The location of each vertex on $Y$
is fixed by definition.)

A necessary condition for the slopes to determine a F\'ary embedding of $G$ is that the supporting lines of edges 
with a common vertex should be concurrent. Let $v$ be a vertex 
not on $Y$, and let $e_i, e_j, e_k$ be three edges incident to $v$.
The fact that the supporting lines of $e_i$, $e_j$, and $e_k$
meet at a common point (the location of $v$) is expressed by the following
\emph{concurrency constraint} in terms of the slopes $s_i,s_j,s_k$:
\begin{equation}\eqlabel{slope0} 
\left|
\begin{matrix}
1&1&1\\
s_i&s_j&s_k\\
y_i&y_j&y_k
\end{matrix}
\right|=
({y_j-y_k}) s_i + ({y_k-y_i}) s_j 
+ ({y_i-y_j})s_k  = 0
\end{equation}
Since $y_1,\ldots,y_m$ are given, this is a linear equation
in $s_1,\ldots,s_m$.
Writing this equation for all triplets of edges incident to a common
vertex $v$ will include many redundant equations. Indeed, it suffices to take $d_v-2$ equations: We choose two fixed
incident edges $e_i$ and $e_j$ and run $e_k$ through the remaining
$d_v-2$ edges, specifying that $e_k$ should go through the common vertex
of $e_i$ and $e_j$.
%We call the resulting collection of $\sum_{v\in V(G)\setminus Y} d_v-2$ equations the \emph{concurrency constraints}.

Whenever convenient, we will use edges of $G$
as indices so that, if $e=e_i$ is an edge of $G$, then $s_e=s_i$
and $y_e=y_i$.  Further, if $e$ is a line segment that
intersects $Y$ in a point, we will use $y_e$ to denote the $y$-coordinate
of the intersection of $e$ and $Y$ and $s_e$ to denote the slope of
$e$'s supporting line.

%It will be important to have as many equations as variables;
%thus, 

We now introduce additional equations for the edges that emanate from a
vertex on $Y$.
Suppose that a vertex $v\in Y$ is incident to edges $a_1,\ldots,a_k\in L\cup Y$ 
and $b_1,\ldots,b_\ell\in Y\cup R$, ordered from bottom to top as in \figref{ab}.
From Property~4 of A-graphs we have $k,\ell\ge1$ and, from Property~8, we have $k+\ell\ge 3$.
Let us first look at the slopes on the right side.
We want these slopes to be increasing:
$s_{b_1} < s_{b_2} < \dots  <s_{b_\ell}$. We stipulate a stronger
condition:
We require that the slopes
$s_{b_2}, \dots, s_{b_{\ell-1}}$ partition the interval
$[s_{b_1},s_{b_\ell}]$ in fixed proportions. In other words:
\begin{equation}
\label{eq:proportion}
s_{b_i} = s_{b_1} + \lambda_i(s_{b_{\ell}}-s_{b_1}),
\end{equation}
for some fixed sequence $0<\lambda_2<\cdots<\lambda_{\ell-1}<1$.

For example, we might set $\lambda_i := (i-1)/(l-1)$.
This gives $\ell-2$ equations, for $\ell\ge 2$. Similarly, we get
$k-2$ equations for the slopes
$s_{a_1}, \dots, s_{a_{k}}$ of the edges on the left side, for $k\ge 2$.
In addition, for $k\ge 2$ and $\ell\ge 2$, we require that the \emph{range} of
slopes
on the two sides are in a fixed proportion:
\begin{equation}
\label{eq:proportion2}
s_{a_1}-s_{a_{k}} = \mu (s_{b_{\ell}}-s_{b_1}),
\end{equation}
for some fixed value $\mu>0$.

We call the equations
\thetag{\ref{eq:proportion}--\ref{eq:proportion2}} the
\emph{proportionality constraints}.
There are $(k+\ell)-3$ such equations for the $k+\ell$ slopes, hence we have three degrees of freedom for the slopes out of a vertex.
%\figref{proportional} illustrates these  degrees of freedom:
Namely, we can shear the edges on the right side vertically, adding the same constant to all
slopes. We can independently shear all edges on the left side.
In addition, we can vertically scale {all} lines jointly (both to
the left and to the right), multiplying all slopes by the same constant factor.
If this factor is negative, we would reverse the order of the
slopes, simultaneously on the left and on the right. We will later see how to prevent this. We can already observe that two slopes on one side determine all remaining slopes on that side. Moreover, the range of slopes on the other side ($s_{a_1}-s_{a_{k}}$ or $s_{b_{\ell}}-s_{b_1}$) is also determined.
%
The notations $\lambda_i$ and $\mu$ are here used in a local sense;
for a different vertex $v$, we may choose different constants.
%\begin{figure}
%	\centering{\includegraphics{figs/proportional}}
%	\caption{The degrees of freedom provided by the proportionality constraints}
%	\figlabel{proportional}
%\end{figure}
We have the following.
\begin{lem} \label{le:number-of-equations}
The total number of equations \thetag{\ref{eq:slope0}},~\thetag{\ref{eq:proportion}}, and~\thetag{\ref{eq:proportion2}} is $m-4$.
\end{lem} 
\begin{proof}
Let $n=|V|$ and let $n_0$ be the number
of vertices on $Y$. Assume that $G$ has $f_3$ triangular and $f_4$
quadrangular faces.

Two triangles for every vertex on $Y$ (Property 5 of A-graphs):
\begin{equation}
\label{eq:f3}
f_3 = 2n_0
\end{equation}
Euler's formula:
\begin{equation}
\label{eq:Euler}
n + f_3+f_4 = m+2
\end{equation}
Double-counting of edge-face incidences leads to the relation
\begin{equation}
\label{eq:edge-face}
3f_3+4f_4=2m.
\end{equation}
We have $d_v-3$ equations for each of the $n_0$ vertices $v$ on $Y$. For each of the 
$n-n_0$ vertices $v$ not on $Y$, 
we have $d_v-2$ equations.
The total number of equations is therefore
\begin{align*}
% \label{eq:number-equations2}
P &= 
\sum_{v\in V(G)\cap Y}^n(d_v-3)+
\sum_{v\in V(G)\cap(L\cup R)}^n(d_v-2)
=
\sum_{v\in V}^n(d_v-2)-n_0
=
2m-2n-n_0.
\end{align*}
Using \thetag{\ref{eq:f3}--\ref{eq:edge-face}}, this can be
simplified to
\begin{align*}
P&=
2m-2n-n_0\\
&= 2m -2n -2f_3-2f_4 +2f_3+2f_4-n_0\\
&= 2m -2(n +f_3+f_4) +\tfrac12(4f_3+4f_4-f_3)\\
&= 2m -2(m+2) +m = m-4.
\end{align*}
This concludes the proof of the lemma.
\end{proof}

To achieve the desired number, $m$, of equations, we add
four \emph{boundary equations}.  If the outer face is a quadrilateral, we set
the slopes $s_1$, $s_a$, $s_b$, and $s_m$ of its edges $e_1$, $e_a$, $e_b$, and $e_m$ to the fixed values of the slopes of the edges of $\Delta$. If the outer face is a triangle $\alpha\beta\gamma$, with $\gamma\in Y$, we set the slopes $s_1$, $s_a$, and $s_m$ of the boundary edges $e_1$, $e_a$, and $e_m$ to the fixed values of the slopes of the edges of $\Delta$ and we pick another (non-boundary) edge $e_b$ incident to $\gamma$ and set its slope $s_b$ to a fixed value; this value is such that $s_b$ is either larger than each of $s_1$, $s_a$, and $s_m$ or smaller than each of $s_1$, $s_a$, and $s_m$, depending on whether the slope of $e_b$ in $G$ is larger than the slope of each of $e_1$, $e_a$, and $e_m$ or smaller than the slope of each of $e_1$, $e_a$, and $e_m$. Observe that no other cases are possible. Together with the proportionality constraints, this effectively pins all the slopes incident to $\gamma$ to fixed values.

Altogether, we have now a system of $m$ linear equations
in the $m$ unknowns $s=(s_1,\ldots,s_m)$, which we can write
compactly as
$A\cdot s = b$, with a square matrix $A$ whose entries come from
\thetag{\ref{eq:slope0}--\ref{eq:proportion2}}.
%are the variables we wish to solve for, and $b$ is a column $m$-vector
%whose entries also come from \eqref{slope}.  
Only four entries of
the right-hand side vector
$b$
are non-zero, due to the four boundary equations.
We will show that $A\cdot s=b$ has a unique
solution and that this solution gives a \Fary\ embedding of $G$.

\subsection{Setting Proportionality Constraints}

%Since our plan is to show how to morph the given embedding of $G$ into the desired embedding, it is important that the given embedding satisfy the appropriate system of equations.  

We now specify how the coefficients in the proportionality constraints
are chosen, so that they are satisfied by the initial embedding.  The statement of \thmref{a-graph} assumes that $G$ is a \Fary\ embedding.  In this embedding, every edge $e$ intersects $Y$
in a single point $(0,y_e')$ and has a slope $s_e'$.  For a particular
vertex $v\in Y$, incident to edges $a_1,\ldots,a_k$ and $b_1,\ldots,b_\ell$
as described above, we use the slopes in the given embedding to set the
coefficients in the proportionality constraints.  In the notation used
in \eqref{proportion}, we set
\[
\lambda_i = (s_{b_i}'-s_{b_1}')/(s_{b_\ell}'-s_{b_1}') 
\]
and in \eqref{proportion2}, we set
\[
\mu = (s_{a_1}'-s_{a_k}')/(s_{b_\ell}'-s_{b_1}') \enspace .
\]
This ensures that the slopes $s_{e_1}',\ldots,s_{e_m}'$ satisfy the
proportionality constraints.


\subsection{Ordering constraints}

We define a relation $\prec$ on the edges of $G$, where $e_1 \prec e_2$ if and only if
\begin{enumerate}
	\item $y_{e_1} < y_{e_2}$ and $e_1$ and $e_2$ have a common endpoint $v\in L$; or
	\item $y_{e_1} > y_{e_2}$ and $e_1$ and $e_2$ have a common endpoint $v\in R$.
\end{enumerate}
We say that a vector $s=(s_1,\ldots,s_m)$ \emph{satisfies the ordering
	constraints} if $s_{e_1} < s_{e_2}$ for every pair $e_1,e_2\in E(G)$
such that $e_1\prec e_2$. This definition captures the condition that vertices of $G$ in $L$ (respectively, $R$) should be embedded so that they remain in $L$ (respectively, $R$), as in the following. 

\begin{obs}\obslabel{left-right}
If a solution $s$ to $A\cdot s=b$ satisfies the ordering constraints, then every vertex that is in $L$ (in $R$) in $G$ is also in $L$ (respectively in $R$) in the embedding corresponding to $s$. 
\end{obs}

\begin{proof}	  
Consider any vertex $v$ that is in $L$ in $G$ and that is incident to (at least) two edges $e_1$ and $e_2$. Assume w.l.o.g.\ that $y_{e_1} < y_{e_2}$, and hence that $e_1 \prec e_2$. Since $s$ satisfies the ordering constraints we have $s_{e_1} < s_{e_2}$, hence the lines with slopes $s_{e_1}$ and $s_{e_2}$ through $(0,y_{e_1})$ and $(0,y_{e_2})$, respectively, meet in $L$. The argument for the vertices in $R$ is analogous. 
\end{proof}	  

Note that $\prec$ is acyclic since $G$ is an A-graph and therefore the slopes $s_{e_1}',\ldots,s_{e_m}'$ of edges
in $G$ satisfy the ordering constraints.  This would not be possible if $\prec$ contained cycles. 
%  Indeed, $i_1\prec
% \cdots \prec i_r$ implies that, for each $j\in\{3,\ldots,r\}$, $y_{i_j}\in
% (\min\{y_{i_{j-1}},y_{i_{j-2}}\}, \max\{y_{i_{j-1}},y_{i_{j-2}}\})$. Thus,
% a chain in $\prec$ corresponds to a sequence of strictly nested intervals.

\begin{lem}\lemlabel{order-gives-embedding}
	Any solution $s$ to $A\cdot s=b$ satisfying
	the ordering constraints % $\prec$
	yields a
	\Fary\ embedding of $G$.
\end{lem}

\begin{proof}
	If $G$ is a plane embedding of a 2-connected graph, then a
	straight-line embedding $G'$ of $G$ is a \Fary\ embedding provided
	that two conditions are met:
	(i) For every vertex~$v$, the clockwise order of the
	edges around $v$ in $G'$ is the same as in $G$; and
	(ii) every face of $G$ is embedded without crossings in $G'$
	(Devillers, Liotta, Preparata, and Tamassia \cite[Lemma~16]{devillers.liotta.ea:checking}).
	
	In our case, $G'$ is a straight-line embedding of $G$ given by a solution
	to $A\cdot s = b$ that satisfies the ordering constraints.  
	
	
	First we show that $G'$ satisfies condition (i). There are three cases to consider:
	\begin{enumerate}
		\item $v\not\in Y$. Since $s$ satisfies the ordering constraints, by \obsref{left-right} we have that $v\in L$ (respectively, $R$) in $G$ if and only if $v\in L$ (respectively, $R$) in $G'$. This, together with the fact that the order in which the edges incident to $v$ intersect $Y$ in $G$ and $G'$ agrees implies that the  order of the edges incident around $v$ in $G$ and $G'$ agrees.
		
		\item
		$v\in Y$, with incident edges $a_1,\ldots,a_k\in
		L\cup Y$ and $b_1,\ldots,b_\ell\in Y\cup R$ as in \figref{ab}. 
		\begin{enumerate}
			\item If $v$ is a boundary vertex, say that $v$ is at $(0,y_1)$ in $G'$, then the slopes of $a_1$, $b_1$ and one of $a_2$ or $b_2$ are fixed by the boundary equations.  The proportionality
			constraints then fix all the slopes of the edges incident to $v$ so that
			their ordering agrees with that of $G$.
			
			\item If $v$ is an interior vertex then, by \obsref{left-right}, the edges $a_1,\ldots,a_k$ are in $L\cup Y$ and the edges $b_1,\ldots,b_\ell$ are in $R\cup Y$ in $G'$. Further, as discussed above, the proportionality constraints ensure that the clockwise order of $v$'s incident edges in $G'$ either matches that of $G$, or it is completely reversed, so that the counterclockwise order of vertices around $v$ in $G'$ is $a_1,\ldots,a_k,b_\ell,\ldots,b_1$. 	Let us assume for contradiction that the latter case happens:
			\begin{equation}
			\label{eq:not-ordered}
			s_{b_1}\ge s_{b_\ell}
			\text{ and }
			s_{a_k}\ge s_{a_1}
			\end{equation}
			Let $e$ be the third edge of the triangle with edges $a_1$ and $b_1$,
			and let 
			$f$ be the third edge of the triangle with edges $a_k$ and $b_\ell$.
			Then the ordering constraints for the endpoints of $e$ imply
			\begin{math}
			s_{b_1}<s_e<s_{a_1}
			\end{math},
			and the ordering constraints for the endpoints of $f$ imply
			\begin{math}
			s_{a_k}<s_f<s_{b_\ell}
			\end{math}.
			Together with \eqref{not-ordered}, this leads to a contradiction.
		\end{enumerate}
	\end{enumerate}
	
	Next we show that $G'$ satisfies condition (ii). The graph $G$ has two
	kinds of faces:
	\begin{enumerate}
		\item Let $q = abcd$ be a quadrilateral face of $G$. By Property 3 of A-graphs we have that $q$ is non-convex in $G$; let $c$ be the reflex vertex of $q$ in $G$. Note that $a\notin Y$ and $c\notin Y$ by Properties 1 and 5 of A-graphs, respectively. Conversely, each of $b$ and $d$ might or might not be on $Y$. Then the ordering constraints and the proportionality constraints at $b$ and $d$ ensure that $q$ is non-crossing in $G'$.
		
%		If $q=abcd$ is a quadrilateral face in $G$ with no vertex on $Y$,
%		then the ordering constraints imply that $q$ is non-crossing in $G'$.
%		Otherwise, by Property~3 of A-graphs, $c$ is reflex vertex of $q$ and,
%		by Property~5 of A-graphs, $c\not\in Y$.  The vertex $a$ opposite $c$
%		is also not in $T$, so so $b\in Y$ and/or $d\in Y$.  In this case,
%		the ordering constraints and the proportionality constraints at $b$
%		and/or $d$ ensure that $q$ is non-crossing in $G'$.
		
		\item
		For a triangular face $\alpha\beta\gamma$, with $\gamma\in Y$, 
		the ordering constraints on the vertices $\alpha$ and $\beta$
		ensure that the triangle does not degenerate, and is therefore non-crossing, in $G'$.
	\end{enumerate}
	
	Therefore, by the result
	of Devillers \etal\
	cited above, $G'$ is a \Fary\ embedding. % of~$Q$.
\end{proof}

Any solution $s$ to $A\cdot s=b$ has the outer face drawn as $\Delta$, by the boundary equations, and has the intersection between $e_i$ and $Y$ at $(0,y_i)$, by the concurrency constraints. Hence, by~\lemref{order-gives-embedding} ensuring the existence of a solution $s$ to $A\cdot s=b$ satisfying the ordering constraints is enough to prove~\thmref{a-graph}.

\subsection{Strong Ordering Constraints}
\label{strong}

For some $\epsilon \ge 0$, we say that $s=(s_1,\ldots,s_m)$ satisfies
the \emph{$\epsilon$-strong ordering constraints} if, for each
$i,j\in\{1,\ldots,m\}$ such that $e_i\prec e_j$, the inequality
$s_j-s_i > \epsilon$ holds.
% A solution $s$ that satisfies
Clearly, any $s$ satisfying the $\epsilon$-strong ordering constraints also satisfies the ordering constraints. The converse holds when equations~\thetag{\ref{eq:slope0}}--\thetag{\ref{eq:proportion2}} are satisfied, for a suitably small $\epsilon$, as in the following.

\begin{lem}\lemlabel{weak-to-strong}
	Any solution $s$ to $A\cdot s=b$ that satisfies
	the ordering constraints
	also satisfies 
	the $\epsilon$-strong ordering constraints
	for all $\epsilon<\min\{|y_i-y_j| : 1\le i< j\le m\}$.
\end{lem}

\begin{proof}
	By \lemref{order-gives-embedding} every vertex is contained in the interior or on the boundary of $\Delta\subset[-1,1]^2$.
	Hence, every $x$-coordinate is
	in the interval $[-1,1]$.
	A vertex incident to $e_i$ and $e_j$ has $x$-coordinate
	$(y_j-y_i)/(s_i-s_j)$.
	From $|(y_j-y_i)/(s_i-s_j)|\le 1$ we derive
	$|s_i-s_j|\ge|y_j-y_i| > \epsilon$.
\end{proof}

\subsection{Uniqueness of Solutions Satisfying Ordering Constraints}

The utility of the $\epsilon$-strong ordering constraints is that they
allow us to appeal to continuity. 
%It is impossible 
%to violate
%the ordering constraints without first violating the
%$\epsilon$-strong ordering constraints.
%But since the ordering constraints imply the
%$\epsilon$-strong ordering constraints,
%it is not possible
%to violate
%the ordering constraints at all.
% by showing that, if $A\cdot s=b$
%were to have some undesireable property, then some function which we
%know to be continuous would have a discontinuity.
An example will be seen in the following proof.

\begin{lem}\lemlabel{unique}
	If $s$ is a solution to $A\cdot s=b$ that satisfies the ordering
	constraints, % $\prec$, 
	then $s$ is 
	the unique solution to $A\cdot s=b$.
\end{lem}

\begin{proof}
	Assume that $\epsilon$ is fixed so that $0<\epsilon<\min\{|y_i-y_j| : 1\le i< j\le m\}$.
	
	Suppose, for a contradiction, that there is a solution $s$ to $A\cdot s=b$ that satisfies the ordering
	constraints, % $\prec$,
	but is not unique.  Since $A\cdot s=b$ is a linear system, there is an entire (at least) 1-parameter family of solutions,
	i.e., there is a non-zero $m$-vector $r$ such that, for every
	$\lambda\in\mathbb R$, $A(s+\lambda r)=b$.
	
	
	Define the continuous (in fact, piecewise linear) function
	\begin{equation*}
	f(\lambda) := \min \{\, (s_j+\lambda r_j)-(s_i+\lambda r_i) : e_i \prec
	e_j\,\}
	,
	\end{equation*}
	and let $\lambda^*$ be the value with the smallest absolute value
	$|\lambda^*|$ such that
	$f(\lambda^*)\le\epsilon/2$. In order to prove that $\lambda^*$ exists, it suffices to prove that a value $\lambda$ exists such that $f(\lambda)\le 0$. Note that the vector $r=(r_1,\ldots,r_m)$ has at least four zero entries
	$r_1=r_a=r_b=r_m=0$ since the slopes $s_1$, $s_a$, $s_b$, and $s_m$
	are fixed.
	Since $G$ is connected and $m\geq 5$, there is at least one vertex $v$ with two incident edges $e_k$
	and $e_\ell$ such that $r_k=0$ and $r_\ell\neq 0$. 
	We can thus pick $\lambda$ so that $(s_\ell+\lambda r_\ell)-(s_k+\lambda r_k)=s_\ell-s_k+\lambda r_\ell=0$,
	and then $f(\lambda)\le 0$. It follows that $\lambda^*$ exists.
	
	Now we know that, for any $\lambda$ between $0$ and $\lambda^*$ and for any $i$ and $j$ such that $e_i\prec e_j$, the difference $(s_j+\lambda r_j)-(s_i+\lambda r_i)$ has the same sign as $s_j-s_i$. It follows that the slopes satisfy the ordering constraints throughout
	this interval, and
	\lemref{weak-to-strong} implies that $f(\lambda^*)\ge\epsilon$, a contradiction.
\end{proof}

%The proof of \lemref{unique} was quite explicit (perhaps overly so)
%in showing the discontinuity caused by the $\epsilon$-strong ordering
%constraints.  In subsequent arguments we will not be quite so explicit.

\subsection{A Parametric Family of Linear Systems}

Note that $A$ and $b$ are functions of $y=(y_1,\ldots,y_m)$ and of the
four slopes $h=(s_1,s_a,s_b,s_m)$. We make this explicit, by writing
$A_1=A(y,h)$ and $b_1=b(y,h)$.

Let $y'=(y_1',\ldots,y_m')$ and $s'=(s_1',\ldots,s_m')$ denote the
$y$-intercepts and the slopes of the edges in the initial embedding of $G$
and let $h'=(s_1',s_a',s_b's_m')$. 

Consider the system $A(y',h')\cdot s = b(y',h')$.  This system has
at least one solution $s=s'$.  We now define
a continuous family of linear systems that interpolates between $A(y',h')\cdot s=b(y',h')$ and $A(y,h)\cdot s=b(y,h)$. Suppose first that the outer face of $G$ is delimited by a quadrilateral.

For $0\le t\le 1$ and $i\in\{1,a,b,m\}$,  define $s_i(t)=(1-t)s_i' + ts_i$ and $h(t)=(s_1(t),s_a(t),s_b(t),s_m(t))$.
Note that
\[  
s_1(t)-s_a(t) = (1-t)(s_1'-s_a') + t(s_1-s_a) > 0. 
\]
Inequalities $s_1'-s_a'>0$ and $s_1-s_a>0$ come from the assumption that the vertex incident to  $e_1$ and $e_m$ is in $L$ both in $G$ and in $\Delta$. Similarly, we have $s_m(t)-s_1(t)>0$ and $s_b(t)-s_m(t)>0$. Then we can define \[
\epsilon_1 = \min_{0\le t\le 1}\min\{s_1(t)-s_a(t), s_m(t)-s_1(t), s_b(t)-s_m(t)\}
\]
and observe that $\epsilon_1>0$.

Analogously, for every $0\le t\le 1$ and for each $i\in\{1,\ldots,m\}$, define $y_i(t) = (1-t)y_i' + ty_i$ and define $y(t)=(y_1(t),\ldots,y_m(t))$.
Observe that, for any
$1\le i< j\le m$ and any $0\le t\le 1$,
\[
y_j(t) - y_i(t) = (1-t)(y'_j-y'_i) + t(y_j-y_i) > 0.
\]
Let 
\[    \epsilon_2=\min_{0\le t\le 1}\min\{y_j(t)-y_i(t): 1\le i< j\le m\}
\]
and observe that $\epsilon_2 >0$.  

The entries in $A_t$ and $b_t$ are obtained in the same way as the entries of $A$ and $b$ were derived earlier, however each entry is now a linear function of~$t$, as $y$ and $h$ are replaced by $y(t)$ and $h(t)$, respectively, in the determination of the equations represented by $A_t$ and $b_t$.
Consider the unique quadrilateral $\Delta(t)$ whose edges cross $Y$ at
$y_1(t)$, $y_a(t)$, $y_b(t)$, $y_m(t)$ and have slopes $s_1(t)$,
$s_a(t)$, $s_b(t)$, and $s_m(t)$, respectively. Note that $\Delta(0)$ is the quadrilateral delimiting the outer face of $G$, while $\Delta(1)=\Delta$. Since $y_1(t),y_a(t),y_b(t),y_m(t)\in [-1,1]$, and since each of $s_1(t)$,
$s_a(t)$, $s_b(t)$, and $s_m(t)$ is at least $\epsilon_1$, we have that $\Delta(t)\subset[-1/\epsilon_1,1/\epsilon_1]\times[-\infty,\infty]$.
Hence, after scaling the $x$-coordinates by $1/\epsilon_1$,
\lemref{weak-to-strong} applies to $A_t\cdot s =b_t$, so any solution $s$
that satisfies $\prec$ also satisfies the $\epsilon^*$-strong ordering
constraints, for $\epsilon^*=\epsilon_1\cdot\epsilon_2$.

If the outer face of $G$ is delimited by a triangle, the arguments are analogous, however the inequalities $s_1(t)-s_a(t)>0$ and $s_m(t)-s_1(t)>0$ become either $s_m(t)-s_1(t)>0$ and $s_a(t)-s_m(t)>0$ or $s_1(t)-s_m(t)>0$ and $s_a(t)-s_1(t)>0$, depending on whether $(0,y_1)$ or $(0,y_m)$ is a vertex of $G$, respectively. Further, the inequality involving $s_b(t)$ now states  either $s_b(t)-s_a(t)>0$ or that $s_b(t)$ is smaller than the smallest between $s_1(t)$ and $s_m(t)$, depending on which of the two holds in $G$ (as when determining the boundary equations).

\subsection{Existence (and uniqueness) of solutions to $A_t\cdot s=b_t$}

We now prove the following lemma which, together with \lemref{order-gives-embedding}, completes the proof of \thmref{a-graph}.

\begin{lem}\lemlabel{uniqueness}
	For every $0\le t\le 1$, the system $A_t\cdot s=b_t$ has a unique solution,
	and this solution satisfies the ordering constraints. % $\prec$.
\end{lem}

\begin{proof}
	Since $A_t$ is an $m\times m$ matrix, the system $A_t\cdot
	s=b_t$ has a unique solution~$s$ if and only if $\det A_t \neq 0$.
	When $\det A_t =0$, the system may have no solutions or
	multiple solutions.  
	When $\det A_t\neq 0$, 
	Cramer's Rule states that
	the solution
	is $s(t)=(s_1(t),\ldots,s_m(t))$ where, for each
	$i\in\{1,\ldots,m\}$,
	\[ 
	s_i(t) = \frac{\det A_t^i}{\det A_t }
	\]
	and $A_t^i$ denotes the matrix $A_t$ with its $i$-th column replaced
	by $b_t$. 
	The numerators $\det A_t^i$ and the common
	denominator $\det A_t $ are polynomials in $t$, and therefore
	continuous
	functions of $t$.
	The solution $s(t)=(s_1(t),\ldots,s_m(t))$ depends continuously on $t$
	as long as  $\det A_t\ne0 $.
	
	We have already established that $A_0\cdot s=b_0$ has a solution $s(0)=s'$
	that satisfies the ordering constraints. By \lemref{unique}, this solution
	is unique, so $\det A_0\neq 0$.
	
	Let $t^*$ be the smallest $t>0$
	%, if it exists, 
	for which 
	$\det A_{t}= 0$. If such a value does not exist we set $t^*=\infty$.
	% or $t>1$, we are
	% done.
	
	First we argue that, for all $0\le t <\min \{1,t^*\}$, the unique solution $s(t)$ to $A_t\cdot s=b_t$ satisfies the ordering constraints. This argument is similar to the proof of \lemref{unique}. Suppose, for a contradiction, that there is a value $0<t<\min\{1,t^*\}$ for which $s(t)$ does not satisfy the ordering constraints. As $t$ increases its value from $0$ to $\min\{1,t^*\}$, since $s(t)$ depends continuously on $t$, a value is reached in which $s(t)$ violates the $\epsilon^*$-strong ordering constraints, while it does not violate the ordering constraints. However, this contradicts	\lemref{weak-to-strong}.
	
	If $t^*>1$ the same argument also extends to $t=1$ and we are done.
	Let us therefore assume that $0<t^*\le 1$ and derive a contradiction.
	We look at whether the limit $s^*=\lim_{t\uparrow t^*}
	s(t)$ exists.
	Each function $s_i(t)$ is a quotient of two polynomials.
	Thus, for $t\to t^*$ it can either converge to $s_i(t^*)$ continuously, or diverge to $+\infty$ or $-\infty$.
	%
	For $t<t^*$ all solutions $s(t)$ to the systems $A_t\cdot s=b_t$ satisfy the $\epsilon^*$-strong ordering constraints.
	Hence, if the limit exists, by continuity, it also satisfies $A_{t^*}\cdot s^*=b_{t^*}$
	and the $\epsilon^*$-strong ordering constraints.
	By \lemref{unique}, the solution $s^*$ is
	the unique solution
	of $A_{t^*}\cdot s=b_{t^*}$, but this contradicts the assumption
	$\det A_{t^*}= 0$.
	
	It remains to rule out the possibility that
	$A_{t^*}\cdot s=b_{t^*}$ has no solution because
	$\lim_{t\uparrow t^*} s(t)$ does not exist.  Define the set $H=\{e_i\in
	\{e_1,\ldots,e_m\}:\text{$\lim_{t\uparrow t^*} s_i(t)$ exists}\}$.
	The set $H$ corresponds to the edges of $G$
	with bounded slope; the remaining edges become vertical as $t\to t^*$.
	%
	\lemref{partition-extended} below shows that $H$ contains all the edges of $G$. Hence $\lim_{t\uparrow t^*} s(t)$ exists. This completes the proof of the lemma.
\end{proof}

%Condition 1 in the following lemma is more general that what we need,
%because it allows us to proceed by induction.

It remains to prove that the set $H$ defined in the proof of \lemref{uniqueness} contains all the edges in $E(G)$. We start by stating some properties of $H$.

	\begin{prop}\proplabel{set-H}
		The set   $H$ has the following properties: 
		\begin{compactenum}[(PR1)]
			\item $H$ contains every edge incident to a vertex on the outer face of $G$.
			\item \label{off-C}
			If a vertex $v\not\in Y$ has two incident edges in
			$H$,
			then all $v$'s incident edges belong to $H$.
			\item \label{on-C}
			If a vertex $v\in Y$ has two incident edges $vx,vy$ with $x,y\in L$ or $x,y\in R$, then all $v$'s incident edges belong to $H$.
			\item If $e_i \prec e_j \prec e_k$ and $e_i,e_k\in H$, 
			then $e_j\in H$.
		\end{compactenum}
	\end{prop}
		
	\begin{proof}
		Let us first prove Property~(PR2).	If $v$ does not lie on $Y$ and two incident edges have bounded slope, then the location of $v$ is fixed in the limit.
		By the concurrency constraints, the slopes of the remaining incident
		edges are also bounded.
		
		We now prove Property~(PR3). If $v$ lies on $Y$ and on the outer face of $G$, then all its incident edges have fixed slopes and are therefore in $H$. Assume next that $v$ lies on $Y$ and is an interior vertex of $G$.  Define the edges $a_1,\ldots,a_k$
		and $b_1,\ldots,b_\ell$ incident to $v$ as in \figref{ab}.  Let $e$ be the third edge of the triangle with edges $a_1$ and $b_1$, and let
		$f$ be the third edge of the triangle with edges $a_k$ and $b_\ell$.
		Assume without loss of generality that two of the edges $a_i$ belong to $H$. Then, by the proportionality constraints, all the edges $a_i$ belong to $H$, and moreover the range $s_{b_\ell}(t)-s_{b_1}(t)$ converges to a bounded limit as $t\to t^*$.  It follows that either all the slopes of the edges $b_j$ are bounded, or they all diverge to $+\infty$,
		or they all diverge to $-\infty$. The ordering constraints for the
		endpoints of $e$ imply \begin{math}
		s_{b_1}<s_e<s_{a_1}
		\end{math}.
		This is inconsistent with $\lim_{t\uparrow t^*} s_{b_1}(t)=+\infty$.
		The ordering constraints for the endpoints of $f$ imply
		\begin{math}
		s_{a_k}<s_f<s_{b_\ell}
		\end{math}.
		This is inconsistent with $\lim_{t\uparrow t^*} s_{b_\ell}(t)=-\infty$. Thus, the only
		possibility is that all the slopes of the edges incident to $v$ are bounded.
		
		We now show that $H$ satisfies Property~(PR1).  If $v$ is a boundary
		vertex with $v\in Y$ then, as discussed above, all of $v$'s incident edges
		have their slopes fixed by the boundary equations and proportionality
		constraints.  If $v\not\in Y$, then the location of $v$ is fixed by the
		boundary equations and therefore the slopes of $v$'s incident edges
		are fixed by the requirement that each edge $e_i$ incident to $v$ also
		contains $(0,y_i)$.
		
		Finally, Property~(PR4) follows easily by the ordering constraints.
	\end{proof}
	
We now present \lemref{partition-extended}, which completes the proof of \lemref{uniqueness} and \thmref{a-graph}. The lemma is proved by induction on something
that starts as an $A$-graph but is then dismantled into something
more general.  A \emph{near-A-graph} is a graph that satisfies all the
conditions of an A-graph except that its outer face can be arbitrarily
complex.  More specifically, each edge of a near-A-graph intersects $Y$
in exactly one point; each inner face is a triangle or a quadrilateral;
each triangular face contains one vertex in each of $Y$, $L$, and $R$;
and for every vertex $v$ on $Y$ each of the faces directly above and
below $v$ is either a triangular face or the outer face.

\begin{lem}\lemlabel{partition-extended}
Let $G$ be a near-A-graph and let $H \subseteq E(G)$ be a set of edges satisfying Properties (PR1)--(PR4) of \propref{set-H}. 
%	\begin{compactenum}
%		\item if $v$ is a vertex on the outer face 
%		of $G$, then edges incident to $v$ belong to $H$;
%		\item
%		if a vertex $v$ does not lie on $Y$ and has two incident edges in
%		$H$,
%		then all its incident edges belong to $H$;
%		\item
%		if a vertex $v$ lies on $Y$ and has two incident edges in
%		$Y\cup L$ or two incident edges in $Y\cup R$
%		then all $v$'s incident edges belong to $H$.
%		\item if $i \prec j \prec k$ and $i,k\in H$, 
%		then $j\in H$.
%	\end{compactenum}
	Then $H=E(G)$.
\end{lem}

\begin{proof}
	The proof is by induction primarily on the number of inner faces of $G$ and secondarily on the number of vertices of $G$. We dismantle $G$
	from outside while maintaining
	Properties~(PR1)--(PR4). In particular:
	\begin{itemize}
		\item If $G$ is not 2-connected but has more than one edge, we cut it
		into pieces with fewer edges.
		\item If $G$ is 2-connected, we modify it and reduce it to a
		graph
		with fewer interior faces,
		keeping the number of edges fixed.
	\end{itemize}
	Eventually, we reduce to a graph with a single edge, and here the
	claim is trivial because the edge belongs to the boundary.
	
	We refer to the edges of $H$ simply as \emph{$H$-edges}.
	% (horizontal edge) if it is
	%   in $B$ and a \emph{v-edge} (vertical edge) otherwise.  Thus, we wish
	%   to show that all edges of $G$ are h-edges. 
%	The edges incident to the outer
%	face of $G$ are called {\em boundary edges}.
	
	If $G$ is not connected then we can independently apply induction on each component
	of $G$. 
	
	If $G$ has a cut vertex $v$ whose removal
	splits $G$ into components $A_1,\ldots,A_r$ then, for each
	$i\in\{1,\ldots,r\}$, we can independently apply induction on the subgraph $G_i$ of $G$
	induced by $V(A_i)\cup\{v\}$. Indeed, every edge of $G_i$ inherits its classification as an $H$-edge from its corresponding edge in $G$. Then it is easy to see that Properties~(PR1)--(PR4) are satisfied by $G_i$. In particular, Property~(PR1) follows from the fact that every boundary vertex of $G_i$ is also a boundary vertex of $G$; this is because each inner face of $G$ is a quadrangle or a triangle, hence $G_i$ cannot be nested inside a different subgraph $G_j$ of $G$.
		
	We are left with the case in which $G$ is a 2-connected near-A-graph whose outer face 
	is delimited by a simple cycle $F$. We distinguish two cases.
	
	{\em Case 1}. The cycle $F$ contains a vertex $v$ on $Y$. Then we identify a triangle $vab$ incident to $v$ and delimiting an inner face of $G$; we open $vab$ up, merging it into the outer face. \figref{lemma-y-3} illustrates the procedure for the case that $ab$ lies below $v$, with $a\in L$ and $b\in R$. Let $u$ and $w$ be the predecessor and successor of $v$ on the
	counterclockwise cycle $F$, and assume w.l.o.g.\ that $u\in R$.
	We construct a new graph $G'$ by splitting $v$ into two vertices $x$
	and $y$ that both lie on $Y$, with $y$ above $x$. We make $x$ adjacent to $u$ and to every neighbor of $v$ between $b$ and $u$.
	We make $y$ adjacent to all the remaining neighbors of $v$.
	\figref{lemma-y-3} shows that this procedure works both for $w\in R$
	and for $w\in L$. Note that $G'$ has one inner face less than $G$, hence induction applies.
		
	\begin{figure}[htb]
		\centering{\includegraphics{figs/open-a-triangle}}
		\caption{Proof of \lemref{partition-extended} with a vertex
			$v$ on $Y$. Integrating a triangle in the outer face.}
		\figlabel{lemma-y-3}
	\end{figure}
	
	
	
	
	{\em Case 2}. The cycle $F$ contains no vertex on $Y$.
	Then $F$ contains at least four vertices. Since $Y$ intersects
	every edge of $F$, we have that $F$
	contains three consecutive vertices
	$u,v,w$ such that $Y$ exits an inner face through $uv$ and enters
	an inner face through $vw$, see \figref{lemma-y-4}.
	This implies that $v$ is a reflex vertex
	of some inner face $q=vabc$ of $G$.  Indeed, $vc$ is the first edge
	incident to $v$ crossed by $Y$ and $va$ is the last such edge.
	We construct a new graph $G'$ by splitting $v$ into two vertices $x$
	and $y$. We make the vertex $x$ adjacent to $u$ and every neighbor
	$z$ of $v$ such $Y$ intersects $vz$ before $vu$.  We make
	$y$ adjacent to all of $v$'s neighbors that are not adjacent to $x$.
	In $G'$, $q$ is part of the outer face, so $G'$ has one less inner
	face than $G$, hence induction applies.
	
	\begin{figure}
		\centering{\includegraphics{figs/lemma-y}}
		\caption{Proof of \lemref{partition-extended} for a reflex
			vertex $v$. Integrating a quadrilateral in the outer face.}
		\figlabel{lemma-y-4}
	\end{figure}
	
	
	
	This finishes the description of how we modify $G$ into $G'$. Every edge of $G'$ inherits its classification as an $H$-edge from its corresponding edge in $G$. We have to show that $G'$ satisfies Properties~(PR1)--(PR4). Actually Property~(PR1) is the only property that needs to be discussed, as the other properties follow trivially from the fact that $G$ satisfies them. 
%	Indeed,  $G'$ some of the $\prec$-relations involving edges incident to $v$ are missing, but no new ones are introduced, so $G'$ still
%	satisfies Condition~4.
%	The same argument applies to
%	Conditions~2 and~3. Some adjacent edges in $G$ might no longer be adjacent
%	in $G'$, but this makes Condition 2 and 3 only weaker.
%	
%	
	
	First, note that all the edges incident to the new vertices $x$ or $y$ were incident to $v$
	before, and thus they are $H$-edges. Second, all the edges incident to any boundary vertex of $G'$ that is also a boundary vertex of $G$ are $H$-edges, since the edges of $G'$ inherit their classification as $H$-edges from their corresponding edges in $G$. It remains to deal with the boundary vertices of $G'$ that are inner vertices of $G$.  
	
	In Case 1 we have two boundary vertices of $G'$, namely $a$ and $b$, which are inner vertices of $G$. Note that $a$ and $b$ do not lie on~$Y$. By Property~(PR1) for $G$, we have that $va$ and $vb$ are $H$-edges, since they are incident to the boundary vertex $v$. From the ordering constraints around $a$ and $b$ we get $va\prec ab\prec vb$
	or
	$vb\prec ab\prec va$, and thus, by Property~(PR4) for $G$, we have $ab\in H$.
	Now we have two $H$-edges $va$ and $ab$ incident to $a$,
	and by Property~(PR2) for $G$ all the edges incident
	to $a$ belong to $H$. It follows that all the edges incident to $a$ in $G'$ are $H$-edges, and similarly for~$b$.
	
	In Case~2 we have three boundary vertices of $G'$, namely $a$, $b$, and $c$, which are inner vertices of $G$. By Property~(PR1) for $G$, we have that $va$ and $vc$ are $H$-edges, since they are incident to the boundary vertex $v$. Consider the quadrilateral $q=vabc$ of $G$. By the ordering constraints, we get
	$vc \prec bc\prec ba\prec va$ or $va \prec ba\prec bc\prec vc$,
	depending on whether $v\in L$ or $v\in R$, respectively. Thus, by Property~(PR4) for $G$, the edges $bc$ and $ba$ are also $H$-edges. The vertex $b$ does not lie on $Y$. The vertex $a$ might lie on $Y$ or not, but if it does, then the two incident edges $va$ and $ab$ lie in the same half-plane.
	The same holds for $c$. Thus by Properties~(PR2) or~(PR3) for $G$ all the edges incident
	to $a$, $b$ and $c$ in $G$ belong to $H$. It follows that all the edges incident to $a$, $b$, and $c$ in $G'$ are $H$-edges.
	
	
	
	%   By Conditions~1 and 2, all edges incident to $v$ are $H$-edges and $v$
	%   is a reflex vertex of $q$. Therefore all edges of $q$ are $H$-edges.
	Since $G'$ satisfies Properties~(PR1)--(PR4) induction applies and all the edges of $G'$ (and thus all the edges of $G$) are $H$-edges. This completes the proof.
\end{proof}

%!TEX root = main.tex
\section{Triangulations}
\seclabel{triangulations}

%We will sometimes make use of this simple fact:
%\begin{obs}\obslabel{quad}
%	If $q=abcd$ is a simple quadrilateral, then neither of the segments $ac$
%	or $bd$ cross any of the edges of $q$.
%\end{obs}

%As is the case with \thmref{a-graph} there is an annoying case distinction
%that occurs when $Y$ contains vertices on the outer face.  

In this section we prove that every collinear set is free. Recall that a {\em triangulation} is a (not necessarily straight-line) plane embedding of an edge-maximal planar graph. Let $T$ be a triangulation, let $r_1,\ldots,r_k$ be a sequence of vertices and edges in $T$, and let $y_1<\cdots<y_k$ be a sequence of numbers. A triangle $\Delta=\alpha\beta\gamma$ is \emph{compatible} with $r_1,\ldots,r_k$ and $y_1,\ldots,y_k$ if the following conditions hold:
\begin{compactenum}
	\item if $r_1$ is a vertex, then $\beta=(0,y_1)$, otherwise $(0, y_1)$ is in the interior of the edge $r_1=\beta\gamma$; and
	\item if $r_m$ is a vertex, then $\alpha=(0,y_m)$, otherwise $(0,y_m)$ is in the interior
	of the edge $r_m=\alpha\gamma$.
\end{compactenum}

%A triangulation is \emph{admissible} if the intersection of $Y$ with each edge of the triangulation is either empty, a single point,
%or the entire edge.  

We have the following.

\begin{thm}\thmlabel{main}
	Let
	\begin{compactenum}
		\item $T$ be a triangulation;
		\item $C$ be an admissible curve for $T$;
		\item $r_1,\ldots,r_k$ be the sequence of vertices and open edges
		of $T$ that are intersected by $C$, in the order in which they are intersected by $C$ (each edge of $T$ that is entirely on $C$ has its end-vertices represented by two consecutive elements $r_i$ and $r_{i+1}$ of $r_1,\ldots,r_k$, while the open edge is not in $r_1,\ldots,r_k$);
		\item $y_1<\cdots<y_k$ be a sequence of numbers; and
		\item $\Delta$ be a triangle that is compatible with 
		$r_1,\ldots,r_k$ and $y_1,\ldots,y_k$.
	\end{compactenum}
	Then, for any $0<\epsilon<\min\{(y_{i+1}-y_i)/3:i\in\{1,\ldots,k-1\}\}$, $T$ has a \Fary\ embedding such that the outer face is delimited by $\Delta$ and such that the following hold for each $i=1,\ldots,k$: 
	\begin{compactenum}
		\item if $r_i$ is a vertex, then it is drawn at $(0,y_i)$; and
		\item if $r_i$ is an edge, then the intersection of $r_i$ with $Y$ has $y$-coordinate in the interval $[y_i-\epsilon,y_i+\epsilon]$.
	\end{compactenum}
\end{thm}

\begin{proof}
	%   We call $y_i$ the (desired) \emph{crossing coordinate} for $r_i$. If
	%   a \Fary\ embedding contains an edge whose intersection with the
	%   $y$-axis is $\{(0,y)\}$ or a vertex at $(0,y)$, we say that the edge
	%   or vertex \emph{crosses the $y$-axis at $y$}.
	%
	%   Let $L=C^-$, $R=C^+$.  
We start by classifying the edges of $T$. An edge of $T$ is \emph{marked} if its intersection with $C$ is non-empty, otherwise it is \emph{unmarked}. A marked edge might intersect $C$ in a single point (then it is a \emph{crossing edge}), or might lie on $C$. If an edge lies on $C$ or if it is unmarked, then it is a \emph{non-crossing edge}. 
	
	%   We prove an extension of the theorem to the case where $T$ is an
	%   non-crossing embedded graph whose faces consist of triangles (3-cycles)
	%   and quadrilateral (4-cycles) with the resriction that, for every
	%   quadrilateral face $q$, all four edges of $q$ are crossing edges.
	%   The proof is by induction on the number of non-crossing edges plus the
	%   number of vertices of $T$.
	
	%   \paragraph{Base Cases:}
	%   There are three base cases tht we handle explicitly.  If $T$ contains
	%   2 or fewer crossing edges, If $T$ is the complete graph, $K_4$ on 4
	%   vertices, but only has only three crossing edges, then the theorem is
	%   also easy to prove directly.  The last base case occurs when all edges
	%   of $T$ are crossing edges.  In this case $T$ is bipartite and therefore
	%   all its faces are quadrilaterals, so $T$ is a quadrilateralization.
	%   This case is handled directly by \lemref{quad2}.
	%
	%   Thus we may assume that $T$ has at least one non-crossing edge and
	%   at least 2 crossing edges.  
	
	The proof is by induction on the number of vertices of $T$, primarily, and on the number of non-crossing edges of $T$, secondarily.
	We begin by describing reductions that allow us to apply the
	inductive hypothesis. When none of these reductions applies,
	we arrive at our base case. To handle the base case 
	we remove every unmarked edge of $T$ to obtain an A-graph, on which we
	apply \thmref{a-graph}.

% we argue that
%	$T$ has a sufficiently simple structure that it can be handled by
%	\thmref{a-graph}.  In particular, when no reduction applies, we can
	
%	Before continuing, we dispense with one easy special case.  If $Y$
%	contains an edge $e$ of the outer face, then every vertex
%	of $G$ is contained in $L\cup Y$ or every vertex of $G$ is contained
%	in $Y\cup R$.  In this case, the definition of compatible triangle
%	implies that the edge $\alpha\beta$ of $\Delta$ is contained in the
%	$y$-axis.  In this case, we can simply apply Tutte's Convex Embedding
%	Theorem to obtain a \Fary\ embedding of $G$ in which the outer face is embedded
%	on $\Delta$ with $e$ embedded on $\alpha\beta$, .  This embedding
%	satisifies all the conditions of the theorem.  Therefore, for the
%	remainder of this proof, we assume that $C$ intersects the interior
%	of at least one inner face of $T$.
	
	\paragraph{Separating Triangles.}
	(See \figref{separating}.)
	If $T$ contains a separating triangle $xyz$, then denote by $T^+$ (respectively, $T^-$) the triangulation obtained from $T$ by removing all the vertices in the interior (respectively, exterior) of $xyz$. Note that $xyz$ delimits an inner face of $T^+$ and the outer face of $T^-$. 
	
	If the interior of $xyz$ does not intersect $C$, we apply induction on $T^+$ (note that $|V(T^+)|<|V(T)|$) and then use Tutte's Convex Embedding Theorem  \cite{tutte:how} to draw $T^-$ so that its outer face is delimited by the triangle representing the cycle $xyz$ in the constructed \Fary\ embedding of $T^+$.
	
	Therefore, assume that the interior of $xyz$ intersects $C$. Then either $C$ passes through a vertex of $xyz$ and intersects the opposite open edge in a single point, or it intersects two open edges of $xyz$, each in a single point, and does not intersect the third edge of $xyz$. 
	% with each of $xy$, $yz$ and $zx$ is either empty or a single point. 
	In both cases, the vertices and edges of $T$ intersected by $C$ that are not in $T^+$ form a contiguous
	subsequence $r_i,\ldots,r_j$ of $r_1,\ldots,r_k$. Each of $r_{i-1}$ and $r_{j+1}$ is either an edge or a vertex of $xyz$.  
	
	\begin{figure}
		\centering{\includegraphics{figs/separating}}
		\caption{Recursing on separating triangles in the proof of
			\thmref{main}}
		\figlabel{separating}
	\end{figure}
	
	Set $\epsilon'$ to be any
	value less than $\min\{\epsilon,y_{i}-y_{i-1}, y_{j+1}-y_j\}$. Apply induction on $T^+$ with the value $\epsilon'$ and the sequences $r_1,\ldots,r_{i-1},r_{j+1},\ldots,r_k$ and
	$y_1,\ldots,y_{i-1},y_{j+1},\ldots,y_k$. In the obtained \Fary\ embedding of $T^+$ let $\Delta'$ be the triangle representing  $xyz$ and let $y_{i-1}'$ and $y_{j+1}'$
	be the respective $y$-coordinates of the intersections of
	$r_{i-1}$ and $r_{j+1}$ with $Y$.  By the choice of
	$\epsilon'$ we have $y_{i-1}'<y_i<\cdots<y_j<y_{j+1}'$.  Observe that
	$\Delta'$ is compatible with $r_{i-1},\ldots,r_{j+1}$ and
	$y_{i-1}',y_i,\ldots,y_j,y_{j+1}'$.
	%
	We apply induction on $T^-$ with value $\epsilon$ using the triangle $\Delta'$ and the sequences $r_{i-1},\ldots,r_{j+1}$ and
	$y_{i-1}',y_i,\ldots,y_{j},y_{j+1}'$.  Combining the \Fary\ embeddings of $T^+$
	and $T^-$ yields the desired \Fary\ embedding of $T$.  Thus, in the following we assume that $T$ has no separating triangles.
	
	\paragraph{Contractible Edges.}
	(See \figref{contractible}.)
	A face of $T$ is a \emph{crossing
		face} if it is incident to two crossing edges.  An
	unmarked edge of $T$ is \emph{contractible} if it is not contained
	in the boundary of any crossing face.  
	\begin{figure}
		\centering{\includegraphics{figs/contractible}}
		\caption{Contracting and uncontracting edges in the proof of
			\thmref{main}}
		\figlabel{contractible}
	\end{figure}
	
	If $T$ contains a contractible edge $xy$ then we contract $xy$ to
	obtain a new vertex $v$ in a smaller triangulation $T'$.   We then apply
	induction on $T'$ with the value $\epsilon'=\epsilon/2$ to obtain a \Fary\
	embedding of $T'$ such that each crossing edge $e_i$ crosses
	$Y$ in the interval $[y_i-\epsilon/2,y_i+\epsilon/2]$.
	
	To obtain a \Fary\ embedding of $T$ we uncontract $v$ by placing $x$ and $y$
	within a ball of radius $\epsilon/2$ centered at $v$. (That such
	a placement is always possible is a standard argument, see, e.g.,~\cite{fary,w-sp-05}.)  Since the
	distance between $y$ and $v$ and the distance between $x$ and $v$ are each at most $\epsilon/2$,
	each crossing edge $r_i$ incident to $x$ or $y$ crosses $Y$ in the interval $[y_i-\epsilon,y_i+\epsilon]$, as required.
	Thus, in the following we assume that $T$ has no separating triangles or contractible
	edges.
	
	%   \paragraph{Eraseable edges}
	%   We say that a non-crossing edge of $xy$ of $T$ is \emph{eraseable}
	%   if neither of its endpoints is on $C$ and both its incident faces
	%   intersect $C$.  If $T$ contains an eraseable edge $xy$, then we remove
	%   the edge $xy$ from $T$ to obtain smaller graph $T'$ on which we can
	%   apply induction. In the resulting drawing of $T'$, $x$ and $y$ lie on
	%   a common face (which may be the outer face of $T'$) and are visible.
	%   We can therefore add the edge $xy$ to obtain the desired drawing
	%   of $T$.
	
	\paragraph{Flippable edges.}
	(See \figref{flippable}.)
	We say that an unmarked edge $xy$ of $T$ is \emph{flippable} if there
	exist distinct vertices $z$, $a$, $b$, and $c$ such that: (1) $xyb$, $zyc$, $xza$ are crossing faces of $T$; (2) $xyz$ is a non-crossing face of $T$; and either (3a) $C$ intersects $za$, $xa$, $xb$, $yb$, $yc$, and $zc$ in this order, or (3b) $C$ intersects $xa$, $xb$, $yb$, $yc$, $zc$, and $za$ in this order. (Note that Case 3b can only occur when $xza$ is the outer face, otherwise $xza$ would be a separating triangle.)  
	\begin{figure}
		\centering{\includegraphics{figs/flippable}}
		\caption{Flipping edges in the proof of
			\thmref{main}}
		\figlabel{flippable}
	\end{figure}
	
	If $T$ contains the flippable edge $xy$ then we remove $xy$ and replace it with $zb$ to obtain a new triangulation $T'$. Note that, since $T$ has no separating triangles, the edge $zb$ is not already present in $T$. Further, $T'$ has the same number of vertices of $T$ and one less non-crossing edge. After choosing a crossing coordinate $y_{zb}$ for $zb$ between those $y_{xb}$ and $y_{yb}$ of $xb$ and $yb$, respectively, we can inductively embed $T'$ with value $\epsilon$ and sequences $r_1,\dots,xb,zb,yb,\dots,r_k$ and $y_1,\dots,y_{xb},y_{zb},y_{yb},\dots,y_k$.
	
	We claim that in the resulting \Fary\ embedding of $T'$, the only open edge
	that intersects the open segment $xy$ is $zb$. It suffices to prove that $z$ is not a reflex vertex in the quadrilateral $xbyz$; note that $b$ is not a reflex vertex in $xbyz$, since $bx$ and $by$ are crossing edges. In Case~(3a), the existence of the
	edges $za$ and $zc$ ensures that, in the \Fary\ embedding of $T'$,
	$xbyz$ is convex. In Case~(3b), the triangle $zxa$ is convex and $xbyz$ is contained in this triangle, therefore $z$ is convex. In either case, removing $zb$ from the \Fary\ embedding of $T'$ and replacing it with $xy$ yields the desired \Fary\ embedding of $T$.
	
	\paragraph{Edges on $C$.}
	
	If $T$ contains an edge $xy$ that lies on $C$, then we treat it as we treated flippable edges. In this case, $xy$ is incident to two triangles $xyz$ and $yxb$ with $z\in C^+$ and $b\in C^-$. We replace $xy$ with an edge $zb$ to obtain a new triangulation $T'$ with the same number of vertices of $T$ and one less non-crossing edge. We apply induction and get a \Fary\ embedding of $T'$, in which $z$ and $b$ are
	on opposite sides of $Y$ and $x$ and $y$ are on $Y$, hence 	neither $z$ nor $b$ is a reflex vertex of the quadrilateral $xzyb$.
	Thus, removing $zb$ and adding $xy$ gives a \Fary\ embedding of $T$.
	
	\paragraph{The Base Case.}
	
	We are left with the case in which $T$ is a triangulation
	with no separating triangles, no contractible edges, no flippable
	edges, and no edge contained in $C$.  If $T$ is the complete graph
	on three or four vertices, then the proof is trivial,
	so we may assume that $T$ has at least 5 vertices.
	
	\begin{claimx} \obslabel{unmarked}
	Any unmarked edge $xy \in C^+$ is on the boundary of two faces $xyz$ and $yxb$ where $z,b\in\{C \cup C^-\}$.
	\end{claimx}
		
	\begin{proof}
	Since $xy$ is not contractible, at least one of $xyz$ and $yxb$ is a
	crossing triangle, so at least one of $z$ and $b$, say $b$, is in $C^-$.
	Suppose then, for the sake of contradiction, that $z\in C^+$. Since neither $zx$ nor $yz$ is contractible,
	they must be incident to crossing faces $xza$ and $zyc$, respectively.
	If $a=b=c$, then $T$ is the complete graph on four vertices,
	which we have already ruled out.  Therefore, assume without loss of
	generality that $b\neq c$.  Since $T$ contains no separating triangles,
	we have $a\neq c$, otherwise $xya$ would separate $z$ from $b$. Similarly, $a\neq b$, otherwise $byz$ would separate $x$ from $a$.	
	This leaves us in the situation in which we have distinct vertices $x$,
	$y$, $z$, $a$, $b$, and $c$ such that $xyz\in C^+$, such that $xyb$, $zyc$, and $xza$
	are crossing faces of $T$, and such that $xyz$ is a non-crossing face of $T$.
	Then at least one of $xy$, $yz$, or $zx$ is a flippable edge. This contradiction proves the observation.
\end{proof}	

	Symmetrically, every unmarked edge $xy\in C^-$ is incident to two faces
	$xyz$ and $yxb$ with $z,b\in \{C \cup C^+\}$.  This implies that no face of $T$ contains more than one unmarked edge.
	
	Thus, every unmarked edge of $T$ is incident to two faces that intersect $C$. The union of these two faces is a quadrilateral whose boundary consists of four edges that intersect $C$.
	Let $\tilde{G}$ denote the embedded plane graph obtained by removing all unmarked edges
	from $T$.  By \thmref{dujmovic-frati}, we know that $\tilde G$ has
	a \Fary\ embedding $G$ whose edges and vertices intersect $Y$ in the
	same order as they intersect $C$ in $\tilde{G}$. We have the following.
	
	\begin{claimx} \label{claim-a-graph}
		$G$ is an A-graph.
	\end{claimx}
	
	\begin{proof}
In order to	prove the claim, we check each of the properties of an A-graph.
	
	\begin{compactenum}
		\item The removal of unmarked edges and the fact that $T$ has no
		edge entirely on $C$ ensures that every edge of $G$ intersects $Y$
		in exactly one point.
		\item Because no face of $T$ is incident to more than one unmarked edge,
		each face of $G$ is a quadrilateral or a triangle.  
		\item A quadrilateral face $q=abcd$ appears in $G$ when we remove the unmarked edge $ac$ from $T$. This, and the fact that every edge of $q$ intersects $Y$, ensures that $a$ or $c$ is a reflex vertex of $q$.
		\item The only triangular faces of $G$ are those consisting
		of three marked edges, which necessarily have one vertex in
		each of $Y$, $L$, and $R$.
		\item Since $T$ has no edge on $C$, every vertex of $T$ on $C$
		is incident to two triangular faces (one above and one below)
		each having three marked edges. These faces are still present in $G$.
	\end{compactenum}
This concludes the proof of the claim.
\end{proof}	

	%
	%The graph $Q^*$ has two triangular faces $vab$
	%   and $vcd$ incident to $v$ such that $ab=r_{i-1}$ and $cd=r_{i+1}$.
	%   Split $v$ into two vertices $x\in L$ and $y\in R$ joined by the edge
	%   $xy$, make $x$ adjacent to all neighbours of $v$ in $R$, and make
	%   $y$ adjacent to all neighbours of $v$ in $L$. See \figref{split}.
	%   This splitting operation eliminates the triangular faces $vab$ and
	%   $vcd$ and introduces the quadrangular faces $xyab$ and $yxcd$.
	%
	%   \begin{figure}
	%      \centering{
	%        \begin{tabular}{c|c}
	%            \includegraphics{figs/split} & \includegraphics{figs/split-outer}
	%        \end{tabular}}
	%      \caption{Splitting vertices on $C$ in the proof of
	%      \thmref{main}.}
	%      \figlabel{split}
	%   \end{figure}

        We would now like to apply \thmref{a-graph} to 
	obtain a \Fary\ embedding of $G$ in which, for each $i\in\{1,\ldots,k\}$, the intersection of $r_i$
	with $Y$ is at $(0,y_i)$ and the appropriate vertices on the outer face of $T$ map to the vertices of the triangle $\Delta$.  Before doing so, we must first prescribe an outer face $\Delta'$ for the \Fary\ embedding of $G$.  If the outer face of $G$ is a 3-cycle, then we use $\Delta'=\Delta$.  Otherwise, suppose the outer face of $G$ is a 4-cycle $\alpha a\beta\gamma$ and $\alpha\beta$ is an unmarked edge of $T$.  In this case, the locations of $\alpha$, $\beta$, and $\gamma$ are given by the three vertices of $\Delta$ (with $\alpha$ and $\beta$ both on the same side of $Y$).  The position of $a$ is either determined by $y_i$, in the case where $a=r_i$ for some $i\in\{1,\ldots,k\}$ or the position of $a$ is determined by the positions of $\alpha$ and $\beta$ and the values $y_{i}$
and $y_j$ where $r_i=\alpha a$ and $r_j=\beta a$.

	In this way, we can apply \thmref{a-graph} to
	obtain a \Fary\ embedding of $G$ in which the intersection of $r_i$
	with $Y$ is at $(0,y_i)$.  Each internal edge $ac$ of $T$ not in $G$
	corresponds to a quadrangular face $q=abcd$ of $G$ in which $a$ or $c$ is a
	reflex vertex.  Therefore, the edge $ac$ can be added to the embedding
	without introducing crossings.  A single external edge $\alpha\beta$
	on the outer face of $T$ might not appear in $G$. In this case the outer
	face of $G$ is a quadrilateral $q'=\alpha a \beta \gamma$ in which $a$ is
	a reflex vertex, so the segment $\alpha\beta$ lies outside of $q'$, and the edge $\alpha\beta$ can therefore be added to the embedding of $G$
	without introducing crossings. Therefore reinserting each edge of $T$ not in $G$ gives the desired \Fary\ embedding of $T$ (and the choice of $\Delta'$ ensures that the outer face of this embedding is $\Delta$). 
        This concludes the proof of \thmref{main}.
\end{proof}
	
We are finally ready to prove \thmref{our-bang}. Given an embedded plane graph $G$, a collinear set $S$ in $G$,
	and any $y_1'<\cdots<y_{|S|}'$, we need to prove that $G$ has
	a \Fary\ embedding in which the vertices in $S$ are embedded at
	$(0,y_1'),\ldots,(0,y_{|S|}')$. We can assume that $G$ is a triangulation. Indeed, if it is not,  edges can be added to it so that it becomes a triangulation, while preserving the property that $S$ is collinear set; after constructing a \Fary\ embedding in which the vertices of $S$ are embedded at $(0,y_1'),\ldots,(0,y_{|S|}')$ the inserted edges can be removed obtaining the desired \Fary\ embedding of the initial graph. Assume hence that $G$ is a triangulation. 	 \thmref{dujmovic-frati} implies
	that there exists a Jordan curve $C$ that is admissible for $G$
	and that contains all the vertices of $S$ in some order, say
	$v_1,\ldots,v_{|S|}$.  The curve $C$ intersects a subset of the edges
	and vertices of $G$ in some order $r_1,\ldots,r_k$.  We choose any
	sequence $y_1<\cdots<y_k$ so that, for all $i\in\{1,\ldots,k\}$ and
	$j\in\{1,\ldots,|S|\}$, $y_i = y_j'$ if $r_i=v_j$.  We select
	any triangle $\Delta$ that is compatible with $r_1,\ldots,r_k$ and
	$y_1,\ldots,y_k$ and choose $\epsilon = \min\{(1/3)(y_{i+1}-y_{i}):
	i\in\{1,\ldots,k-1\}\}$.  \thmref{main} then gives us a \Fary\
	embedding of $G$ in which the vertices in $S$ are embedded at $(0,y_1'),\ldots,(0,y_{|S|}')$, as required by \thmref{our-bang}.











\section{Open Problems}

In this paper we proved that every collinear set is a free set. Several problems concerning collinear and free sets remain open. Here we mention our favorite two.

Let $f(n)$ be the minimum, over all $n$-vertex planar graphs $G$, of the size of the largest collinear set in $G$. What is the growth rate of $f(n)$? The best known bounds are $f(n)\in\Omega(\sqrt{n})$ and $f(n)\in \mathcal{O}(n^\sigma)$, for $\sigma < 0.986$ \cite{bose.dujmovic.ea:polynomial,ravsky.verbitsky:on}. Our results prove that $f(n)$ is also the minimum size of the largest free set over all $n$-vertex planar graphs; this makes determining the growth rate of $f(n)$ even more relevant. For example, any improvement in the lower bound would immediately give an improved result for untangling planar graphs.
%
%\begin{op}
%	What is the growth rate of $f(n)$?
%\end{op}

We find it interesting to understand whether our main theorem,
\thmref{our-bang}, can be generalized so that the $y$-coordinates are arbitrarily prescribed not only for the vertices on~$Y$, but also for the crossing points of the edges with $Y$. Note that \thmref{main} {\em almost} gives this generalization, as every edge crossing $Y$ is at most $\epsilon$ away from its prescribed crossing point, for any arbitrarily small $\epsilon$. 

\section*{Acknowledgement}

Part of this research was conducted during the 5\textsuperscript{th} and the 6\textsuperscript{th} Workshops on Geometry and Graphs, held at the Bellairs Research Institute, March 5--10, 2017 and March 11--16, 2018.  We are grateful to the organizers and participants for providing a stimulating research environment.
%
%This work was initiated at the \emph{Fifth Workshop on Geometry and
%	Graphs}, held at the Bellairs Research Institute, March 5--10, 2017 and
%was picked up again at the \emph{Sixth Workshop on Geometry and Graphs},
%March 11--16, 2018.  We are grateful to the organizers and the other
%workshop participants for providing a stimulating research environment.

%\newpage
\bibliographystyle{plain}
\bibliography{freecoll}
\goodbreak
\end{document}









