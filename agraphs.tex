%!TEX root = main.tex
\section{A-Graphs}
\seclabel{quad}
\seclabel{quadrangulations}

In this section, we study a special class of graphs that are closely
related to quadrangulations in which every edge crosses $Y$. (See \figref{a-graph} for an example.)

\begin{defn}\deflabel{a-graph}
	An \emph{A-graph}, $G$, is a plane straight-line graph with $n\ge 3$ vertices that has the following properties:
	\begin{compactenum}
		\item Every edge of $G$ intersects $Y$ in exactly one
                  point, possibly an endpoint.
                  \label{p1}
		\item Every face of $G$, including the outer face, is
                  a quadrilateral or a triangle (not containing any
                  disconnected components inside).
		\item Every quadrilateral face of $G$ is non-convex.
		\item Every triangular face contains one vertex
 on $Y$, one in $L$, and one in $R$.  
		\item Every vertex $v$ on $Y$ is incident to precisely
		two triangular faces, one ``above $v$'', which
                contains the line segment between $v$ and
                $v+(0,\epsilon)$ for some $\epsilon>0$, and one ``below
                $v$'', containing the line segment between $v$ and $v-(0,\epsilon)$ for some $\epsilon >0$.
                  \label{p-last}
	\end{compactenum}
\end{defn}

%\begin{wrapfigure}[11]{r}{.3\textwidth}
%		\Vspace{-1mm}
\begin{figure}
		\centering{\includegraphics[scale = 0.95]{figs/a-graph-new}}
		\caption{An A-graph with 2 vertices on $Y$.}
		\figlabel{a-graph}
\end{figure}
%	\end{wrapfigure}
	
In the special case where $G$ has no vertices in $Y$, the graph $G$ is a quadrangulation in which every edge crosses $Y$. Further, Property~5 applies even if $v$ is on the outer face of $G$ (in which case it implies that the outer face of $G$ must be a triangle).
Some additional properties of $G$ follow from % \defref{a-graph}:
Properties~\ref{p1}--\ref{p-last}.
\begin{compactenum}\setcounter{enumi}{5}
	\item $G$ is connected.
	\item Every vertex of $G$ has degree at least~2.   
	\item If $n\ge 4$, then every vertex in $Y$ has degree at least~3. 
\end{compactenum}
Property~6 follows directly from Property~2.
Property~7 follows from the fact that every vertex is incident to at
least one face and every face is a simple cycle.
Property~8 follows from the fact that every vertex on $Y$ is incident
to at least two triangular faces, which involve at least 4 vertices,
unless $n=3$.
(Property~3 and 4 are actually redundant---Property 3 follows from Properties~1 and 5; Property~4 follows from Property~1.)

%We will show that every A-graph $G$ has a \Fary\ drawing with prescribed
%intersections with $Y$ and a prescribed outer face.  Since the outer
%face of an A-graph can be a triangle or a quadrilateral, in the following
%theorem, $\Delta$ is a triangle or quadrilateral defined as follows:
%\begin{enumerate}
%   \item If $(0,y_1)$ is a vertex of $G$, then $\Delta$ is a triangle
%   with one vertex at $(0,y_1)$ and the opposite edge edge crossing $Y$ at $y_m$.
%
%   \item If $(0,y_m)$ is a vertex of $G$, then $\Delta$ is a triangle
%   with one vertex at $(0,y_m)$ and the opposite edge crossing $Y$ at $y_1$.
%
%   \item Otherwise,
%      $\Delta$ is a quadrilateral whose edges cross $Y$ at $y_1$, $y_a$,
%      $y_b$, and $y_m$, where $e_1$, $e_a$, $e_b$, and $e_m$ are the four edges on the outer face of $G$
%\end{enumerate}

In the following theorem,
we will show that every A-graph $G$ has a \Fary\ drawing with prescribed intersections with $Y$ and a prescribed outer face.

\begin{figure*}
   \begin{center}\begin{tabular}{ccc}
      \includegraphics{figs/outerface-cases-2} & 
      \includegraphics{figs/outerface-cases-3} & 
      \includegraphics{figs/outerface-cases-4} \\
      (a) & (b) & (c)
   \end{tabular}\end{center}
   \caption{The three possibilities for the outer face in
     \thmref{a-graph}.
%(In (a) and (b), we have assumed that the edges $e_1$ and $e_m$ are on the
%outer face, although the order of edges that are
%incident to the same vertex on $Y$ is not specified in the theorem.)
   }
   \figlabel{outerface-cases}
\end{figure*} 

\begin{thm}\thmlabel{a-graph}\ %\newline
\begin{compactitem}
\item Let $G$ be an A-graph.
\item Let $e_1,\ldots,e_m$ be the sequence of edges in $G$,
  in the order they are intersected by $Y$. Ties between edges having
  a common endpoint on $Y$ are broken arbitrarily,
except that $e_1$ and $e_m$ are
always edges on the outer face.
\item Let $y_1\le\cdots\le y_m$ be any sequence of numbers where, for
  each $i\in\{1,\ldots,m-1\}$, $y_i=y_{i+1}$ if and only if $e_i$
  and $e_{i+1}$ have a common endpoint in $Y$.
		% \item Let $\Delta$ be a triangle or quadrilateral, where:
		% \begin{compactenum}
		% 	\item If $(0,y_1)$ is a vertex of $G$, then $\Delta$ is a triangle
		% 	with a vertex at $(0,y_1)$ and the opposite edge crossing $Y$ at $y_m$.
			
		% 	\item If $(0,y_m)$ is a vertex of $G$, then $\Delta$ is a triangle
		% 	with a vertex at $(0,y_m)$ and the opposite edge crossing $Y$ at $y_1$.
			
		% 	\item Otherwise, $\Delta$ is a quadrilateral whose edges cross $Y$ at $y_1$, $y_a$,
		% 	$y_b$, and $y_m$, where $e_1$, $e_a$, $e_b$, and $e_m$ are the four edges on the outer face of $G$.
		% \end{compactenum}
\end{compactitem}
Then $G$ has a \Fary\ drawing in which the intersection
between $e_i$ and $Y$ is the single point $(0,y_i)$, for each
$i\in\{1,\ldots,m\}$.

Moreover, the shape  $\Delta$ of the outer face can be prescribed,
subject only to the constraint that $\Delta$
has to be consistent with the graph and the data $y_1,\ldots,y_m$.
Specifically, we have three possibilities, which
 are illustrated in \figref{outerface-cases}.

\begin{compactenum}[a)]
\item If the outer face of $G$ is a triangle containing the lowest
  vertex on $Y$, then $\Delta$ must be a triangle with a vertex at
	$(0,y_1)$, and the opposite edge crosses $Y$ at $(0,y_m)$.
\item Symmetrically, if the outer face of $G$ is a triangle containing
  the highest vertex on $Y$, then $\Delta$ must be a triangle with a vertex
	at $(0,y_m)$, and the opposite edge crosses $Y$ at $(0,y_1)$.
\item Otherwise, the outer face of $G$ is a quadrilateral.  Let $e_1$,
  $e_a$, $e_b$, and $e_m$ be the %four
  edges of the outer face.  Then $\Delta$ has to be a quadrilateral
	whose edges cross $Y$ at $(0,y_1)$, $(0,y_a)$, $(0,y_b)$, and $(0,y_m)$.
\end{compactenum}
%If the outer face is a quadrilateral, the drawing with specified outer
%face is unique. If the outer face is a triangle, there is one degree
%of freedom.
\end{thm}

It would have been more natural to represent the intersection of $Y$ with $G$
as a mixed sequence of vertices and edges. However, to simplify the
statement of the theorem and its proof, we have chosen to specify the
desired drawing by a number $y_i$ for every edge, subject to equality
constraints. %The reason for this choice will shortly become apparent.
The convention in Condition~2 about $e_1$ and $e_m$ being boundary
edges is introduced only for
 notational convenience.

The rest of this section is devoted to proving \thmref{a-graph}. We
are going to prove \thmref{a-graph} in its strongest form, in which the
outer face $\Delta$ is prescribed.  We begin by making some simplifying
assumptions, all without loss of generality.
%
% The truth of the theorem is independent of the order of edges that are
% incident to the same vertex on $Y$. For notational convenience,  we
% can therefore assume that $e_1$ and $e_m$ are edges on the outer face.
% With this assumption, the three possibilities, 4(a--c), for the
% outer face are illustrated in \figref{outerface-cases}.
%
% The theorem does not specify the order of edges that are
% incident to the same vertex on $Y$. For notational convenience,  we
%  refine the edge order such that that $e_1$ and $e_m$ are
%  always edges on the outer face. (This convention has been used in
%  \figref{outerface-cases}.)
%
First, we assume w.l.o.g.\ that $\Delta$ and all vertices of $G$
are contained in the strip $[-1,1]\times(-\infty,+\infty)$.  This can
be achieved by a uniform scaling.  Second, if the outer face of $G$
is a quadrilateral, we assume w.l.o.g.\ that the common vertex
of $e_1$ and $e_m$ in the given drawing of $G$ is in $L$, as in
Figures~\ref{fig:a-graph} and \ref{fig:outerface-cases}c, and the
vertex of desired output shape $\Delta$ incident to $e_1$ and $e_m$
is also in $L$; this can be achieved by a reflection of $G$ or $\Delta$
with respect to $Y$.


% this is
% also not a loss of generality, since if the vertex of $\Delta$
% incident to both $e_1$ and $e_m$ is in $R$, then $\Delta$ can be
% reflected with respect to $Y$, obtaining a quadrilateral $\Delta'$
% whose vertex incident to both $e_1$ and $e_m$ is in $L$, then a \Fary\
% drawing of $G$ can be constructed in which the outer face is
% $\Delta'$, and finally the \Fary\ drawing can be reflected with
% respect to $Y$, thus obtaining a \Fary\ drawing of $G$ in which the
% outer face is $\Delta$.

If $m=3$ or $m=4$, then $G$ is a 3- or a 4-cycle, respectively, hence it suffices to draw it as $\Delta$. Therefore we assume, from now on, that $m\ge 5$.  
%%%%% copia fino a qui %%%%%%

% Before continuing, we pause to fully specify the ordering
% $e_1,\ldots,e_m$. This ordering is unambiguous except where
% some vertex $v\in Y$ is incident to several edges
% $e_{i},\ldots,e_{i+d}$, where $d\ge 2$ by Property~8 of A-graphs. Refer to \figref{ab}.  In this case we partition $v$'s neighbors into two
% sets $\alpha_1,\ldots,\alpha_k\in L$ and $\beta_1,\ldots,\beta_\ell\in
% R$, where $\alpha_1,\ldots,\alpha_k$ are ordered clockwise around $v$
% and $\beta_1,\ldots,\beta_\ell$ are ordered counterclockwise.  We then use
% the convention that $e_i,\ldots,e_{i+k-1}=v\alpha_1,\ldots,v\alpha_k$
% and $e_{i+k},\ldots,e_{i+d}=v\beta_1,\ldots,v\beta_\ell$.

We will describe the desired \Fary\ drawing by assigning a slope $s_i$
to each edge $e_i\in E(G)$.   Since there can be no vertical edges,
each slope $s_i$ is well-defined. We have $m=|E(G)|$ slope variables,
$s_1,\ldots,s_m$. 
We can see that these variables determine the drawing:
 Since every edge $e_i$ contains the point $(0,y_i)$,
the slope $s_i$ fixes the line through $e_i$.  Since every vertex $v$
not on $Y$ is incident to at least two edges that contain distinct points
on $Y$, the location of $v$ is fixed by any two of $v$'s incident edges.  
(The location of each vertex on
$Y$ is fixed by definition.)  Our strategy is to construct a
system of $m$ linear equations in the $m$ variables $s_1,\ldots,s_m$,
and to show that
this system is feasible and that its solution gives the desired \Fary\
drawing of $G$.

A necessary condition for the slopes to determine a F\'ary drawing of
$G$ is that the %supporting lines of 
edges 
with a common vertex should be concurrent. Let $v$ be a vertex 
not on $Y$, and let $e_i, e_j, e_k$ be three edges incident to $v$.
The fact that the supporting lines of $e_i$, $e_j$, and $e_k$
meet at a common point (the location of $v$) is expressed by the following
\emph{concurrency constraint} in terms of the slopes $s_i,s_j,s_k$:
\ifSODA
\begin{align}\eqlabel{slope0} 
\left|
\begin{matrix}
1&1&1\\
s_i&s_j&s_k\\
y_i&y_j&y_k
\end{matrix}
\right|&=
({y_j{-}y_k}) s_i + ({y_k{-}y_i}) s_j 
         + ({y_i{-}y_j})s_k  %\\&
                               = 0
%\nonumber                               
\end{align}
\else
\begin{equation}\eqlabel{slope0} 
\left|
\begin{matrix}
1&1&1\\
s_i&s_j&s_k\\
y_i&y_j&y_k
\end{matrix}
\right|=
({y_j-y_k}) s_i + ({y_k-y_i}) s_j 
+ ({y_i-y_j})s_k  = 0
\end{equation}
\fi
Since $y_1,\ldots,y_m$ are given, this is a linear equation
in $s_1,\ldots,s_m$.
Writing this equation for all triplets of edges incident to a common
vertex $v$ will include many redundant equations. Indeed,
if $v$ has degree $d_v$,
it suffices to take $d_v-2$ equations: For each vertex $v\in V(G)$, we choose two fixed
incident edges $e_i$ and $e_j$ and run $e_k$ through the remaining
$d_v-2$ edges, specifying that $e_k$ should go through the common vertex
of $e_i$ and $e_j$.
%We call the resulting collection of $\sum_{v\in V(G)\setminus Y} d_v-2$ equations the \emph{concurrency constraints}.

Whenever convenient, we will use edges of $G$
as indices so that, if $e=e_i$ is an edge of $G$, then $s_e=s_i$
and $y_e=y_i$.  Further, if $e$ is a line segment that
intersects $Y$ in a point, we will use $y_e$ to denote the $y$-coordinate
of the intersection of $e$ and $Y$ and $s_e$ to denote the slope of~$e$. %'s supporting line.

%It will be important to have as many equations as variables;
%thus, 

We now introduce additional equations for the edges that emanate from a
vertex on $Y$; refer to \figref{proportional}.
Suppose that a vertex $v\in Y$ is incident to edges $a_1,\ldots,a_k\in L\cup Y$ 
and $b_1,\ldots,b_\ell\in Y\cup R$, ordered from bottom to top as in \figref{ab}.

\begin{figure*}
  \begin{center}
    \includegraphics{figs/proportional}
  \end{center}
  \caption{The proportionality constraints on slopes of edges incident to a vertex $v\in Y$.}
  \figlabel{proportional}
\end{figure*}
From Property~4 of A-graphs we have $k,\ell\ge1$ and in addition
 $k+\ell\ge 3$ by Property~8.
Let us first look at the slopes on the right side.
We want these slopes to be increasing:
$s_{b_1} < s_{b_2} < \dots  <s_{b_\ell}$. We stipulate a stronger
condition:
We require that the slopes
$s_{b_2}, \dots, s_{b_{\ell-1}}$ partition the interval
$[s_{b_1},s_{b_\ell}]$ in fixed proportions. In other words:
\begin{equation}
\label{eq:proportion}
s_{b_i} = s_{b_1} + \lambda_i(s_{b_{\ell}}-s_{b_1}),
\end{equation}
for some fixed sequence $0<\lambda_2<\cdots<\lambda_{\ell-1}<1$.

For example, we might set $\lambda_i := (i-1)/(l-1)$.
This gives $\ell-2$ equations, for $\ell\ge 2$. Similarly, we get
$k-2$ equations for the slopes
$s_{a_1}, \dots, s_{a_{k}}$ of the edges on the left side, for $k\ge 2$.
In addition, for $k\ge 2$ and $\ell\ge 2$, we require that the \emph{range} of
slopes
on the two sides are in a fixed proportion:
\begin{equation}
\label{eq:proportion2}
s_{a_1}-s_{a_{k}} = \mu (s_{b_{\ell}}-s_{b_1}),
\end{equation}
for some fixed value $\mu>0$.

We call the equations
\thetag{\ref{eq:proportion}--\ref{eq:proportion2}} the
\emph{proportionality constraints}.
There are $(k+\ell)-3$ such equations for the $k+\ell$ slopes, hence
we have three degrees of freedom for the slopes incident to a vertex.
\figref{proportional} illustrates these  degrees of freedom:
Namely, we can shear the edges on the right side vertically, adding the same constant to all
slopes. We can independently shear all edges on the left side.
In addition, we can vertically scale {all} lines jointly (both to
the left and to the right), multiplying all slopes by the same constant factor.
If this factor is negative, we would reverse the order of the
slopes, simultaneously on the left and on the right. We will later see
that this undesirable possibility is prevented in conjunction with
other constraints that we are going to impose. We can already observe
that any two slopes on one side determine all remaining slopes on that side. Moreover, the range of slopes on the other side ($s_{a_1}-s_{a_{k}}$ or $s_{b_{\ell}}-s_{b_1}$) is also determined.
%
The notations $\lambda_i$ and $\mu$ are here used in a local sense;
for a different vertex $v$, we may choose different constants.
%\begin{figure}
%	\centering{\includegraphics{figs/proportional}}
%	\caption{The degrees of freedom provided by the proportionality constraints}
%	\figlabel{proportional}
%\end{figure}
%We have the following.
\begin{lem} \label{le:number-of-equations}
The total number of equations \thetag{\ref{eq:slope0}},~\thetag{\ref{eq:proportion}}, and~\thetag{\ref{eq:proportion2}} is $m-4$.
\end{lem} 
\begin{proof}
Let $n=|V|$ and let $n_0$ be the number
of vertices on $Y$. Assume that $G$ has $f_3$ triangular and $f_4$
quadrangular faces.

We have two triangles for every vertex on $Y$ (Properties 4 and 5 of A-graphs):
\begin{equation}
\label{eq:f3}
f_3 = 2n_0
\end{equation}
Euler's formula gives
\begin{equation}
\label{eq:Euler}
n + f_3+f_4 = m+2.
\end{equation}
Double-counting of edge-face incidences leads to the relation
\begin{equation}
\label{eq:edge-face}
3f_3+4f_4=2m.
\end{equation}
Denoting the degree of a vertex $v$ by $d_v$,
we have $d_v-3$ equations for each of the $n_0$ vertices $v$ on $Y$. For each of the 
$n-n_0$ vertices $v$ not on $Y$, 
we have $d_v-2$ equations.
The total number of equations is therefore
\ifSODA
\begin{align*}
% \label{eq:number-equations2}
P &= 
\sum_{v\in V\cap Y}(d_v-3)+
    \sum_{v\in V\cap(L\cup R)}(d_v-2)
  \\&
=
\sum_{v\in V}(d_v-2)-n_0
=
2m-2n-n_0.
\end{align*}
\else
\begin{align*}
% \label{eq:number-equations2}
P &= 
\sum_{v\in V\cap Y}(d_v-3)+
    \sum_{v\in V\cap(L\cup R)}(d_v-2)
=
\sum_{v\in V}(d_v-2)-n_0
=
2m-2n-n_0.
\end{align*}
\fi
Using \thetag{\ref{eq:f3}--\ref{eq:edge-face}}, this can be
simplified to
\ifSODA
\begin{align*}
  P&
      =
   2m-2n-n_0\\
 &
                 = 2m -2n -(2f_3+2f_4) +(2f_3+2f_4)-n_0\\
&= 2m -2(n +f_3+f_4) +\tfrac12(4f_3+4f_4-f_3)\\
 &= 2m -2(m+2) +\tfrac12(2m) = m-4.
   \qedhere
\end{align*}
\vspace{-2,2\baselineskip}
\hrule height 0pt
\else
\begin{align*}
  P%&
      =
   2m-2n-n_0%\\
 &
                 = 2m -2n -(2f_3+2f_4) +(2f_3+2f_4)-n_0\\
&= 2m -2(n +f_3+f_4) +\tfrac12(4f_3+4f_4-f_3)\\
 &= 2m -2(m+2) +\tfrac12(2m) = m-4.
   \qedhere
\end{align*}
\fi
%This concludes the proof of the lemma.
\end{proof}

To achieve the desired number $m$ of equations, we add
four \emph{boundary equations}.  If the outer face is a quadrilateral,
the desired slopes of the boundary edges already give us four equations:
We set
the slopes $s_1$, $s_a$, $s_b$, and $s_m$ of the boundary edges $e_1$, $e_a$,
$e_b$, and $e_m$ to the fixed values of the slopes of the edges of
$\Delta$.

If the outer face is a triangle, the shape $\Delta$ gives us
only three constraints for the slopes of the three edges $e_1$, $e_a$,
$e_m$.  If the triangle is $\alpha\beta\gamma$ with $\gamma\in Y$,
we arbitrarily pick another (non-boundary) edge $e_b$ incident to $\gamma$
and set its slope $s_b$ to an appropriate fixed value; this value has
to be either larger or smaller than each of $s_1$, $s_a$, and $s_m$
depending on whether $\gamma$ is the topmost or the bottommost point
on $Y$ and whether $e_b$ lies in $L$ or $R$. Together with the
proportionality constraints, this effectively pins \emph{all} slopes
incident to $\gamma$ to fixed values.

In both cases, we get 4 equations of the form
\begin{equation}
  \label{eq:boundary}
  s_i = h_i, %\text{  for $i=1,a,b,m.$}
\end{equation}
where $i\in\{1,a,b,m\}$.


Altogether, we now have a system of $m$ linear equations
in the $m$ unknowns $s=(s_1,\ldots,s_m)$, which we can write
compactly as
$A\cdot s = b$, with a square matrix $A$ whose entries come from
\thetag{\ref{eq:slope0}--\ref{eq:proportion2}}
and \eqref{eq:boundary}.
%are the variables we wish to solve for, and $b$ is a column $m$-vector
%whose entries also come from \eqref{slope}.  
Only four entries of
the right-hand side vector
$b$
are non-zero, due to the four boundary equations.
We will show that $A\cdot s=b$ has a unique
solution and that this solution gives a \Fary\ drawing of $G$.

\subsection{Setting the Proportionality Constraints}
\label{sec:setting}

% Our plan is to morph the given drawing of $G$
%into the desired drawing.

Our plan is to construct the desired drawing by a continuous morph, 
starting from the given drawing of $G$. Since the proportioonality 
constraints are not part of the output specification but were 
artificially added to achieve the right number of equations, we can make 
our life easy by just setting their coefficients so that they are 
satisfied by the initial drawing.
Specifically, the statement of \thmref{a-graph} assumes that $G$ is a
\Fary\ drawing.  In this drawing, every edge $e$
%intersects $Y$
%in a single point $(0,y_e')$ and
has a slope $s_e'$.
We use these slopes to set the
coefficients in the proportionality constraints.
Consider a
vertex $v\in Y$, incident to edges $a_1,\ldots,a_k$ and $b_1,\ldots,b_\ell$
as described above.
In the notation used
in \eqref{eq:proportion}, we set
\[
\lambda_i = (s_{b_i}'-s_{b_1}')/(s_{b_\ell}'-s_{b_1}') .
\]
The coefficients for the edges
 $a_1,\ldots,a_k$ on the left side are set similarly.
If $k\ge2$ and $l\ge 2$,
we set
\[
\mu = (s_{a_1}'-s_{a_k}')/(s_{b_\ell}'-s_{b_1}')
\]
 in \eqref{eq:proportion2}.
This ensures that the initial slopes $s_{1}',\ldots,s_{m}'$ satisfy the
proportionality constraints.


\subsection{Ordering constraints}

We define a relation $\prec$ on the edges of $G$, where $e_1 \prec e_2$ if and only if
\begin{itemize}
	\item $y_{e_1} < y_{e_2}$ and $e_1$ and $e_2$ have a common endpoint $v\in L$; or
	\item $y_{e_1} > y_{e_2}$ and $e_1$ and $e_2$ have a common endpoint $v\in R$.
\end{itemize}
We say that a vector $s=(s_1,\ldots,s_m)$ \emph{satisfies the ordering
	constraints} if $s_{e_1} < s_{e_2}$ for every pair $e_1,e_2\in E(G)$
such that $e_1\prec e_2$. This definition captures the condition that vertices of $G$ in $L$ (respectively, $R$) should be drawn so that they remain in $L$ (respectively, $R$), as in the following. 

\begin{obs}\obslabel{left-right}
If a solution $s$ to $A\cdot s=b$ satisfies the ordering constraints, then every vertex that is in $L$ (in $R$) in $G$ is also in $L$ (respectively in $R$) in the drawing corresponding to $s$. 
\end{obs}

\begin{proof}	  
Consider any vertex $v$ that is in $L$ in $G$ and that is incident to
(at least) two edges $e_1$ and $e_2$ with $y_{e_1} < y_{e_2}$, and hence $e_1 \prec e_2$. Since $s$ satisfies the ordering constraints we have $s_{e_1} < s_{e_2}$, hence the lines with slopes $s_{e_1}$ and $s_{e_2}$ through $(0,y_{e_1})$ and $(0,y_{e_2})$, respectively, meet in $L$. The argument for the vertices in $R$ is analogous. 
\end{proof}	  

By construction, the slopes $s_{e_1}',\ldots,s_{e_m}'$ of edges
in $G$ satisfy the ordering constraints, so the relation $\prec$ is acyclic.
%  Indeed, $i_1\prec
% \cdots \prec i_r$ implies that, for each $j\in\{3,\ldots,r\}$, $y_{i_j}\in
% (\min\{y_{i_{j-1}},y_{i_{j-2}}\}, \max\{y_{i_{j-1}},y_{i_{j-2}}\})$. Thus,
% a chain in $\prec$ corresponds to a sequence of strictly nested intervals.

\begin{lem}\lemlabel{order-gives-drawing}
	Any solution $s$ to $A\cdot s=b$ satisfying
	the ordering constraints % $\prec$
	yields a
	\Fary\ drawing of $G$ with $\Delta$ as the outer face.
\end{lem}

\begin{proof}
	If $G$ is a plane drawing of a 2-connected graph, then another
	straight-line drawing $G'$ of the same graph $G$ is a \Fary\ drawing provided
	that two conditions are met:
	(i) For every vertex~$v$, the clockwise order of the
	edges around $v$ in $G'$ is the same as in $G$; and
	(ii) in the drawing $G'$, every face cycle of $G$ is drawn without crossings
	(Devillers, Liotta, Preparata, and Tamassia \cite[Lemma~16]{devillers.liotta.ea:checking}).
	
	In our case, $G'$ is a straight-line drawing of $G$ given by a solution
	to $A\cdot s = b$ that satisfies the ordering constraints.  


	First we show that $G'$ satisfies condition (i).
        More specifically, we establish the following stronger
        property for every vertex $v$.
\begin{quote}
  \thetag{$*$}
The edges going to the right from $v$ are the same in $G$ and $G'$,
and their slopes have the same order in $G$ and $G'$.

The same properties hold for the edges to the left.
\end{quote}

      
        We distinguish the following cases:
	\begin{enumerate}
		\item $v\not\in Y$. Since $s$ satisfies the ordering
                  constraints, by \obsref{left-right} we know that
                  $v$ is on the same side ($L$ or $R$)
                  in $G$ and
                  in $G'$.
All incident edges go to one side.
                  This, together with the fact that the
                  orders in which the edges incident to $v$ intersect
                  $Y$ in $G$ and $G'$ agree implies that the
                  slope orders of the edges around $v$ in $G$ and $G'$ agree.
		
		\item
		$v\in Y$, with incident edges $a_1,\ldots,a_k\in
		L\cup Y$ and $b_1,\ldots,b_\ell\in Y\cup R$ as in
                \figref{ab}.
                Again, \obsref{left-right} ensures that
                these edges remain on the same side in $G'$.

\begin{enumerate}
\item If $v$ is a boundary vertex, % say that $v$ is at $(0,y_1)$ in
  % $G'$,
  then the boundary equations fix the slopes of the two incident
  boundary edges, $a_1, b_1$ or $a_k, b_l$, plus a third edge.  As we
  already observed when the proportionality constraints were defined,
  these constraints then fix the slopes of all edges incident to $v$,
  so that their ordering agrees with that of $G$.
			
\item If $v$ is an interior vertex then,
  %, by \obsref{left-right}, the edges $a_1,\ldots,a_k$ are in $L\cup
  %Y$ and the edges $b_1,\ldots,b_\ell$ are in $R\cup Y$ in
  %$G'$. Further,
  as discussed above, the proportionality constraints ensure that the
  slope order of $v$'s incident edges in $G'$ either matches that of
  $G$ on each side, or it is completely reversed on both sides.
  % , so that the counterclockwise
                                %order of vertices around $v$ in $G'$
                                %is
                                %$a_1,\ldots,a_k,b_\ell,\ldots,b_1$.
  Let us assume for contradiction that the latter case happens:
  \begin{equation}
    \label{eq:not-ordered}
    s_{b_1}\ge s_{b_\ell}
    \text{ and }
    s_{a_k}\ge s_{a_1}
  \end{equation}
  Let $e$ be the third edge of the triangle with edges $a_1$ and
  $b_1$, and let $f$ be the third edge of the triangle with edges
  $a_k$ and $b_\ell$, see \figref{ab}.
			Then the ordering constraints for the endpoints of $e$ imply
			\begin{math}
			s_{b_1}<s_e<s_{a_1}
			\end{math},
			and the ordering constraints for the endpoints of $f$ imply
			\begin{math}
			s_{a_k}<s_f<s_{b_\ell}
			\end{math}.
			Together with \eqref{eq:not-ordered}, this leads to a contradiction.
		\end{enumerate}
\end{enumerate}

\begin{figure}
  \centering
  {\includegraphics[scale = 1]{figs/abx}}
  \caption{The ordering of the edges incident to a vertex $v$ on $Y$.}
  \figlabel{ab}
\end{figure}


From the statement \thetag{$*$}, it is now easy to derive that
$G'$ satisfies condition (ii).
The graph $G$ has triangle and quadrilateral faces.
%
For a triangular face, \thetag{$*$} ensures that the triangle does not
degenerate, and is therefore non-crossing, in $G'$.
%The graph $G$ has two kinds of
%faces:
A quadrilateral face $q$ must be non-convex in $G$ by Property 3 of
A-graphs, and for each vertex, the two incident edges of~$q$ go in
the same direction (left or right).
Thus, Property \thetag{$*$} ensures that $q$ is non-crossing in~$G'$.


% \begin{enumerate}
% 		\item Let $q = abcd$ be a quadrilateral face of
%                   $G$. By Property 3 of A-graphs, $q$ is non-convex in
%                   $G$;


%                   let $c$ be the reflex vertex of $q$ in $G$. Note that $a\notin Y$ and $c\notin Y$ by Properties 1 and 5 of A-graphs, respectively. Conversely, each of $b$ and $d$ might or might not be on $Y$. Then the ordering constraints and the proportionality constraints at $b$ and $d$ ensure that $q$ is non-crossing in $G'$.
		
% %		If $q=abcd$ is a quadrilateral face in $G$ with no vertex on $Y$,
% %		then the ordering constraints imply that $q$ is non-crossing in $G'$.
% %		Otherwise, by Property~3 of A-graphs, $c$ is reflex vertex of $q$ and,
% %		by Property~5 of A-graphs, $c\not\in Y$.  The vertex $a$ opposite $c$
% %		is also not in $T$, so so $b\in Y$ and/or $d\in Y$.  In this case,
% %		the ordering constraints and the proportionality constraints at $b$
% %		and/or $d$ ensure that $q$ is non-crossing in $G'$.
		
% 		\item
% 		For a triangular face $\alpha\beta\gamma$, with $\gamma\in Y$, 
% 		the ordering constraints on the vertices $\alpha$ and $\beta$
% 		ensure that the triangle does not degenerate, and is therefore non-crossing, in $G'$.
% 	\end{enumerate}
	Therefore, by the result of Devillers \etal\ cited above, $G'$
	is a \Fary\ drawing. That $G'$ has $\Delta$ as the outer
	face follows from the inclusion of the boundary equations in
	$A\cdot s = b$.
\end{proof}

Any solution $s$ to $A\cdot s=b$ has the outer face drawn as $\Delta$,
by the boundary equations, and the intersection between $e_i$ and $Y$ is $(0,y_i)$ by construction. Hence, by~\lemref{order-gives-drawing}, ensuring the existence of a solution $s$ to $A\cdot s=b$ satisfying the ordering constraints is enough to prove~\thmref{a-graph}.

\subsection{Strong Ordering Constraints}
\label{strong}

For some $\epsilon > 0$, we say that $s=(s_1,\ldots,s_m)$ satisfies
the \emph{$\epsilon$-strong ordering constraints} if, for each
$i,j\in\{1,\ldots,m\}$ such that $e_i\prec e_j$, the inequality
$s_j-s_i \ge \epsilon$ holds.
% A solution $s$ that satisfies
Clearly, any $s$ satisfying the $\epsilon$-strong ordering constraints
also satisfies the ordering constraints. 
The converse holds, for a suitably small $\epsilon$ (the inequalities
being strict in the definition of ordering constraints). The following
lemma tells us that this $\epsilon$ can be determined by $\Delta$ and by the
sequence $y_1,\ldots,y_m$.
%
%The converse holds when
%equations
%$A\cdot s=b$
%% ~\thetag{\ref{eq:slope0}}--\thetag{\ref{eq:proportion2}}
%are satisfied, for a suitably small $\epsilon$:

\begin{lem}\lemlabel{weak-to-strong}
  If
  $\Delta\subset[-1,1]\times(-\infty,+\infty)$, then
	any solution $s$ to $A\cdot s=b$ that satisfies
	the ordering constraints
	also satisfies 
	the $\epsilon$-strong ordering constraints
	for all $\epsilon\le\min\{\,|y_i-y_j| : e_i\prec e_j\,\}$.
\end{lem}

\begin{proof}
	By \lemref{order-gives-drawing} every vertex is contained in
	 $\Delta$. %\subset[-1,1]\times(-\infty,+\infty)$.
	Hence, every $x$-coordinate is in the interval $[-1,1]$.
	If $e_i\prec e_j$, then the common vertex of $e_i$ and $e_j$ has $x$-coordinate
	$(y_j-y_i)/(s_i-s_j)$.	From $|(y_j-y_i)/(s_i-s_j)|\le 1$ we
	derive $|s_i-s_j|\ge|y_j-y_i| \ge \epsilon$.
\end{proof}

\subsection{Uniqueness of Solutions Satisfying Ordering Constraints}

\lemref{weak-to-strong} and the $\epsilon$-strong ordering
constraints play a crucial role in our proof because they
allow us to appeal to continuity: If the slopes change continuously,
it is impossible 
to violate
the ordering constraints without first violating the
$\epsilon$-strong ordering constraints.
But since the ordering constraints imply the
$\epsilon$-strong ordering constraints,
it is impossible
to violate
the ordering constraints at all.
% by showing that, if $A\cdot s=b$
%were to have some undesireable property, then some function which we
%know to be continuous would have a discontinuity.
An example of this argument will be seen in the following proof.

\begin{lem}\lemlabel{unique}
	If $s$ is a solution to $A\cdot s=b$ that satisfies the ordering
	constraints, % $\prec$, 
	then $s$ is 
	the unique solution to $A\cdot s=b$.
\end{lem}

\begin{proof}
	Assume that $\epsilon$ is fixed so that $0<\epsilon\le\min\{\,|y_i-y_j| : e_i\prec e_j\,\}$.
	
	Suppose, for contradiction, that there is a solution $s$ to
        $A\cdot s=b$ that satisfies the ordering
        constraints, % $\prec$,
	but is not unique.  Since $A\cdot s=b$ is a linear system, it
        must then have a 1-parameter family of solutions $s+\lambda r$,
        $\lambda\in\mathbb R$, for some non-zero $m$-vector $r$.
        %such that, for every
	%$\lambda\in\mathbb R$, $A(s+\lambda r)=b$.
	
	
	Define the continuous (in fact, piecewise linear) function
	\begin{equation*}
	f(\lambda) := \min \{\, (s_j+\lambda r_j)-(s_i+\lambda r_i) : e_i \prec
	e_j\,\}
	.
	\end{equation*}
 Let $\lambda^*$ be the value with the smallest absolute value
	$|\lambda^*|$ such that
	$f(\lambda^*)\le\epsilon/2$. In order to prove that
        $\lambda^*$ exists, it suffices to prove that $f(\lambda)\le
        0$ can be achieved. The vector $r=(r_1,\ldots,r_m)$ has at least four zero entries
	$r_1=r_a=r_b=r_m=0$ since the slopes $s_1$, $s_a$, $s_b$, and $s_m$
	are fixed.
	Since $G$ is connected and $m\geq 5$, there is at least one vertex $v$ with two incident edges $e_k$
	and $e_\ell$ such that $r_k=0$ and $r_\ell\neq 0$. 
	We can thus pick $\lambda$ so that $(s_\ell+\lambda r_\ell)-(s_k+\lambda r_k)=s_\ell-s_k+\lambda r_\ell=0$,
	and then $f(\lambda)\le 0$. It follows that $\lambda^*$ exists.
	
	Now we know that, for any $\lambda$ between $0$ and $\lambda^*$ and for any $i$ and $j$ such that $e_i\prec e_j$, the difference $(s_j+\lambda r_j)-(s_i+\lambda r_i)$ has the same sign as $s_j-s_i$. It follows that the slopes satisfy the ordering constraints throughout
	this interval, but then
	\lemref{weak-to-strong} implies that $f(\lambda^*)\ge\epsilon$, a contradiction.
\end{proof}

%The proof of \lemref{unique} was quite explicit (perhaps overly so)
%in showing the discontinuity caused by the $\epsilon$-strong ordering
%constraints.  In subsequent arguments we will not be quite so explicit.

\subsection{A Parametric Family of Linear Systems}

We now define a parametric family of linear systems
 $A^t\cdot s = b^t$, parameterized by $0\le t\le 1$
 by varying the intersection points $y=(y_1,\ldots,y_m)$
 and the boundary slopes $h=(h_1,h_a,h_b,h_m)$.
 Let us first see how the coefficients $A$ and right-hand sides $b$ of the system change when these data are changed.
% The coefficients
% in $A^t$
% depend on $y=(y_1,\ldots,y_m)$:
% More precisely, 
 The coefficients of the
 concurrency constraints
 \eqref{eq:slope0} depend linearly on $y$,
 whereas
 the
 proportionality constraints 
 \thetag{\ref{eq:proportion}--\ref{eq:proportion2}} remain unchanged,
 and the boundary constraints~\eqref{eq:boundary} have just the constant coefficient~1.
In the right-hand sides $b^t$,
the four nonzero entries
%depend linearly on 
are the four slopes
$h=(h_1,h_a,h_b,h_m)$.

We derive the intermediate systems 
$A^t\cdot s = b^t$ by linear interpolation between the initial data
and the target data:
%We denote by $A^0\cdots s = b^0$
For the ``starting system'', we use
the intercepts
 $y^0=(y_1^0,\ldots,y_m^0)$
 and the slopes
 $h^0=(h_1^0,h_a^0,h_b^0,h_m^0)$
% $s'=(s_1',\ldots,s_m')$
 of the edges in the initial drawing~$G$.
% We denote by $A^1\cdots s = b^1$
In the ``target system'', we use
the specified target intercepts
 $y^1=(y_1,\ldots,y_m)$
 and the slopes
 $h^1=(h_1,h_a,h_b,h_m)$ from the target shape~$\Delta$.
(If $\Delta$ is a triangle, this vector includes $h_b$ as an arbitrarily chosen
additional slope, as described earlier.)

 
We define the intermediate data $y^t$ and $h^t$
by linear interpolation:
%We define the intermediate systems 
%$A^t\cdot s = b^t$ by linear interpolation:
\begin{equation*}
  y^t = (1-t)y^0 + ty^1,
  % A^t = (1-t)A^1 + tA^1,
  \qquad
  h^t = (1-t)h^0 + th^1.
%  b^t = (1-t)b^1 + tb^1.
\end{equation*}
This defines the corresponding intermediate systems 
$A^t\cdot s = b^t$, whose coefficients and right-hand sides depend
linearly on the parameter~$t$.

It is important to note that
the starting system
$A^0\cdot s = b^0$
has at least one solution, % $s^0$, 
namely the slopes
$s^0=(s_1^0,\ldots,s_m^0)$
 of the edges in the initial drawing $G$.
 The proportionality constraints
  \thetag{\ref{eq:proportion}--\ref{eq:proportion2}} were designed
  in this way, as described in \secref{setting}.
 The
 concurrency constraints
 \eqref{eq:slope0} are fulfilled because the initial drawing $G$ is a
 straight-line drawing.
 The boundary constraints~\eqref{eq:boundary} are fulfilled by
 construction.



We will show that \lemref{weak-to-strong} can be applied to the
system $A^t\cdot s=b^t$, for every $0\le t\le 1$. We define
an appropriate threshold value $  \epsilon^*$ by
\begin{align*}
  \epsilon^*&
              =
              \min_{e_i\prec e_j}  
              \min_{0\le t\le 1}
              |y_j^t-y_i^t| %: e_i\prec e_j\,\}  
 \\            &
              =%\min_{0\le t\le 1} %\bigl\{\,
              \min_{e_i\prec e_j}  
              \min\{|y_j^0-y_i^0|,|y_j^1-y_i^1|\}%: e_i\prec e_j\,
              % \bigr\}
              > 0
\end{align*}





%  Note that $A$
% and $b$ are functions of $y=(y_1,\ldots,y_m)$ and of the four slopes
% $h=(h_1,h_a,h_b,h_m)$. We make this explicit, by writing $A_1=A(y,h)$
% and $b_1=b(y,h)$. 
% Let $y'=(y_1',\ldots,y_m')$ and $s'=(h_1',\ldots,h_m')$ denote the
% $y$-intercepts and the slopes of the edges in the initial drawing of $G$
% and let $h'=(h_1',h_a',h_b',h_m')$. 

% Consider the system $A(y',h')\cdot s = b(y',h')$.  This system has
% at least one solution $s=s'$.  We now define
% a continuous family of linear systems that interpolates between $A(y',h')\cdot s=b(y',h')$ and $A(y,h)\cdot s=b(y,h)$. Suppose first that the outer face of $G$ is delimited by a quadrilateral.

% For $0\le t\le 1$ and $i\in\{1,a,b,m\}$,  define $h_i(t)=(1-t)h_i' + th_i$ and $h(t)=(h_1(t),h_a(t),h_b(t),h_m(t))$.
% Note that
% \[  
% h_1(t)-h_a(t) = (1-t)(h_1'-h_a') + t(h_1-h_a) > 0. 
% \]
% Inequalities $h_1'-h_a'>0$ and $h_1-h_a>0$ come from the assumption that the vertex incident to  $e_1$ and $e_m$ is in $L$ both in $G$ and in $\Delta$. Similarly, we have $h_m(t)-h_1(t)>0$ and $h_b(t)-h_m(t)>0$. Then we can define \[
% \epsilon_1 = \min_{0\le t\le 1}\min\{h_1(t)-h_a(t), h_m(t)-h_1(t), h_b(t)-h_m(t)\}
% \]
% and observe that $\epsilon_1>0$.

% Analogously, for every $0\le t\le 1$ and for each $i\in\{1,\ldots,m\}$, define $y_i(t) = (1-t)y_i' + ty_i$ and define $y(t)=(y_1(t),\ldots,y_m(t))$.
% Observe that, for any
% $1\le i< j\le m$ and any $0\le t\le 1$,
% \[
% y_j(t) - y_i(t) = (1-t)(y'_j-y'_i) + t(y_j-y_i) > 0.
% \]
% Let 
% \[    \epsilon_2=\min_{0\le t\le 1}\min\{|y_j(t)-y_i(t)|: e_i\prec e_j\}
% \]
% and observe that $\epsilon_2 >0$.  

% The entries in $A_t$ and $b^t$ are obtained in the same way as the entries of $A$ and $b$ were derived earlier, however each entry is now a linear function of~$t$, as $y$ and $h$ are replaced by $y(t)$ and $h(t)$, respectively, in the determination of the equations represented by $A_t$ and $b^t$.
% Consider the unique quadrilateral $\Delta(t)$ whose edges cross $Y$ at
% $y_1(t)$, $y_a(t)$, $y_b(t)$, $y_m(t)$ and have slopes $h_1(t)$,
% $h_a(t)$, $h_b(t)$, and $h_m(t)$, respectively. 

% If the outer face of $G$ is delimited by a triangle, the arguments are analogous, however the inequalities $h_1(t)-h_a(t)>0$ and $h_m(t)-h_1(t)>0$ become either $h_m(t)-h_1(t)>0$ and $h_a(t)-h_m(t)>0$ or $h_a(t)-h_1(t)>0$ and $h_1(t)-h_m(t)>0$, depending on whether $(0,y_1)$ or $(0,y_m)$ is a vertex of $G$, respectively. Further, the inequality involving $h_b(t)$ now states  either $h_b(t)-h_a(t)>0$ or that $h_b(t)$ is smaller than the smallest between $h_1(t)$ and $h_m(t)$, depending on which of the two holds in $G$ (as when determining the boundary equations).

% Next we show that, for every $0\le t\le 1$, \lemref{weak-to-strong} holds for the system $A^t\cdot s=b^t$.

\begin{lem}
   For every $0\le t\le 1$, a solution $s$ to $A^t\cdot s = b^t$
   that satisfies the ordering constraints also satisfies the
   $\epsilon^*$-strong ordering constraints. 
\end{lem}

\begin{proof}
We denote by $\Delta^t$ the shape of the outer face as specified by
$y^t$ and $h^t$.  
It suffices to prove that this shape
is contained in
$[-1,1]\times(-\infty,+\infty)$,
at which point \lemref{weak-to-strong} applies.

We show that each vertex of $\Delta^t$ is in $[-1,1]\times[-\infty,+\infty]$.
If such a vertex $v$ does not lie on $Y$ and is incident to
the two outer edges
$e_i$ and $e_j$, with $i,j\in \{1,a,b,m\}$ and $e_i \prec e_j$, it
has $x$-coordinate $( y_i^t - y_j^t ) / ( s_j^t - 
s_i^t )$.
%$e_i$ and $e_j$ are the two outer edges 
%incident to $v$.
Consider the case that $v\in R$. So we want to show that
\begin{equation}
( y_i^t - y_j^t ) / ( s_j^t - s_i^t )  \le 1 \enspace . \eqlabel{pff}
\end{equation}
By the ordering constraints, $s_j^t - s_i^t > 0$, so \eqref{eq:pff} is
equivalent to
\begin{equation*}
  y_i^t - y_j^t  \le  s_j^t - s_i^t.
%  \\
%y_i^t - y_j^t  + s_i^t  -  s_j^t  \le 0\\
%y'_i + t(y_i-y'_i) - y'_j - t(y_j-y'_j) + s'_i + t(s_i-s'_i) - s'_j - 
%t(s_j-s'_j) \le 0\\
\end{equation*}
This inequality holds for $t=0$ and for $t=1$. The left side is linear
in $t$.
Since $e_i$ and $e_j$ are boundary edges, the right side is also
linear in $t$.
So the inequality holds for 
every $t\in [0,1]$.
In the case $v\in L$, the proof that $v$'s $x$-coordinate is at 
least $-1$ is similar.
\end{proof}


\subsection{Existence (and uniqueness) of solutions to $A^t\cdot s=b^t$}

We now prove the following lemma which, together with \lemref{order-gives-drawing}, completes the proof of \thmref{a-graph}.

\begin{lem}\lemlabel{uniqueness}
	For every $0\le t\le 1$, the system $A^t\cdot s=b^t$ has a
	unique solution $s^t$, and this solution satisfies the ordering
	constraints.
\end{lem}

\begin{proof}
	Since $A^t$ is an $m\times m$ matrix, the system $A^t\cdot
	s=b^t$ has a unique solution~$s^t$ if and only if $\det A^t \neq 0$.
	When $\det A^t =0$, the system may have no solutions or
	multiple solutions.  
	When $\det A^t\neq 0$, 
	Cramer's Rule states that
	the solution
	is $s^t=(s_1^t,\ldots,s_m^t)$ where, for each
	$i\in\{1,\ldots,m\}$,
	\[ 
	s_i^t = \frac{\det A^t_i}{\det A^t }
	\]
	and $A^t_i$ denotes the matrix $A^t$ with its $i$-th column replaced
	by $b^t$. 
	The numerators $\det A^t_i$ and the common
	denominator $\det A^t $ are polynomials in $t$, and therefore
	continuous
	functions of $t$.
	The solution $s^t=(s_1^t,\ldots,s_m^t)$ depends continuously on $t$
	as long as  $\det A^t\ne0 $.
	
	We have already established that $A^0\cdot s=b^0$ has a
	solution $s^0$ that satisfies the ordering constraints. By
	\lemref{unique}, this solution is unique, so $\det A^0\neq 0$.
	
	Let $t^*$ be the smallest $t>0$
	%, if it exists, 
	for which 
	$\det A^{t}= 0$. If such a value does not exist we set $t^*=2$.
	% or $t>1$, we are
	% done.
	
	First we argue that, for all $0\le t <\min \{1,t^*\}$, the unique solution $s^t$ to $A^t\cdot s=b^t$ satisfies the ordering constraints. This argument is similar to the proof of \lemref{unique}. Suppose, for a contradiction, that there is a value $0<t<\min\{1,t^*\}$ for which $s^t$ does not satisfy the ordering constraints. As $t$ increases its value from $0$ to $\min\{1,t^*\}$, since $s^t$ depends continuously on $t$, a value is reached in which $s^t$ violates the $\epsilon^*$-strong ordering constraints, while it does not violate the ordering constraints. However, this contradicts	\lemref{weak-to-strong}.
	
	If $t^*>1$ the same argument also extends to $t=1$ and we are done.
	Let us therefore assume that $0<t^*\le 1$ and derive a contradiction.
	We look at the one-sided limit $s^*=\lim_{t\uparrow t^*}
	s^t$
as $t$ approaches $t^*$ from below.
	Each function $s_i^t$ is a quotient of two polynomials.
	Thus, for $t\to t^*$ it can either converge to $s_i^{t^*}$, or diverge to $+\infty$ or $-\infty$.
	%
	For $t<t^*$ all solutions $s^t$ to the systems $A^t\cdot s=b^t$ satisfy the $\epsilon^*$-strong ordering constraints.
	Hence, if the limit exists, by continuity, it also satisfies $A^{t^*}\cdot s^*=b^{t^*}$
	and the $\epsilon^*$-strong ordering constraints.
	By \lemref{unique}, the solution $s^*$ is
	the unique solution
	of $A^{t^*}\cdot s=b^{t^*}$, but this contradicts the assumption
	that $\det A^{t^*}= 0$.
	
	It remains to rule out the possibility that
	$A^{t^*}\cdot s=b^{t^*}$ has no solution because
	$\lim_{t\uparrow t^*} s^t$ does not exist.  Define the set $H=\{\,e_i\in
	\{e_1,\ldots,e_m\}:\text{$\lim_{t\uparrow t^*} s_i^t$ exists}\,\}$.
	The set $H$ corresponds to the edges of $G$
	with bounded slope; the remaining edges become vertical as $t\to t^*$.
	%
	\lemref{partition-extended} below shows that $H$ contains all edges of $G$. Hence $\lim_{t\uparrow t^*} s^t$ exists. This completes the proof of the lemma.
\end{proof}

%Condition 1 in the following lemma is more general that what we need,
%because it allows us to proceed by induction.

It remains to prove that the set $H$ defined in the proof of \lemref{uniqueness} contains all edges in $E(G)$. We start by stating some properties of $H$.

	\begin{prop}\proplabel{set-H}
		The set   $H$ has the following properties: 
		\begin{compactenum}[(PR1)]
			\item $H$ contains every edge incident to a vertex on the outer face of $G$.
			\item \label{off-C}
			If a vertex $v\not\in Y$ has two incident edges in
			$H$,
			then all $v$'s incident edges belong to $H$.
			\item \label{on-C}
			If a vertex $v\in Y$ has two incident edges $vx,vy\in H$ with $x,y\in L$ or $x,y\in R$, then all $v$'s incident edges belong to $H$.
			\item If $e_i \prec e_j \prec e_k$ and $e_i,e_k\in H$, 
			then $e_j\in H$.
		\end{compactenum}
	\end{prop}
		
	\begin{proof}
 (PR1) If $v$ is a boundary vertex with $v\not\in Y$, then
the location of $v$ is fixed and the $y$-intercepts and therefore slopes of
$v$'s incident edges are fixed.  If $v$ is a boundary vertex with $v\in Y$ 
then $\Delta$ is a triangle and $v$ has three incident edges with slopes 
fixed by the boundary equations. Two of these edges are boundary edges, 
so two of these edges lie on the same side, say $L$, and the third edge 
lies on the other side, say $R$. By the proportionality constraints \eqref{eq:proportion}
all edges in $L$ are bounded, and thus belong to $H$. By the 
proportionality constraints \eqref{eq:proportion2} the range of slopes used by the edges in 
$R$ is bounded, and as one of them is fixed all of them have bounded 
slopes, and thus belong to $H$.

		(PR2)	If $v$ does not lie on $Y$ and two incident edges have bounded slope, then the location of $v$ is fixed in the limit.
		By the concurrency constraints, the slopes of the remaining incident
edges are also bounded.

(PR3) The case where $v$ lies on the outer face is subsumed by (PR1).
% By (PR1) we consider the case where $v$ lies on $Y$ and on the outer
% face of $G$.  In this case, all of $v$'s incident edges have fixed
% slopes and are therefore in $H$.
Assume therefore that $v$ lies on $Y$ and is an interior vertex of $G$.  Define the edges $a_1,\ldots,a_k$
		and $b_1,\ldots,b_\ell$ incident to $v$ as in \figref{ab}.  Let $e$ be the third edge of the triangle with edges $a_1$ and $b_1$, and let
		$f$ be the third edge of the triangle with edges $a_k$ and $b_\ell$.
		Assume without loss of generality that two of the edges $a_i$ belong to $H$. Then, by the proportionality constraints, all edges $a_i$ belong to $H$, and moreover the range $s_{b_\ell}^t-s_{b_1}^t$ converges to a bounded limit as $t\to t^*$.  It follows that either all slopes of the edges $b_j$ are bounded, or they all diverge to $+\infty$,
		or they all diverge to $-\infty$. The ordering constraints for the
		endpoints of $e$ imply \begin{math}
		s_{b_1}<s_e<s_{a_1}
		\end{math}.
		This is inconsistent with $\lim_{t\uparrow t^*} s_{b_1}^t=+\infty$.
		The ordering constraints for the endpoints of $f$ imply
		\begin{math}
		s_{a_k}<s_f<s_{b_\ell}
		\end{math}.
		This is inconsistent with $\lim_{t\uparrow t^*} s_{b_\ell}^t=-\infty$. Thus, the only
		possibility is that all slopes of the edges incident to $v$ are bounded.
		
%		We now show that $H$ satisfies Property~(PR1).  If $v$ is a boundary
%		vertex with $v\in Y$ then, as discussed above, all of $v$'s incident edges
%		have their slopes fixed by the boundary equations and proportionality
%		constraints.  If $v\not\in Y$, then the location of $v$ is fixed by the
%		boundary equations and therefore the slopes of $v$'s incident edges
%		are fixed by the requirement that each edge $e_i$ incident to $v$ also
%		contains $(0,y_i)$.
		
(PR4) This follows from the ordering
constraints, since, for all $0\le t< t^*$,
		$s_i^t<s_j^t<s_k^t$ and both $\lim_{t\uparrow
		t^*}s_i^t$ and $\lim_{t\uparrow t^*}s_k^t$ are defined.
	\end{proof}
	
We now present \lemref{partition-extended}, which completes the proof of \lemref{uniqueness} and \thmref{a-graph}. The lemma is proved by induction on something
that starts as an $A$-graph but is then dismantled into something
more general.  A \emph{near-A-graph} is a graph that satisfies all
conditions of an A-graph except that its outer face can be arbitrarily
complex, even disconnected.  More specifically, each edge of a near-A-graph intersects $Y$
in exactly one point; each inner face is a triangle or a quadrilateral,
without any disconnected compoments inside;
each triangular face contains one vertex in each of $Y$, $L$, and $R$;
and for every vertex $v$ on $Y$ each of the faces directly above and
below $v$ is either a triangular face or the outer face.

\begin{lem}\lemlabel{partition-extended}
Let $G$ be a near-A-graph and let $H \subseteq E(G)$ be a set of edges satisfying Properties (PR1)--(PR4) of \propref{set-H}. 
%	\begin{compactenum}
%		\item if $v$ is a vertex on the outer face 
%		of $G$, then edges incident to $v$ belong to $H$;
%		\item
%		if a vertex $v$ does not lie on $Y$ and has two incident edges in
%		$H$,
%		then all its incident edges belong to $H$;
%		\item
%		if a vertex $v$ lies on $Y$ and has two incident edges in
%		$Y\cup L$ or two incident edges in $Y\cup R$
%		then all $v$'s incident edges belong to $H$.
%		\item if $i \prec j \prec k$ and $i,k\in H$, 
%		then $j\in H$.
%	\end{compactenum}
	Then $H=E(G)$.
\end{lem}

\begin{proof}
	The proof is by induction primarily on the number of inner faces of $G$ and secondarily on the number of vertices of $G$. We dismantle $G$
	from outside while maintaining
	Properties~(PR1)--(PR4). In particular:
	\begin{itemize}
		\item If $G$ is not 2-connected but has more than one
                  edge, we will cut it
		into pieces with fewer edges.
		\item If $G$ is 2-connected, we will modify it and reduce it to a
		graph
		with fewer interior faces,
		keeping the number of edges fixed.
	\end{itemize}
Eventually, we will reduce to a graph with a single edge, and here the
claim is trivial because the edge belongs to the boundary.

We will now go into the details of the proof.
	We refer to the edges of $H$ simply as \emph{$H$-edges}.
	% (horizontal edge) if it is
	%   in $B$ and a \emph{v-edge} (vertical edge) otherwise.  Thus, we wish
	%   to show that all edges of $G$ are h-edges. 
%	The edges incident to the outer
%	face of $G$ are called {\em boundary edges}.
	
	If $G$ is not connected then we can independently apply induction on each component
	of $G$. 
	
	If $G$ has a cut vertex $v$ whose removal
	splits $G$ into components $A_1,\ldots,A_r$ then, for each
	$i\in\{1,\ldots,r\}$, we can independently apply induction on the subgraph $G_i$ of $G$
	induced by $V(A_i)\cup\{v\}$. Every edge of $G_i$ inherits its
        classification as an $H$-edge from its corresponding edge in
        $G$.
        Then it is easy to see that Properties~(PR1)--(PR4) are
        satisfied by $G_i$.
        Properties~(PR2)--(PR4) are obviously preserved under taking subgraphs. 
        Property~(PR1) follows from the fact that every boundary vertex of $G_i$ is also a boundary vertex of $G$; this is because each inner face of $G$ is a quadrangle or a triangle, hence $G_i$ cannot be nested inside a different subgraph $G_j$ of $G$.

	\begin{figure*}[htb]
		\centering{\includegraphics{figs/open-a-triangle}}
		\caption{Proof of \lemref{partition-extended} with a vertex
			$v$ on $Y$. Integrating a triangle into the outer face.}
		\figlabel{lemma-y-3}
	\end{figure*}	
        

	We are left with the case in which $G$ is a 2-connected near-A-graph whose outer face 
	is delimited by a simple cycle $F$. We distinguish two cases.
	
	{\em Case 1}. The cycle $F$ contains a vertex $v$ on $Y$ that is incident to an inner triangular face $vab$. In this case
        we \emph{open up} $vab$, merging it into the outer face. \figref{lemma-y-3} illustrates the procedure for the case that $ab$ lies below $v$, with $a\in L$ and $b\in R$. Let $u$ and $w$ be the predecessor and successor of $v$ on the
	counterclockwise cycle $F$, and assume w.l.o.g.\ that $u\in R$.
	We construct a new graph $G'$ by splitting $v$ into two vertices $x$
	and $y$ that both lie on $Y$, with $y$ above $x$. We make $x$ adjacent to $u$ and to every neighbor of $v$ between $b$ and $u$.
	We make $y$ adjacent to the remaining neighbors of $v$.
	\figref{lemma-y-3} shows that this procedure works both for $w\in R$
	and for $w\in L$. Note that $G'$ has one inner face less than $G$, hence induction applies.
		
	
	{\em Case 2}. If Case~1 does not hold,
%	
%	Some edge $e$ of $F$ has one endpoint in $L$ and one endpoint in $R$.  In this case, there is at least one more edge $e'\neq e$ with one endpoint in each of $L$ and $R$.  The edges $e$ and $e'$ have at most one vertex in common and $F$ is not a triangle consisting, otherwise its third edge $e''$ would be completely contained in $R$ or in $L$.
%	Therefore $F$ contains at least four vertices. 
%	
%	In this case, we claim that
        every vertex of $F$ on $Y$ has all its neighbours
	in $L$ or has all its neighbours in $R$.  Since the same is already true for every vertex not on $Y$, a traversal of $F$ has to zigzag/alternate between edges that move to the left and edges that move to the right.  Thus, $F$ must contain some reflex vertex $v$.  The vertex $v$ cannot lie on $Y$, because otherwise we would be in Case~1.  Let $uv$ and $vw$ be the two consecutive edges of $F$ incident on $v$ and let $p$ and $q$ be the intersections of $uv$ and $vw$ with $Y$, with $q$ above $p$.	
	(Note that $u$ and/or $w$ may be contained in $Y$.)
	This implies that $v$ is a reflex vertex of some inner face
	$q=vabc$ of $G$.  Indeed, $vc$ is the first edge incident
	to $v$ intersected by $Y$ and $va$ is the last edge incident
	to $v$ intersected by $Y$. (Note the possibility that $a=w$
	and/or $c=u$.)	We construct a new graph $G'$ by splitting
	$v$ into two vertices $x$ and $y$. We make the vertex $x$
	adjacent to $u$ and every neighbour $z$ of $v$ such that $Y$
	intersects $vz$ before $vu$.  We make $y$ adjacent to all of
	$v$'s neighbors that are not adjacent to $x$.  \Figref{lemma-y-4}
	illustrates this procedure; the lower half illustrates the case
	where $w\in Y$. In $G'$, $q$ is part of the outer face, so $G'$
	has one less inner face than $G$, hence induction applies.

	\begin{figure*}[ht]
          \centering\includegraphics{figs/lemma-y2}
            %\\ \includegraphics{figs/lemma-y-on-y}
          \caption{Proof of \lemref{partition-extended} for a reflex
            vertex $v$. Integrating a quadrilateral into the outer face.}
		\figlabel{lemma-y-4}
	\end{figure*}
	
	
	
	This finishes the description of how we modify $G$ into $G'$. Every edge of $G'$ inherits its classification as an $H$-edge from its corresponding edge in $G$. We have to show that $G'$ satisfies Properties~(PR1)--(PR4). Actually Property~(PR1) is the only property that needs to be discussed, as the other properties follow trivially from the fact that $G$ satisfies them. 
%	Indeed,  $G'$ some of the $\prec$-relations involving edges incident to $v$ are missing, but no new ones are introduced, so $G'$ still
%	satisfies Condition~4.
%	The same argument applies to
%	Conditions~2 and~3. Some adjacent edges in $G$ might no longer be adjacent
%	in $G'$, but this makes Condition 2 and 3 only weaker.
%	
%	
	
	First, note that all edges incident to the new vertices $x$ or $y$
	were incident to $v$ before, and thus they are $H$-edges. Second,
	all edges incident to any boundary vertex of $G'$ that is also
	a boundary vertex of $G$ are $H$-edges, since the edges of $G'$
	inherit their classification as $H$-edges from their corresponding
	edges in $G$. It remains to deal with the boundary vertices of
	$G'$ that are inner vertices of $G$.

	In Case 1 we have two boundary vertices of $G'$ that might be
	inner vertices of $G$, namely $a$ and $b$. These vertices
	do not lie on~$Y$. By Property~(PR1) for $G$, both $va$ and
	$vb$ are $H$-edges, since they are incident to the boundary
	vertex $v$. From the ordering constraints around $a$ and $b$
	we get $va\prec ab\prec vb$ or $vb\prec ab\prec va$, and thus,
	by Property~(PR4) for $G$, we have $ab\in H$.  Now we have two
	$H$-edges $va$ and $ab$ incident to $a$, and by Property~(PR2)
	for $G$ all edges incident to $a$ belong to $H$. It follows that
	all edges incident to $a$ in $G'$ are $H$-edges, and similarly
	for~$b$.

	In Case~2 we have three boundary vertices of $G'$ that might be inner
	vertices of $G$, namely $a$, $b$, and $c$. By Property~(PR1) for
	$G$, both $va$ and $vc$ are $H$-edges, since they are incident
	to the boundary vertex $v$. Consider the quadrilateral $q=vabc$
	of $G$. By the ordering constraints, we get $vc \prec bc\prec
	ba\prec va$ or $va \prec ba\prec bc\prec vc$, depending on
	whether $v\in L$ or $v\in R$. Thus, by Property~(PR4) for $G$,
	the edges $bc$ and $ba$ are also $H$-edges. The vertex $b$ does
	not lie on $Y$. The vertex $a$ might lie on $Y$ or not, but if it
	does, then the two incident edges $va$ and $ab$ lie in the same
	half-plane.  The same holds for $c$. Thus by Properties~(PR2)
	or~(PR3) for $G$ all edges incident to $a$, $b$ and $c$ in $G$
	belong to $H$. It follows that all edges incident to $a$, $b$,
	and $c$ in $G'$ are $H$-edges.
	
	
	
	%   By Conditions~1 and 2, all edges incident to $v$ are $H$-edges and $v$
	%   is a reflex vertex of $q$. Therefore all edges of $q$ are $H$-edges.
	Since $G'$ satisfies Properties~(PR1)--(PR4) induction applies and all edges of $G'$ (and thus all edges of $G$) are $H$-edges. This completes the proof.
\end{proof}

%%% Local Variables:
%%% mode: latex
%%% TeX-master: "freecoll-SODA"
%%% End:
