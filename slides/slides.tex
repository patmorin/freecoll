\documentclass[xcolor=dvipsnames]{beamer}
\usepackage{ods}
%\usepackage{ods-figs}
\usepackage[cm]{sfmath}
\usepackage[utf8]{inputenc}
%\usepackage{enumitem}
%\usepackage{enumitem}
%\setitemize{itemsep=1.5ex}
%\setlength{\leftmargini}{0pt}
\usepackage{array}

\newcommand{\Fary}{F\'ary}

\title{Every Collinear Set is Free}
\author{Vida Dujmović \and Fabrizio Frati \and Daniel Gonçalves
        \and Pat Morin \and Günter Rote}
\titlegraphic{\includegraphics[height=1em]{by}}

\begin{document}

\begin{frame}
  \titlepage
\end{frame}

\begin{frame}
  \frametitle{Collinear Sets}

  \centering{\multiinclude[<+>][format=pdf,start=1,graphics={height=4cm}]{figs/cs}}
  \begin{itemize}
    \item<2-> $S\subset V(G)$ is a \emph{collinear set} if there exists a \Fary\ embedding of $G$ in which $S$ is contained in a line.
    \item<3-> $S$ is \emph{free} if, for every $y_1<y_2<\cdots<y_{|S|}$, there exists a \Fary\ embedding of $S$ that places $S$ at $(0,y_1),\ldots,(0,y_{|S|})$.
  \end{itemize}
\end{frame}

\begin{frame}
  \frametitle{New Result}

  \begin{itemize}
     \item<+-> Ravsky and Verbitsky (2008) pose two problems:
       \begin{itemize}
          \item<+-> Give bounds on the size of the largest collinear set in an $n$-vertex planar graph.
          \item<+-> Give bounds on the size of the largest free collinear set in an $n$-vertex planar graph.
       \end{itemize}
   \item<+->
   ``It seems that \emph{a priori} we even cannot exclude equality. To clarify
   this question, it would be helpful to (dis)prove that every collinear
   set in any straight line drawing is free.''
   \item<+-> \textbf{Theorem 1:} Every collinear set is free.
  \end{itemize}
\end{frame}


\begin{frame}
   \frametitle{Known Results}
   \begin{itemize}
      \item<+-> Ravsky-Verbitsky-2008: There are $n$-vertex planar graphs whose
      largest collinear set is of size $O(n^{0.986})$. 
      \item<+-> Uses Gr\"unbaum-Walther-1973 (3-regular planar graphs without long cycles)
      \centerline{\includegraphics[height=4cm]{figs/dual}}
    \end{itemize}
\end{frame}


\begin{frame}
   \frametitle{Known Results}
   \begin{itemize}
      \item<+-> Bose-Dujmovi\'c-Hurtado-Langerman-M-Wood-2009: Every
      $n$-vertex planar graph has a free collinear set of size
      $\Omega(\sqrt{n})$.
      \item<+-> Proof by canonical ordering or by Schnyder decomposition
   \end{itemize}
\end{frame}


\begin{frame}
   \frametitle{Known Results}
   \begin{itemize}
      \item<+->Da Lozzo-Dujmovi\'c-Frati-Mchedlidze-Roselli-2018:
      A set $S\subseteq V(G)$ is a collinear set if and only if there is a ``nice''
      curve that contains $V$.\\


  \centering{\multiinclude[<+>][format=pdf,start=1,end=2,graphics={height=4cm}]{figs/nice-curve}}

   \end{itemize}
\end{frame}

\begin{frame}
   \frametitle{Applications (of \emph{free} collinear sets)}

   \begin{itemize}
      \item<+-> Dujmovic-2015: Free collinear sets have many applications, including untangling, column planarity, universal point subsets, and partial simultaneous geometric embeddings.
      \item<+->\textbf{Lemma F:} For any point set $P$ with $|P|=|S|$, $G$ has a \Fary\ embedding where $S\to P$.
      \begin{itemize}
        \item<+-> $P=\{(x_1,y_1),\ldots,(x_{|S|},y_{|S|}))\}$, and assume $y_i\neq y_j$, for all $i\neq j$.
        \item<+-> $G$ has a \Fary\ embedding where $S\to \{(0,y_1),\ldots,(0,y_{|S|}))\}$
        \item<+-> For small enough $\epsilon >0$,  $G$ has a \Fary\ embedding where
          $S\to \{(\epsilon x_1,y_1),\ldots,(\epsilon x_{|S|},y_{|S|})\}$
        \item<+-> Divide all x-coordinates by $\epsilon$ to get an embedding where $S\to P$. \hfill{QED} 
      \end{itemize}
   \end{itemize}
\end{frame}
 
\begin{frame}
   \frametitle{So Far}

   \begin{itemize}
       \item<+-> DDFMR2018: Collinear sets are easy to find (just find a nice curve)
       \item<+-> D2015: Free collinear sets useful (can be embedded on any point set)
       \item<+-> Here: Every collinear set is free
       \item<+-> Therefore, free collinear sets are easy to find and useful
   \end{itemize}
\end{frame}

\begin{frame}
   \frametitle{The Proof}
   \begin{itemize}[<+->]
     \item The hard part is edges that cross the y-axis
     \begin{tabular}{m{4cm}m{1cm}m{4cm}}
      \includegraphics[height=3cm]{figs/easy-hard-1}& versus& \includegraphics[height=3cm]{figs/easy-hard-2}
     \end{tabular}
     \item So first study how to make edges cross where we need them to
     \item The hardest thing would be an edge maximal planar graph in which
       every edge crosses the y-axis
     \item bipartite \& edge maximal $\Rightarrow$ quadrangulation
   \end{itemize}

\end{frame}


\begin{frame}
   \frametitle{Quadrangulations}
   \begin{itemize}
     \item \textbf{Theorem Q:} Let 
        \begin{itemize}
          \item $Q$ be a quadrangulation whose $m$ edges all cross the y-axis in the order $e_1,\ldots,e_m$ and
          \item $y_1<\cdots<y_m$ be any increasing sequence of numbers.
        \end{itemize}
        Then
        \begin{itemize}
           \item $Q$ has a \Fary\ embedding in which each $e_i$ crosses the y-axis at $y_i$ and
           \item we can specify the outer face in the embedding (consistent with $y_1,y_a,y_b,y_m$).
        \end{itemize}
        \centerline{\includegraphics[height=3.8cm]{figs/quad}}
   \end{itemize}
\end{frame}

\begin{frame}
   \frametitle{Proof of Theorem Q}
   \begin{itemize}[<+->]
     \item Euler's Formula: An $n$-vertex quadrangulation has $m=2n-4$ edges.
     \item Let $s_i$ be the slope of $e_i$ in the desired embedding, so $e_i$ is part of the line $\ell_i = \{(x,y):y=x s_i + y_i\}$
     \item If $e_i$, $e_j$, and $e_k$ are all adjacent to $v$, then
     \[\det\left[\begin{array}{ccc}1&1&1\\s_i&s_j&s_k\\y_i&y_j&y_k\end{array}\right] = (y_j-y_k)s_i + (y_k-y_i)s_j + (y_i-y_j)s_k = 0 \]
     this is a linear equation in $s_1,\ldots,s_m$.
     \item Pick 2 edges $e_i$ and $e_j$ incident to each vertex and use this equation on each additional vertex $e_k$ adjacent to $v$ 
     \item This gives $\sum_{v\in V(G)}(d_v-2) = 2n-8 = m - 4$ linear equations
     \item But the four slopes $s_1,s_a,s_b,s_m$ are fixed by the outer face, so we have $m$ equations
   \end{itemize}
\end{frame}
 

\begin{frame}
   \frametitle{Proof of Theorem Q}
   \[
       \begin{array}{c}
       \end{array}
       \left[\begin{array}{ccccccc}
       1 & \cdots & 0 & \cdots&  0 & \cdots & 0 \\
       0 & \cdots & 1 & \cdots&  0 & \cdots & 0  \\
       0 & \cdots & 0 & \cdots&  1 & \cdots & 0  \\
       0 & \cdots & 0 & \cdots&  0 & \cdots & 1  \\
       \vdots & \ddots & \vdots & \ddots & \vdots & \ddots & \vdots  \\
       \vdots & \ddots & \vdots & \ddots & \vdots & \ddots & \vdots  \\
       \vdots & \ddots & \vdots & \ddots & \vdots & \ddots & \vdots  \\
       \end{array}\right]
       \cdot
       [s_1,\ldots,s_m] =
       \left[\begin{array}{c}
        \bar{s}_1\\
        \bar{s}_a\\
        \bar{s}_b\\
        \bar{s}_m\\
        0\\ 
        \vdots \\
        0
      \end{array}\right]
   \]
   \begin{itemize}[<+->]
     \item Looks promising, but does it have a solution, and does the solution
       give a \Fary\ embedding?
     \item Yes (but the proof is too long for today)
     \item Main idea: Use continuity and the fact that there exists
      $y_1'<\cdots<y_m'$ for which the resulting system has a solution
   \end{itemize}
\end{frame}



\begin{frame}
   \frametitle{Theorem Q'}
   \begin{itemize}[<+->]
     \item Theorem Q is nice, but we need \emph{vertices} on the y-axis
     \item Theorem Q has an extension that specifies some edges to contract
       and place \emph{on} the y-axis:
     \item Theorem Q':
       \centerline{
      \begin{tabular}{m{4cm}m{.5cm}m{4cm}}
      \includegraphics[height=3cm]{figs/qprime-1}& $\Rightarrow$ & \includegraphics[height=3cm]{figs/qprime-2}
     \end{tabular}
       }
   \end{itemize}
\end{frame}
 
\begin{frame}
   \frametitle{Main Theorem}
   \begin{itemize}
      \item \textbf{Theorem M:} Let
      \begin{itemize}
           \item $T$ be a (\Fary\ embedding of a) triangulation; 
           \item $r_1,\ldots,r_k$ be the sequence of edges and vertices
              of $T$ that intersect the y-axis (in the order they intersect);
           \item $y_1<\cdots<y_k$ be any increasing sequence; and
           \item $\Delta$ be a triangle whose intersection with y-axis is $[y_1,y_k]$.$^*$
      \end{itemize}
      \item Then, for any $\epsilon>0$, $T$ has a \Fary\ embedding in which 
      \begin{itemize}
          \item $r_i$ is at $(0,y_i)$ if $r_i$ is a vertex;
          \item $r_i$ intersects the y-axis at $[y_i-\epsilon,y_i+\epsilon]$ if $r_i$ is an edge;
          \item The outer face of $T$ is $\Delta$
      \end{itemize}
   \end{itemize}
   \centerline{\includegraphics[height=3cm]{figs/main}}
\end{frame} 


\begin{frame}
   \frametitle{Main Theorem}
   \resizebox{!}{1cm}{
   \begin{minipage}{\textwidth}
   \begin{itemize}
      \item \textbf{Theorem M:} Let
      \begin{itemize}
           \item $T$ be a (\Fary\ embedding of a) triangulation; 
           \item $r_1,\ldots,r_k$ be the sequence of edges and vertices
              of $T$ that intersect the y-axis (in the order they intersect);
           \item $y_1<\cdots<y_k$ be any increasing sequence; and
           \item $\Delta$ be a triangle whose intersection with y-axis is $[y_1,y_k]$.$^*$
      \end{itemize}
      \item Then, for any $\epsilon>0$, $T$ has a \Fary\ embedding in which 
      \begin{itemize}
          \item $r_i$ is at $(0,y_i)$ if $r_i$ is a vertex;
          \item $r_i$ intersects the y-axis at $[y_i-\epsilon,y_i+\epsilon]$ if $r_i$ is an edge;
          \item The outer face of $T$ is $\Delta$
      \end{itemize}
   \end{itemize}
   \end{minipage}
   }
   \centerline{\includegraphics[height=5cm]{figs/main}}
\end{frame} 

\begin{frame}
   \frametitle{Proof of Theorem M}
   \begin{itemize}[<+->]
   \item Induction on number of vertices + non-crossing edges
   \item Use several types of reductions:
     \begin{itemize}[<+->]
        \item Induction on separating triangles
        \item Contracting non-crossing edges
        \item Flipping non-crossing edges
        \item Flipping edges in the y-axis
      \end{itemize}
   \item Base case: No reduction applies, but graph is simple enough to 
        apply Theorem Q'
   \end{itemize}
\end{frame}

\begin{frame}
   \frametitle{Induction on Separating Triangles}
    \begin{center}
      \includegraphics[width=.9\textwidth]{figs/separating}
    \end{center}
\end{frame}

\begin{frame}
   \frametitle{Edge Contraction}
    \begin{center}
      \includegraphics[width=.9\textwidth]{figs/contractible}
    \end{center}
\end{frame}

\begin{frame}
   \frametitle{Flippable Edges}
    \begin{center}
      \includegraphics[width=.9\textwidth]{figs/flippable}
    \end{center}
\end{frame}

\begin{frame}
   \frametitle{Edges on the y-axis}
    \begin{itemize}
       \item Just flip them
    \end{itemize}
\end{frame}


\begin{frame}
   \frametitle{The Base Case}
    \begin{itemize}[<+->]
       \item No separating triangles, no contractible edges, no flippable edges,
 no edges in the y-axis
       \item \textbf{Claim:} Every non-crossing edge is incident to two crossing triangles
       \item Removing all non-crossing edges creates a bunch of quadrangular faces 
       \item Now \emph{split} all vertices on the y-axis
       \centerline{\includegraphics[height=2cm]{figs/split}} 
    \end{itemize}
\end{frame}


\begin{frame}
   \frametitle{The Base Case}
    \resizebox{!}{1cm}{
    \begin{minipage}{\textwidth}
    \begin{itemize}
       \item No separating triangles, no contractible edges, no flippable edges,
 no edges in the y-axis
       \item \textbf{Claim:} Every non-crossing edge is incident to two crossing triangles
       \item Removing all non-crossing edges creates a bunch of quadrangular faces 
       \item Now \emph{split} all vertices on the y-axis
    \end{itemize}
    \end{minipage}}
    \centerline{\includegraphics[height=4cm]{figs/split}} 
    \begin{itemize}
      \item<2-> We now have a quadrangulation $Q$ and independent set $S$ of \emph{split} edges.
    \end{itemize}
\end{frame}

\begin{frame}
   \frametitle{The Base Case}
    \resizebox{!}{1cm}{
    \begin{minipage}{\textwidth}
    \begin{itemize}
       \item No separating triangles, no contractible edges, no flippable edges,
 no edges in the y-axis
       \item \textbf{Claim:} Every non-crossing edge is incident to two crossing triangles
       \item Removing all non-crossing edges creates a bunch of quadrangular faces 
       \item Now \emph{split} all vertices on the y-axis
    \end{itemize}
    \centerline{\includegraphics[height=2cm]{figs/split}} 
    \end{minipage}}
    \begin{itemize}[<+->]
      \item We now have a quadrangulation $Q$ and independent set $S$ of \emph{split} edges.
      \item Apply Theorem Q' \hfill{QED} 
    \end{itemize}
\end{frame}


\begin{frame}
   \frametitle{The End}

    \centerline{\Huge{Thank You!}}
\end{frame}
 


\end{document}

\begin{frame}
   \frametitle{Proof Sketch}

   \begin{itemize}
       \item<+-> DDFMR2018: Collinear sets are easy to find (just find a nice curve)
       \item<+-> D2015: Free collinear sets useful (can be embedded on any point set)
       \item<+-> Here: Every collinear set is free
       \item<+-> Therefore, free collinear sets are easy to find and useful
   \end{itemize}
\end{frame}


\begin{frame}
   \frametitle{The Proof}
   \begin{itemize}[<+->]
     \item The hard part is edges that cross the y-axis
     \begin{tabular}{m{4cm}m{1cm}m{4cm}}
      \includegraphics[height=3cm]{figs/easy-hard-1}& versus& \includegraphics[height=3cm]{figs/easy-hard-2}
     \end{tabular}
     \item So first study how to make edges cross where we need them to
     \item The hardest thing would be an edge maximal planar graph in which
       every edge crosses the y-axis
     \item bipartite \& edge maximal $\Rightarrow$ quadrangulation
   \end{itemize}

\end{frame}



\end{document}


\begin{frame}
  \frametitle{ArrayDeque}

  \begin{center}
    \multiinclude[<+>][format=pdf,start=1]{figs/arraydeque}
  \end{center}
\end{frame}

\end{document}
\begin{frame}
  \frametitle{ArrayStack}
  \framesubtitle{Implementing a List using an array}
  \begin{block}{Theorem}
    \begin{itemize}
      \item<+->[]An ArrayStack implements the List interface.
      \item<+->[]Ignoring the time spent in calls to $\fn{resize}()$,
      \begin{itemize}
          \item<+->$\fn{get}(i)$ runs in $O(1)$ time;
          \item<+->$\fn{set}(i,x)$ runs in $O(1)$ time;
          \item<+->$\fn{add}(i,x)$ runs in $O(1+n-i)$ time; and
          \item<+->$\fn{remove}(i)$ runs in $O(1+n-i)$ time.
      \end{itemize}
    \end{itemize}
  \end{block}

  \begin{block}<+->{Question}
     What about the calls to $\mathsf{resize}()$?
  \end{block}

\end{frame}


\begin{frame}
   \frametitle{Proof Outline}
   \begin{itemize}
     \item The hard part is crossing edges
      \centerline{\includegraphics[height=2cm]{figs/easy-hard-1}\includegraphics[height=2cm]{figs/easy-hard-2}}
   \end{itemize}

\end{frame}



\closing

\end{document}

