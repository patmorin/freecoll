\documentclass{patmorin}
\usepackage[utf8]{inputenc}
\usepackage{amsthm,amsmath,graphicx}
\usepackage{array}
\usepackage{pat}
\usepackage{hyperref}
\usepackage[dvipsnames]{xcolor}
\definecolor{linkblue}{named}{Blue}
\hypersetup{colorlinks=true, linkcolor=linkblue,  anchorcolor=linkblue,
citecolor=linkblue, filecolor=linkblue, menucolor=linkblue,
urlcolor=linkblue, pdfcreator=Me, pdfproducer=Me} \setlength{\parskip}{1ex}
\usepackage{tikz}


\DeclareMathOperator{\sgn}{sgn}


\listfiles
\newcommand{\lstlabel}[1]{\label{lst:#1}}
\newcommand{\lstref}[1]{Listing~\ref{lst:#1}}
\newcommand{\Lstref}[1]{\lstref{#1}}

\DeclareMathOperator{\block}{block}
\newcommand{\naive}{na\"{\i}ve}


\newcommand{\reals}{\mathbb{R}}
\newcommand{\integers}{\mathbb{Z}}
\newcommand{\naturals}{\mathbb{N}}
\newcommand{\dist}{{d}}

\title{\MakeUppercase{Every Collinear Set is Free}}

\author{Bellairs Workshop on Geometry and Graphs 2017--18}

\begin{document}
\maketitle


\begin{abstract}
  We show that if a planar graph $G$ has a planar straight-line drawing
  in which a subset $S$ of its vertices are collinear, then there is a
  planar straight-line drawing of $G$ in which all vertices in $S$ are
  on the $y$-axis and in which they have prescribed $y$-coordinates.
  This solves an open problem posed by Ravsky and Verbitsky in 2008.
  In their terminology, we show that every collinear set is free.
  This result has applications in graph drawing, untangling, universal
  point subsets, and related areas.
\end{abstract}


\section{Introduction}

In a planar graph, $G=(V,E)$, a \emph{collinear set} is a set of vertices
$S\subset V$ such that $G$ has a planar straight-line drawing in which
all vertices in $S$ are drawn on a single line.  A collinear set $S$
is a \emph{free collinear set} if, for any collinear set of points
$X\subset\R^2$, $|X|=|S|$, $G$ has a planar straight-line drawing in
which the vertices of $S$ are drawn on the points in $X$.  Ravsky and
Verbitsky \cite{ravsky.verbitsky:on} ask the following question:

\begin{quote}
   How far or close are parameters $\tilde{v}(G)$ and $\bar{v}(G)$? It
   seems that \emph{a priori} we even cannot exclude equality. To clarify
   this question, it would be helpful to (dis)prove that every collinear
   set in any straight line drawing is free.
\end{quote}

In the context of this quote, $\tilde{v}(G)$ and $\bar{v}(G)$ are the
respective sizes of the largest collinear set and largest free collinear
set in $G$.  In this note, we prove that $\tilde{v}(G)=\bar{v}(G)$ by
showing that every collinear set is a free collinear set.  

Dujmovi\'c and Frati gave the following characterization of collinear sets:
\begin{thm}
   A set $S\subseteq V$ is a collinear set in a planar graph $G=(V,E)$
   if and only if there exists a (topological) drawing of $G$ and a
   jordan curve $C$ such that each vertex of $S$ is drawn on $C$ and the
   intersection of each edge with $C$ has at most one connected component.
\end{thm}
The surprising aspect of this characterization is that one can
simultaneously straighten the drawing of the graph so that it becomes a
straight line drawing and straighten the Jordan curve so that it becomes
(say) the y-axis while preserving the combinatorial relationship between
$C$ and $G$.

\subsection{Proof Sketch}


Without loss of generality, we may assume that the the graph we are
interested in is a (topological) triangulation $T$ and the line we are
interested in is the y-axis.

%Our proof has three steps:
%\begin{enumerate}
%   \item Using combinatorial operations---edge subdivision, edge
%   contraction, and the removal of separating triangles---and induction
%   on the number of vertices, we reduce to a problem in which $T$
%   has no separating triangles and every edge of $T$ is incident to a
%   triangle that has two edges intersecting the y-axis and $S$ is an
%   independent set that is on the y-axis.
%
%   \item Using local operations---removing edges and vertices and
%   introducting edges---on $T$ we obtain a (topological) quadrangulation
%   $Q$ for which every edge intersects the y-axis and $S$ is on the
%   y-axis.  Furthermore, $Q$ is structured so that, if we take any
%   non-crossing straigh-line drawing of $Q$ we can rei
%
%   \item Thus we have a quadrangulation $Q$ whose edges intersect
%   the y-axis in the order $e_1,\ldots,e_m$ and the vertex set of $Q$
%   contains $S$, still on the y-axis.
%
%   Let $y_1\le\cdots\le y_m$ be any sequence of numbers such that
%   $y_i=y_{i+1}$ if and only if $e_i$ and $e_{i+1}$ have a common endpoint
%   in $S$.  We show that, for any $\epsilon >0$, $Q$ has a planar
%   straight-line drawing such that the intersection of $e_i$ with the
%   y-axis is at $y_i\pm\epsilon$.  Furthermore, if $y_i=y_{i+1}$, then
%   the intersection of $e_i$ and $e_{i+1}$ with the y-axis is exactly
%   at $y_i$. In this way every vertex of $S$ is drawn on the y-axis
%   at precisely the desired location.
%\end{enumerate}


works in three basic steps: 


\section{Definitions}

Jordan curve

Embedded planar graph = each vertex is a distinct point and each edge is a closed curve whose endpoints are its two vertices.  No two edges intersect except possibly at their common endpoint.

Triangulation = $n$-vertex embedded graph in which each face is bounded by a 3-cycle.

separating triangle

edge contraction

Quadrangulation = $n$-vertex embedded graph in which each face is bounded by a 4-cycle. Has $2n-4$ edges.

Drawing = Embedding of a graph with no crossings

Straight-line drawing = drawing (implies non-crossing) in which edges are represented by line segments 


\section{The Proof}

\subsection{Drawing Quadrangulations}

\begin{lem}
    Let $Q$ be a quadrangulation, let $C$ be a Jordan curve that
    intersects each edge of $Q$ in exactly one point and does not contain
    any vertex of $Q$, let $e_1,\ldots,e_m$ the edges of $Q$ in the
    (cyclic) order they are intersected by $C$, and let $y_1<\cdots<y_m$
    be any increasing sequence of numbers.  Then $Q$ has a straight-line
    drawing in which, for each $i\in\{1,\ldots,m\}$, the intersection
    of $e_i$ with the y-axis is a single point $(0,y_i)$.
\end{lem}

\begin{proof}
   G\"unter's proof showing that the resulting system of equations has a unique solution.
\end{proof}

\begin{lem}
   Let $Q$ be a straight-line drawing of a quadrangulation drawn so
   that each of its (closed) edges intersects the y-axis in exactly one
   point. Let $v$ be a vertex of $Q$ that is not on the y-axis, whose
   incident edges cross the y-axis at $y_1'<\cdots<y_d'$ and suppose that
   no other edges of $Q$ cross the y-axis in the interval $[y_1',y_d']$.
   Then, for any $y'\in[y_1',y_d']$, moving $v$ to $(0,y')$ yields a
   straight-line drawing of $Q$.
\end{lem}

\begin{proof}
  Easy geometric argument.
\end{proof}

\subsection{Drawing Collinear Sets}

We begin with a result that is not about vertices, but rather about edges.

\begin{thm}
   Let $T$ be a triangulation and let $C:[0,1]\to\R^2$ be a simple curve
   both of whose endpoints are both contained in the interior of some
   face $f$ of $T$ and whose intersection with each edge of $Q$ consists
   of at most one point.. Let $r_1,\ldots,r_k$ be the sequence of
   vertices and edges of $Q$ that are intersected by $C$, and ordered in
   the order that they are intersected by $C$, and let $y_1<\cdots<y_k$
   be any sequence of numbers.  Then, for any $\epsilon>0$, $T$ has a
   straight-line drawing in which,
   for each $i\in\{1,\ldots,k\}$, 
   \begin{enumerate}
       \item $r_i$ is drawn on the y-axis, with y-coordinate $y_i$,
         if $r_i$ is a vertex; or
       \item (if $r_i$ is an edge) the intersection of $r_i$ with the
         y-axis has a y-coordinate in the interval
         $[y_i-\epsilon,y_i+\epsilon]$.
   \end{enumerate}
\end{thm}

\begin{proof}
   Make a Jordan curve $J$ by joining the endpoints of $C$ with a path
   contained in the interior of $f$.  Let $\bar{J}$ be the closed finite
   subset of the plane whose boundary is $J$. We say that an edge of
   $T$ is a \emph{crossing edge} if it intersects $\bar{J}$ but is not
   contained in $\bar{J}$.  Note that this means that a crossing edge may
   have an endpoint on $J$ if its other endpoint is outside of $\bar{J}$.
   We say that a triangular face of $T$ is a \emph{crossing face} if
   is incident to two crossing edges.  We say that points in $\bar{J}$
   are \emph{to the left of $C$} and other points are \emph{to the right
   of $C$}.

   We will prove something stronger: Let $\alpha$, $\beta$, and $\gamma$
   denote the three vertices of $f$, in counterclockwise order.
   Assume, without loss of generality that $\alpha,\beta\in \bar{J}$
   and $\gamma\not\in\bar{J}$.  Let $\Delta$ be any triangle with two
   vertices $\alpha'$ and $\beta'$ to the left of the y-axis and one
   vertex $\gamma'$ to the right of the y-axis and such that the boundary
   of $\Delta$ intersects the y-axis at contains $(0,y_1)$ and $(0,y_k)$.
   We will show that $T$ has a straight-line drawing that satisfies the
   conditions of the theorem and in which $\alpha$ is drawn at $\alpha'$,
   $\beta$ is drawn at $\beta'$ and $\gamma$ is drawn at $\gamma'$.
   We will prove this by induction on the number of vertices of $T$.

   \paragraph{Separating Triangles:}
   If $T$ contains a separating triangle $xyz$ then we remove the
   component of $T-\{x,y,z\}$ that does not contain any vertex of $f$ to
   obtain a triangulation $T'$ in which $xyz$ is a face.  Observe that,
   since the intersection of $C$ with each of $xy$, $yz$ and $zx$
   consists of at most a single point, the vertices and edges of $T$
   intersected by $C$ that are not in $T'$ appear as a contiguous
   subsequence $r_i,\ldots,r_j$.

   Observe that each of $r_{i-1}$ and $r_{j+1}$ is either an edge
   or vertex of the triangle $xyz$.  Set $\epsilon'$ to be any
   value less than $\min\{\epsilon,y_{i}-y_{i-1}, y_{j+1}-y_j\}$.
   and apply induction on $T'$ using the value $\epsilon'$
   and the sequences $r_1,\ldots,r_{i-1},r_{j+1},\ldots,r_k$ and
   $y_1,\ldots,y_{i-1},y_{j+1},\ldots,y_k$ to obtain a drawing of $T'$.
   In the resulting drawing $xyz$ becomes a triangular face $\Delta'$.

   In the resulting drawing, Let $y_{i-1}'$ and $y_{j+1}'$ be
   the respective y-coordinates of the intersections of $r_{i-1}$
   and $r_{j+1}$ with the y-axis.  By our choice of $\epsilon'$,
   $y_{i-1}'<y_i<\cdots<y_j<y_{j+1}'$.

   Let $T''$ be the triangulation obtained by removing, from $T$, the
   component of $T-\{x,y,z\}$ that contains some vertex of $f$.  Now we
   apply induction on $T''$ using the triangle $\Delta'$ and the sequences
   $r_{i-1},\ldots,r_{j+1}$ and $y_{i-1}',y_i,\ldots,y_{j},y_{j+1}'$.
   Combining the drawings of $T'$ and $T''$ yields a drawing of $T$
   that satisfies the requirements of the theorem.  Thus, we may assume
   that $T$ has no separating triangles.

   \paragraph{Contractible Edges:}
   If $T$ contains an edge $xy$ that is not on the boundary of any
   crossing face, then we contract the edge $xy$ to obtain a new vertex
   $u$ in a smaller triangulation $T'$.   We can then apply induction
   on $T'$ with the value $\epsilon'=\epsilon/2$ to obtain a drawing
   of $T'$ that satisfies all the conditions of the theorem with a smaller value of $\epsilon$.  In order
   to extend this to a drawing of $T$ we must show how to remove $u$
   and draw $x$ and $y$.

   Now


   \paragraph{The Base Case}
   Note that the case $n=3$ is trivial.

   We now apply
   induction on $T'$ to obtain a straight-line drawing of $T'$ in which




   Assume, without loss of generality that $f$ has exactly two vertices outside of $\bar{J}$.  We will prove something slightly stronger:  Let $\Delta$ be any triangle 

    

   Deform $C$ slightly in the neighbourhood of each $v_i$ to obtain a
   curve $C'$ that does not contain any vertex of $T$ but still intersects
   each crossing edge of $T$.  Let $e_1,\ldots,e_m$ denote the sequence
   of edges of $T$ intersected by $C'$ in the order they are intersected
   by $C'$, with the starting point chosen so that all edges of
   $e_1,\ldots,e_m$ incident to $v_i$ appear before all edges incident to
   $v_{i+1}$, for all $i\in\{1,\ldots,k-1\}$.

   

   

   Without loss of generality, assume that the outer face of $T$ has
   exactly one vertex in $\bar{C}$.  We will prove a stronger result:
   Let $a_1<a_2<\cdots<a_m$ be any increasing sequence of numbers.
   Let $\Delta$ be a triangle with to vertices to the left of the y-axis
   and one vertex to the right of the y-axis and whose two intersections
   with the y-axis are at $(0,a_1)$ and $(0,a_m)$.  We will show that
   $T$ has a straight-line drawing such that the three vertices of
   the outer face are the vertices of $\Delta$ and such that, for each
   $i\in\{1,\ldots,m\}$, the intersection of $e_i$ with the y-axis is
   at $(a_i,0)$.

    
\end{proof}



\section{Discussion}

Discuss all the consequences of \lemref{free} here\ldots
Basically, any graph class that has a collinear set of size $f(n)$ can 
be untangled while keeping $f(n)$ vertices fixed.  For such graph classes,
every point set of size $f(n)$ is a universal subset for $n$-vertex graphs
in the class.



\end{document}


