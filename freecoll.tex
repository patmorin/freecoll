\documentclass{patmorin}
\usepackage[utf8]{inputenc}
\usepackage{amsthm,amsmath,graphicx}
\usepackage{array}
\usepackage{pat}
\usepackage{hyperref}
\usepackage[dvipsnames]{xcolor}
\definecolor{linkblue}{named}{Blue}
\hypersetup{colorlinks=true, linkcolor=linkblue,  anchorcolor=linkblue,
citecolor=linkblue, filecolor=linkblue, menucolor=linkblue,
urlcolor=linkblue, pdfcreator=Me, pdfproducer=Me} \setlength{\parskip}{1ex}
\usepackage{tikz}

\listfiles
\newcommand{\lstlabel}[1]{\label{lst:#1}}
\newcommand{\lstref}[1]{Listing~\ref{lst:#1}}
\newcommand{\Lstref}[1]{\lstref{#1}}

\DeclareMathOperator{\block}{block}
\newcommand{\naive}{na\"{\i}ve}


\newcommand{\reals}{\mathbb{R}}
\newcommand{\integers}{\mathbb{Z}}
\newcommand{\naturals}{\mathbb{N}}
\newcommand{\dist}{{d}}

\title{\MakeUppercase{A Lemma on Order Types with Applications to Collinear Sets in Graph Drawing}\thanks{This research is partially funded by NSERC and the Ontario Ministry of Research and Innovation.}}

\author{Vida Dujmovi\'c,\thanks{Department of Computer Science and Electrical Engineering, University of Ottawa}\, and Pat Morin\thanks{School of Computer Science, Carleton University}}

\begin{document}
\maketitle


\begin{abstract}
  We define strong order types and $k$-strong order types and use them to
  show that any collinear set of vertices in a plane drawing of a graph
  $G$ is also a free collinear set (Dujmovi\'c and Frati 2016) of $G$.
\end{abstract}


\section{The Whole Thing}

Let $P=p_1,\ldots,p_n$ be a sequence of points (possibly with repetitions).  The \emph{full order type} of $P$ is a function $f_P\colon \{1,\ldots,n\}^3\mapsto \{0,1,\ldots,8\}$.

\end{document}


